\documentclass[11pt]{article}
\usepackage{portuges}
\usepackage[pdftex]{hyperref}
\hypersetup{pdfnewwindow=true}
\usepackage[latin1]{inputenc}
\title{CURRICULUM VITAE\\ 
\vspace{3cm}

\vspace{3cm}}
\author{Maria João Tinoco Varanda Pereira}
\date{\today} 


\begin{document} 

\maketitle 




\newpage
\section{Identifica\c{c}\~{a}o} 
\begin{quote} 
\begin{tabular}{ll}
\em{Nome:}          &  \bf{Maria João Tinoco Varanda Pereira}\\ 
\em{Data de nascimento:} & 9 de Novembro de 1971\\ 
\em{Cartão de Cidadão:}  & num. 9662973\\ 
\em{Nacionalidade:}  & Portuguesa\\
\em{Estado Civil:} & casada\\ 
\em{Resid\^{e}ncia:}  & Rua do Albergue, nº33\\ 
                   & 5300-305 Bragança\\ 
                   & Telef. 273 382009\\ 
\em{E-mail:} & mjoao@ipb.pt\\
\em{Página Pessoal:} & \href{http://www.ipb.pt/~mjoao}{http://www.ipb.pt/$\sim$mjoao}\\
\em{Local de Trabalho:} & Departamento de Informática e Comunicações,\\
&gab. 94\\ 
&Escola Superior de Tecnologia e Gestão\\
&de Bragança, Quinta de Sta. Apolónia,\\
&5300 Bragança\\  
&Telef. 273-303071\\
\em{Tempo de Serviço:} &20,5 anos (\href{run:Diplomas/carreira.pdf}{Comprovativo 1.1})\\ 
\em{Categoria:} &Professora Coordenadora desde março de 2017.\\
&(\href{run:NOVOSCOMPROVATIVOS/coordenador.pdf}{Comprovativo 1.2})\\ 
\end{tabular} 
\end{quote} 
\newpage

\section{Desempenho técnico-científico}
\subsection{Formação Académica} 
\begin{itemize} 
\item{ {\em{{Doutoramento em Informática}}}, na área da Tecnologia da Programação, na Universidade do Minho, finalizado em 2003, cuja dissertação tem o título "Sistematização da Animação de Programas" e a orientação ficou a cargo do Professor Doutor Pedro Rangel Henriques (\href{run:Diplomas/dout.pdf}{Comprovativo 2.1.1}) (\href{run:Publicacoes/publicacoes/8.pdf}{Tese}).}
\item{ {\em{{Mestrado em Informática}}}, na especialidade de Ciências da Computação, na Universidade do Minho, em 1996, cuja dissertação tem o título "Concepção e Especificação de uma Linguagem Visual" e a orientação ficou a cargo do Professor Doutor Pedro Rangel Henriques (\href{run:Diplomas/mest.pdf}{Comprovativo 2.1.2}) (\href{run:Publicacoes/publicacoes/1.pdf}{Tese}).}
\item{ {\em{{Licenciatura em Engenharia de Sistemas e Informática}}}, obtida na Universidade do Minho em 1994, com classificação final de 14 valores (\href{run:Diplomas/licenc.pdf}{Comprovativo 2.1.3}).}
\end{itemize}


\subsection{Qualidade e difusão dos resultados da atividade de investigação}

Nesta secção são listadas todas as publicações por ordem cronológica inversa, com ligação para o seu conteúdo integral, com ligação às bases de indexação (Comprovativos 2.2) e com indicação do número de citações ISI (Ci), número de citações Scopus (Cs) e número de auto-citações (AC) em cada um dos casos. As publicações sem referência a bases de indexação correspondem a artigos não indexados.

\subsubsection{Autoria de capítulos em livros/revistas científicos com arbitragem}
\begin{itemize}

\item{ Simone Bello Kaminski Aires, João Paulo Aires, Maria João Tinoco Varanda Pereira, Luís Manuel Alves, “Aplicando uma metodologia de aprendizagem colaborativa no ensino de programação”, Revista Online "A Plurivalência da Engenharia da Computação e seu Amplo Campo de Aplicação", 2021.}

\item{ Alves, P., Morais, C., Miranda, L., Pereira, M. (2020). Aprendizagem baseada em projetos: Implementação num curso de técnico superior profissional. In J. Ruiz-Rey, N. Quero-Torres, M. Cebrián-de-la-Serna, & P. Hernández-Hernández (Coord), (2020). Tecnologías emergentes y estilos de aprendizaje para la enseñanza (pp. 47-57). Malaga: Publicaciones Gtea 10 (ISBN: 978-84-09-16847-7).}

\item{ Pereira M.J.V., Oliveira N., Da Cruz D., Henriques P.R., {\bf{ {An effective way to teaching Language Processing Courses}}} para o livro {\bf{ {Innovative teaching strategies and New Learning Paradigms in Computer Programming}}, pp. 131-152, IGI Global, 2014. (\href{run:Publicacoes/PublicacoesSCOPUS.pdf}{SCOPUS})(\href{run:Publicacoes/publicacoes/78.pdf}{pdf 2.2.1}).}}


\end{itemize}

\subsubsection{Autoria de relatórios técnicos com ISBN}
\begin{itemize}
\item{ Barbu M., Vilanova R., Vicario J.L., Varanda M.J., Alves P., Popdora M., Prada M.A., Morán A., Torrebruno A., Marin S., Tocu R., {\bf{ {Data Mining Tool for Academic Data Exploitation}}}, First report of SPEET- Student Profile for Enhancing Enginering Tutoring, Erasmus$+$ Project with Universidad Autonoma de Barcelona, 2017 (ERASMUS$+$KA2/KA203)(ISBN 978-972-745-228-6).}
\item{ Vicario, Jose; Vilanova, Ramon; Bazzarelli, M.; Paganoni, Anna; Spagnolini, Umberto; Torrebruno, Aldo; Prada, Miguel; Morán, Antonio; Dominguez, Manuel; Pereira, Maria João; Alves, Paulo; Podpora, Michal; Barbu, Marian (2018). SPEET - Data mining tool for academic data exploitation: selection of most suitable algorithms (ERASMUS$+$KA2/KA203) (ISBN 978-989-20-8738-2).}
\item{ Prada, Miguel; Dominguez, Manuel; Morán, Antonio; Vilanova, Ramon; Vicario, Jose; Pereira, Maria João; Alves, Paulo; Podpora, Michal; Barbu, Marian; Torrebruno, Aldo; Spagnolini, Umberto; Paganoni, Anna (2018). Data mining tool for academic data exploitation: graphical data analysis and visualization. (ERASMUS$+$KA2/KA203) (ISBN 978-989-20-8739-9).}
\item{ Barbu, Marian; Vilanova, Ramon; Vicario, Jose; Pereira, Maria João; Alves, Paulo; Podpora, Michal; Kawala-Janik, A.; Prada, Miguel; Dominguez, Manuel; Spagnolini, Anna; Fontana, L. (2019). Data mining tool for academic data exploitation: publication report on engineering students profiles (ERASMUS$+$KA2/KA203) (ISBN 978-989-20-9286-7).}
\item{ IO5 }
\end{itemize}

\subsubsection{Autoria de artigos em revista de circulação internacional}
\begin{enumerate}

\item{ Using the Methodology Problem-Based Learning to Teaching Programming to Freshman Students, João Paulo Aires, Simone Aires, Maria João Varanda Pereira, Luís Alves, IJIET - International Journal of Information and Education Technology, Vol 13(3), pp. 448-455, Paper ID:  IJIET-5697, March 2023. (DOI: 10.18178/ijiet.2023.13.3.1825)}

\item{ Franco T, Sestrem L, Henriques PR, Alves P, Varanda Pereira MJ, Brandão D, Leitão P, Silva A. Motion Sensors for Knee Angle Recognition in Muscle Rehabilitation Solutions. Sensors. 2022; 22(19):7605. \url{https://doi.org/10.3390/s22197605} }

\item{ M. A. Prada et al., {\bf{ {Educational data mining for tutoring support in higher education: A web-based tool case study in engineering degrees}}} in IEEE Access, doi: 10.1109/ACCESS.2020.3040858.}

\item{ Luís Alves, Dušan Gajic, Pedro Rangel Henriques, Vladimir Ivancevic, Vladimir Ivkovic, Maksim Lalic, Ivan Lukovic, Maria João Varanda Pereira, Srdan Popov, Paula Correia Tavares, {\bf{ C Tutor Usage in Relation to Student Achievement and Progress: A Study of Introductory Programming Courses in Portugal and Serbia}}, Computer Applications in Engineering Education Journal, pp. 1-14, 2020. (SCOPUS)}

\item{ Joana Margarida Miguel, Maria João Varanda Pereira, Pedro Rangel Henriques, Mario Berón, {\bf{ {Assuring Data Privacy with PRIVAS - A Tool for Data Publishers}}}, IADIS International Journal on Computer Science and Information Systems, Vol. 14, No. 2, pp. 41-58, 2019 (ISSN: 1646-3692).(ISI) (\url{http://www.iadisportal.org/ijcsis/})}
\item{ Berón, M. M., Bernardis, H., Miranda, E. A., Riesco, D. E., Pereira, M. J. V., & Henriques, P. R. (2016), {\bf{ {Measuring the understandability of WSDL specifications, web service understanding degree approach and system}}}, Computer Science and Information Systems, 13(3), 779-807. doi:10.2298/CSIS160124026B. (SCOPUS) (\href{run:Publicacoes/publicacoes/93.pdf}{pdf 2.2.2.1})}
\item{ Pereira M.J.V., Fonseca J., Henriques P.R., {\bf{ Ontological approach for DSL development}}, Computer Languages, Systems \& Structures, Vol. 45, pp.35-52, Elsevier, 2016. (SCOPUS) (\href{run:Publicacoes/publicacoes/85.pdf}{pdf 2.2.2.1})}
\item{ Azcurra J., Berón M., Henriques P.R., Pereira M.J.V.,{\bf{ AId: Uma Ferramenta para Análise de Identificadores en Programas Java}}, Revista de Tecnología y Ciencia da Universidad Tecnológica Nacional, Buenos Aires, Number 27, pp. 17-32, Nov 2015, Edição Especial da revista com os melhores artigos da CoNaIISI 2014 - 2º Congreso Nacional de Ingeniería Informática/ Sistemas de Información, Argentina. (\href{run:Publicacoes/publicacoes/86.pdf}{pdf 2.2.2.2}) }
\item{ Berón M., Bernardis H., Miranda E., Riesco D., Pereira M.J.V., Henriques P.R., {\bf{ WSDLUD: A Metric to Measure the Understanding Degree of WSDL Descriptions}}, Languages, Applications and Technologies, Vol. 563, pp. 91-100, Series Communications in Computer and Information Science, Springer International Publishing, 2015. (\href{run:Publicacoes/ComprovativosISI.pdf}{ISI}, \href{run:Publicacoes/PublicacoesSCOPUS.pdf}{SCOPUS}, \href{run:Publicacoes/ComprovativosDBLP.pdf}{DBLP}) (\href{run:Publicacoes/publicacoes/81.pdf}{pdf 2.2.2.3}) }
\item{ Carvalho N., Almeida J.J., Henriques P.R., Pereira M.J.V., {\bf{ From Source Code Identifiers to Natural Language Terms}}, Journal of Systems and Software, Vol. 100, pp. 117-128, Feb 2015. (\href{run:Publicacoes/ComprovativosISI.pdf}{ISI}, \href{run:Publicacoes/PublicacoesSCOPUS.pdf}{SCOPUS}, \href{run:Publicacoes/ComprovativosDBLP.pdf}{DBLP}) (\href{run:Publicacoes/publicacoes/79.pdf}{pdf 2.2.2.4}) (Cs=Ci=2, AC=0)}
\item{ Carvalho N., Almeida J.J., Henriques P. R., Pereira M. J. V., {\bf{ CONCLAVE: Ontology-driven Measurement of Semantic Relatedness between Source Code Elements and Problem Domain Concepts}}, Computational Science and its Applications, Lecture Notes in Computer Science  8584, part VI, pp. 116-131, 2014. (\href{run:Publicacoes/ComprovativosISI.pdf}{ISI}, \href{run:Publicacoes/PublicacoesSCOPUS.pdf}{SCOPUS}, \href{run:Publicacoes/ComprovativosDBLP.pdf}{DBLP}) (\href{run:Publicacoes/publicacoes/74.pdf}{pdf 2.2.2.5})}
\item{ Harrison R., Da Cruz D., Henriques P.R., Pereira M.J.V., Liu S.H., Menzies T., Mernik M., Rodríguez D., {\bf{ Report from the first international workshop on realizing artificial intelligence synergies in software engineering (RAISE 2012)}}, ACM SIGSOFT Software Engineering Notes 37(5), pp. 34-35, 2012. (\href{run:Publicacoes/ComprovativosDBLP.pdf}{DBLP}) (\href{run:Publicacoes/RAISE2012.pdf}{pdf 2.2.2.6})}
\item{ Lukovic I., Pereira M.J.V., Oliveira N., Cruz D., Henriques P.R., {\bf{ A DSL for PIM Specifications: Design and Attribute Grammar based Implementation}}, ComSIS - Computer Science and Information Systems Journal, Vol. 8, N. 2, pp. 379-403, 2011. (\href{run:Publicacoes/ComprovativosISI.pdf}{ISI}, \href{run:Publicacoes/PublicacoesSCOPUS.pdf}{SCOPUS}, \href{run:Publicacoes/ComprovativosDBLP.pdf}{DBLP}) (Ci=6, AC=0),(Cs=10, AC=0) (\href{run:Publicacoes/publicacoes/55.pdf}{pdf 2.2.2.7})}
\item{ Kosar T., Oliveira N., Mernik M., Pereira M.J.V., Crepinsek M., Cruz D., Henriques P.R., {\bf{ Comparing General-Purpose and Domain-Specific Languages: An Empirical Study}}, ComSIS - Computer Science and Information Systems Journal, Special Issue on Compilers, Related Technologies and Applications, Vol. 7, Number 2, pp. 247-264, April 2010. (ISSN: 1820-0214) (\href{run:Publicacoes/ComprovativosISI.pdf}{ISI}, \href{run:Publicacoes/PublicacoesSCOPUS.pdf}{SCOPUS}, \href{run:Publicacoes/ComprovativosDBLP.pdf}{DBLP}) (Ci=35, AC=1) (Cs=67, AC=1) (\href{run:Publicacoes/publicacoes/49.pdf}{pdf 2.2.2.8})}
\item{ Oliveira N., Pereira M.J.V., Henriques P.R., Cruz D., Cramer B., {\bf{ VisualLISA: A Visual Environment to Develop Attribute Grammars}}, ComSIS - Computer Science and Information Systems Journal, Special Issue on Compilers, Related Technologies and Applications, Vol. 7, Number 2, pp. 265-290, April 2010. (ISSN: 1820-0214) (\href{run:Publicacoes/ComprovativosISI.pdf}{ISI}, \href{run:Publicacoes/PublicacoesSCOPUS.pdf}{SCOPUS}, \href{run:Publicacoes/ComprovativosDBLP.pdf}{DBLP}) (Ci=3, AC=1) (Cs=10, AC=2) (\href{run:Publicacoes/publicacoes/52.pdf}{pdf 2.2.2.9})}
\item{ Cruz D.,Béron M.,Henriques P.R., Pereira M.J.V., {\bf{ Code Inspection Approaches for Program Visualization}}, Acta Electrotechnica et Informatica, Faculty of Electrical Engineering and Informatics, Technical University of Kosice, Slovakia, Vol. 9, n 3, pp. 32-42, Jul-Sep 2009. (ISSN: 1335-8243) (\href{run:Publicacoes/publicacoes/40.pdf}{pdf 2.2.2.10})}
\item{ Pereira M.J.V., Mernik M., Cruz D., Henriques P.R., {\bf{ Program Comprehension for Domain-Specific Languages}}, ComSIS - Computer Science and Information Systems Journal, Special Issue on Compilers, Related Technologies and Applications, Volume 5, Number 2, pp. 1-17, Dec 2008. (ISSN: 1820-0214) (\href{run:Publicacoes/ComprovativosISI.pdf}{ISI}, \href{run:Publicacoes/PublicacoesSCOPUS.pdf}{SCOPUS}, \href{run:Publicacoes/ComprovativosDBLP.pdf}{DBLP}) (Ci=9, AC=1) (Cs=14,AC=3) (\href{run:Publicacoes/publicacoes/33.pdf}{pdf 2.2.2.11})}
\item{ Cruz D., Henriques P.R., Pereira M.J.V., {\bf{ Alma versus DDD}}, ComSIS - Computer Science and Information Systems Journal, Special Issue on Compilers, Related Technologies and Applications, Volume 5, Number 2, pp. 119-136, Dec 2008. (ISSN: 1820-0214) (\href{run:Publicacoes/ComprovativosISI.pdf}{ISI}, \href{run:Publicacoes/PublicacoesSCOPUS.pdf}{SCOPUS}, \href{run:Publicacoes/ComprovativosDBLP.pdf}{DBLP}) (\href{run:Publicacoes/publicacoes/34.pdf}{pdf 2.2.2.12})}
\item{ Cruz D., Henriques P.R., Pereira M.J.V., {\bf{ Constructing program animations using a pattern-based approach}}, ComSIS - Computer Science and Information Systems Journal, Special Issue on Advances in Programming Languages, Faculty of Technical Sciences, Novi Sad, Serbia, Volume 4, Number 2, pp. 99-116, Dec 2007. (ISSN: 1820-0214) (\href{run:Publicacoes/ComprovativosDBLP.pdf}{DBLP}) (\href{run:Publicacoes/publicacoes/25.pdf}{pdf 2.2.2.13})}
\item{ Rebernak D., Mernik M., Henriques P.R., Cruz D., Pereira M.J.V., {\bf{ Specifying Languages Using Aspect-oriented Approach: AspectLISA}}, Journal of Computing and Information Technology, pp. 343-350, Volume 14, Number 4, Dec 2006. (\href{run:Publicacoes/ComprovativosDBLP.pdf}{DBLP})(ISSN: 1330-1136) (\href{run:Publicacoes/publicacoes/20.pdf}{pdf 2.2.2.14})}
\item{ Rebernak D., Mernik M., Henriques P.R., Pereira M.J.V., {\bf{ AspectLISA: an aspect-oriented compiler construction system based on attribute grammars}}, Electronic Notes in Theoretical Computer Science, pp. 37-53, Volume 164, Issue 2, Oct 2006.(ISSN: 1571-0661) (\href{run:Publicacoes/PublicacoesSCOPUS.pdf}{SCOPUS}, \href{run:Publicacoes/ComprovativosDBLP.pdf}{DBLP}) (Cs=4, AC=0) (\href{run:Publicacoes/publicacoes/14.pdf}{pdf 2.2.2.15})}
\item{ Lopes R., Raimundo N., Varanda M.J., Oliveira J., Roque V., {\bf{ Executable Graphics for PBNM}}, Operations and Management in lp$-$Based Networks, Lecture Notes in Computer Science, Volume 3751, pp. 108-117, 2005. (\href{run:Publicacoes/ComprovativosISI.pdf}{ISI}, \href{run:Publicacoes/PublicacoesSCOPUS.pdf}{SCOPUS}) (ISSN 0302-9743) (\href{run:Publicacoes/publicacoes/83.pdf}{pdf 2.2.2.16})}
\item{ Henriques P.R., Pereira M.J.V., Mernik M., Lenic M., Gray J., Wu H., {\bf{ Automatic Generation of Language-based Tools using LISA System}}, IEE Software Journal, pp. 54-69, Volume 152, Issue 2, April 2005. (ISSN: 1462-5970) (\href{run:Publicacoes/ComprovativosISI.pdf}{ISI}, \href{run:Publicacoes/PublicacoesSCOPUS.pdf}{SCOPUS}, \href{run:Publicacoes/ComprovativosDBLP.pdf}{DBLP}) (Ci=28, AC=2)(Cs=48, AC=4) (\href{run:Publicacoes/publicacoes/11.pdf}{pdf 2.2.2.17})}
\item{ Kosar T., Mernik M., Henriques P.R., Pereira M.J.V., Zumer V., {\bf{ Software Development with Grammatical Approach}}, Informatica Journal, pp. 393-404, Volume 28, nº4, 2004. (ISSN: 1854-3871) (\href{run:Publicacoes/PublicacoesSCOPUS.pdf}{SCOPUS}, \href{run:Publicacoes/ComprovativosDBLP.pdf}{DBLP}) (Cs=3, AC=0) (\href{run:Publicacoes/publicacoes/10.pdf}{pdf 2.2.2.18}) }
\item{ Henriques P.R., Pereira M.J.V., Mernik M., Lenic M., {\bf{ Automatic Generation of Language-based Tools}},  Electronic Notes in Theoretical Computer Science, pp. 77-96, Volume 65, Issue 3, July 2002. (ISSN: 1571-0661) (\href{run:Publicacoes/PublicacoesSCOPUS.pdf}{SCOPUS}, \href{run:Publicacoes/ComprovativosDBLP.pdf}{DBLP}) (Cs=9, AC=2) (\href{run:Publicacoes/publicacoes/6.pdf}{pdf 2.2.2.19}) }

\end{enumerate}

\subsubsection{Publicações em Atas de Conferências Internacionais}
\begin{enumerate}

\item {AquaVitae: Innovating Personalized Meal Recommendation for Enhanced Nutritional Health, Henrique Marcuzzo, Maria João Varanda Pereira, Paulo Alves, Juliano Foleis, OL2A - International Conference on Optimization, Learning Algorithms and Applications, Setembro 2023, Ponta Delgada.}

\item {Characterization and Identification of Programming Languages, Júlio Alves, Alvaro Neto, Maria João Varanda Pereira, Pedro Henriques, SLATE 2023 - 12th Symposium on Languages, Applications and Technologies, Open Access Series in Informatics (OASIcs), Volume 113, 10.4230/OASIcs.SLATE.2023.13, 2023.}

\item {NLP/AI Based Techniques for Programming Exercises Generation, Tiago Freitas, Alvaro Neto, Maria João Varanda Pereira, Pedro Henriques, 4th International Computer Programming Education Conference (ICPEC 2023), Open Access Series in Informatics (OASIcs), vol 112, 10.4230/OASIcs.ICPEC.2023.9, 2023.}

\item {Franco, Tiago, Henriques, P. R., Alves, P., Varanda Pereira, M. J., Sestrem, L., Leitão, P., Silva, A., myHealth: a Mobile App for Home Muscle Rehabilitation, 10th International Conference on Serious Games and Applications for Health, Sidney, Australia, August, 2022. }

\item {Alvaro Neto, Cristiana Araújo, Maria João Varanda, Pedro Henriques, Value-Focused Investigation into Programming Languages Affinity, Third International Computer Programming Education Conference (ICPEC 2022), pp. 1-12, Open Access Series in Informatics (OASIcs), vol. 102, 2022. }

\item {P. Alves, C. Morais, L. Miranda, M.J. V. Pereira, J. Vaz (2022) Pedagogical Methodologies: impact on remote emergency teaching and use perspectives in higher education, INTED2022 Proceedings, pp. 1770-1779. }

\item {de Pinho, R., Pereira, M.J.V., Azevedo, A., Henriques, P.R. (2022). Relating Biometric Sensors with Serious Games Results. Information Systems and Technologies. WorldCIST 2022. Lecture Notes in Networks and Systems, vol 468. Springer, Cham. \url{https://doi.org/10.1007/978-3-031-04826-5_10} }

\item {Franco, Tiago, Henriques, P. R., Alves, P., Varanda Pereira, M. J., Pedrosa, T., Silva, F., Leitão, P., Oliveira, L. (2022). System Architecture for Home Muscle Rehabilitation Treatment. Information Systems and Technologies. WorldCIST 2022. Lecture Notes in Networks and Systems, vol 470. Springer, Cham. \url{https://doi.org/10.1007/978-3-031-04829-6_27} }


\item {Tiago Franco, Pedro R. Henriques, Paulo Alves, Maria João Varanda Pereira, {\bf {Approaches to Classify Knee Osteoarthritis Using Biomechanical Data}}, Communications in Computer and Information Science (1st International Conference on Optimization, Learning Algorithms and Applications, OL2A 2021), 2021, 1488 CCIS, pp. 417-429 (DOI:10.1007/978-3-030-91885-9$_$31).}

\item {Manuel Sousa, Maria João Varanda Pereira, Pedro Rangel Henriques, {\bf{ Lyntax – A Grammar-based Tool for Linguistics}},  10th Symposium on Languages, Applications and Technolo-gies (SLATE 2021).}

\item {João Paulo Aires, Simone Bello Kaminski, Maria João Varanda Pereira, Luís Alves, {\bf{ Active Methodologies in Incoming Programming Classes}}, Second International Computer Programming Education Conference (ICPEC 2021).}

\item {Diogo Soares, Maria João Varanda Pereira, Pedro Rangel Henriques, {\bf{ Integrating a Graph Builder into Python Tutor}}, Second International Computer Programming Education Conference (ICPEC 2021).}

\item {Alvaro Costa Neto, Cristiana Araújo, Maria João Varanda Pereira, and Pedro Rangel Henriques, {\bf{ Programmers' Affinity to Languages}}, Second International Computer Programming Education Conference (ICPEC 2021).}

\item {C. C. Hideo Nakai, J. Eduardo Fernandes and M. J. V. Pereira, {\bf{ Platform for Space Management in a Higher Education School}}, 16th Iberian Conference on Information Systems and Technolo-gies (CISTI), 2021, pp. 1-6.}

\item {P. Alves, C. Morais, L. Miranda, M. J. V. Pereira and J. Vaz, {\bf{ Digital tools in higher education in the context of Covid-19}}, 16th Iberian Conference on Information Systems and Technologies (CISTI), 2021, pp. 1-6.}

\item {Paulo Alves, Carlos Morais, Luísa Miranda, Maria João Varanda Pereira, Josiana Vaz, {\bf{ Remote Learning: Students’ Satisfaction and Perspectives in Higher Education}}, 20th European Conference on e-Learning - ECEL 2021, pp. 18-26 (DOI:10.34190/EEL.21.040)}

\item {Franco T., Alves P., Pedrosa T., Varanda Pereira M.J., Canão J., {\bf{ Implementation of Big Data Analytics Tool in a Higher Education Institution}}, Trends and Applications in Information Systems and Technologies, WorldCIST 2021, Advances in Intelligent Systems and Computing, vol 1365, pp. 207-216, Springer, 2021 (\url{https://doi.org/10.1007/978$-$3$-$030$-$72657$-$7} 20).}

\item {Renato Preigschadt de Azevedo, Maria João Varanda Pereira and Pedro Rangel Henriques, {\bf{ Development of Q\&A Systems using AcQA}}, SLATe 2020 - Symposium on Languages, Applications and Technologies, Julho 2020. }

\item {Mauro C. Argañaraz, Mario M. Berón, Maria J. Varanda Pereira, Pedro R. Henriques, {\bf{ Detection of Vulnerabilities in Smart Contracts Specifications in Ethereum Platforms}}, SLATe 2020 - Symposium on Languages, Applications and Technologies, Julho 2020.}
\item {Marcela Viana P. Almeida, Luís M. Alves, Maria João Varanda Pereira, Glívia Angélica R. Barbosa, {\bf{ EasyCoding – Methodology to Support Programming Learning}}, First International Computer Programming Education Conference (ICPEC 2020), pp. 1-8, Vol 81, OpenAccess Series in Informatics, DOI: 10.4230/OASIcs.ICPEC.2020.1 (SCOPUS)}

\item	{Alves, Paulo; Carlos, Morais; Luísa, Miranda; Pereira, Maria João, {\bf{ La metodología de aprendizaje basado en proyectos: estudio en un curso de desarrollo de software}} I Congreso Internacional Tecnologías Emergentes en Educación, Málaga, 2019. (resumo)}

\item {Joana Miguel, Maria João Varanda Pereira, Pedro Rangel Henriques, Mario Berón {\bf{ PRIVAS- Automatic Anonymization of Databases}}, 16th International Conference on Applied Computing, Cagliari, Italy, Nov 2019. }

\item {Luís Alves, Dušan Gajic,  Pedro Rangel Henriques, Vladimir Ivancevic, Maksim Lalic, Ivan Lukovic, Maria João Varanda Pereira, Srdan Popov, Paula Correia Tavares, {\bf{ Student Entrance Knowledge, Expectations and Motivation within Introductory Programming Courses in Portugal and Serbia}},  SEFI Annual Conference, Budapest, Hungary, September 2019. (SCOPUS 2020)}

\item {Martinho Aragão, Maria João Varanda Pereira, Pedro Rangel Henriques, {\bf{ Scaling up a Programmers’ Profile Tool}},  SLATe 2019 – Symposium on Languages, Applications and Technologies, Departamento de Matemática, Universidade de Coimbra, Junho 2019. (SCOPUS)}

\item {Vilanova, R. and Dominguez, M. and Vicario, J. and Prada, M.A. and Barbu, M. and Varanda, M.J. and Alves, P. and Podpora, M. and Spagnolini, U. and Paganoni, {\bf{ Data-driven tool for monitoring of students performance}}, IFAC-PapersOnLine, 2019, volume 52, number 9, pp. 190-195, doi: 10.1016/j.ifacol.2019.08.188. (SCOPUS)}

\item {R. Vilanova, J. Vicario, M. A. Prada, M. Barbu, M. Dominguez, M.J. Varanda, M. Podpora, U. Spagnolini, P. Alves, A. Paganoni, {\bf{ CHARACTERIZATION OF ENGINEERING STUDENT PROFILES AT EUROPEAN INSTITUTIONS BY USING SPEET IT-TOOL}}, EDULEARN 2019 - International Conference on Education and New Learning Technologies, Palma de Mallorca, Spain, July 2019. (ISI)}

\item {R. Vilanova, J. Vicario, M.A. Prada, M. Barbu, M. Dominguez, M.J. Varanda, M. Podpora, U. Spagnolini, P. Alves, A. Paganoni, {\bf{ SPEET: WEB BASED IT TOOL FOR ACADEMIC DATA ANALYSIS}}, INTED 2019 – International Technology, Education and Development Conference, Valencia, Spain, March 2019. (ISI)}

\item {P.Alves, C. Morais, L. Miranda, M. Pereira, {\bf{ Project Based Learning: Higher Education Student's Perceptions}} , INTED 2019 – International Technology, Education and Development Conference, Valencia, Spain, March 2019. (ISI)}

\item {Renato Azevedo, Maria João Varanda Pereira, Pedro Rangel Henriques, {\bf{ DSL based Automatic Generation of Q\&A Systems}},  WorldCIST 2019 – 7th World Conference on Information Systems and Technologies, La Toja Island, Galicia, Spain, April 2019. (SCOPUS)}

\item {Johnny Lima, Paulo Alves, Maria Pereira and Simone Almeida, {\bf{ Using Academic Analytics to Predict Dropout Risk in Engineering Courses}},  316-321, ECEL 2018 - 17th European Conference on E-Learning, Atenas, Greece. (SCOPUS)}

\item {Agustín Ferrari, Edgardo Bernardis, Mario Berón, Hernán Bernardis, Maria Joao Tinoco Varanda Pereira, Miguel Bustos, Daniel Riesco,ATENOS: Un Programa para Mejorar la Seguridad en WSDL, CoNaIISI 2018 – 6to Congreso Nacional de Ingeniería Informática – Sistemas de Información, Universidad CAECE – Mar del Plata, Buenos Aires, Argentina, Nov 2018, ISSN 2347-0372.}

\item {Luís Alves, Pedro Rangel Henriques, Vladimir Ivancevic, Maksim Lalic, Ivan Lukovic, Maria João Varanda Pereira, and Paula Correia Tavares, {\bf{ A Comparison of Introductory Programming Courses between Portugal and Serbia}}, AIIT 2018 – International Conference on Applied Internet and Information Technologies, Bitola, Macedonia, 5th October 2018.}

\item {R. Vilanova, J. Vicario, M. A. Prada, M. Barbu, M. Dominguez, M.J. Varanda, M. Podpora, U. Spagnolini, P. Alves, A. Paganoni, {\bf{ SPEET: Software Tools for Academic Data Analysis}}, EDULEARN 2018 - International Conference on Education and New Learning Technologies, Palma de Mallorca, Spain, July 2018. (ISI)}

\item {M. Dominguez, R. Vilanova, M.A. Prada, J. Vicario, M. Barbu, M. J.Varanda, M. Podpora, U. Spagnolini, P. Alves and A. Paganoni, {\bf{ SPEET: Visual Data Analysis of Engineering Students Performance from Academic Data}}, LASI 2018, Learning Analytics Summer Institutes, Universidad de Leon, Spain, June 2018. (SCOPUS)}

\item {Hernán Bernardis, Edgardo Bernardis, Mario M. Berón, Daniel E. Riesco, Maria Joao V. Pereira, {\bf{ Técnicas y Herramientas para Regular la Seguridad en Web Services Basados en WSDL}}, WICC 2018 -XX Workshop de Investigadores en Ciencias de la Computación, Corrientes, Argentina, Abril 2018.}

\item {Martini R., Araújo C., Henriques P., Pereira M.J.V., {\bf{ CaVa: An Example of the Automatic Generation of Virtual Learning Spaces}}, WorldCist'18 - 6th World Conference on Information Systems and Technologies, Naples, Italy, March 2018. (SCOPUS)}

\item {Azevedo R., Henriques P., Pereira M.J.V., {\bf{ Extending PythonQA with knowledge from StackOverflow}}, WorldCist'18 - 6th World Conference on Information Systems and Technologies, Naples, Italy, March 2018. (SCOPUS)}

\item {R. Vilanova, J. Vicario, M.A. Prada, M. Barbu, M. Dominguez, M.J. Varanda, M. Podpora, U. Spagnolini, P. Alves, A. Paganoni, {\bf{ SPEET: an International Collaborative Experience in Data Mining for Education}}, 10th annual International Conference of Education, Research and Innovation, pp 1648-1657, Seville, Spain, 16-18 November, 2017.(SCOPUS)}

\item {Tomeyan G., Pereira M.J.V., Margarov G., {\bf{ Plagiarism detection system for Armenian language}}, 11th International Conference on Computer Science and Information Technologies, Yerevan, Armenia, Setembro de 2017. (SCOPUS 2018)}

\item {Araújo C., Martini R., Henriques P., Pereira M.J.V., {\bf{  CAVAdsl - Criação de Espaços Virtuais de Aprendizagem a partir de um repositório de documentos XML}}, Simpósio de Humanidades Digitais do Sul: Escrita Criativa e Coleções Digitais, Universidad Complutense de Madrid, Setembro de 2017.}

\item {Ramos M., Pereira M.J.V., Henriques P., {\bf{ A QA System for learning Python}}, communication in WAPL 2017- 6th Workshop on Advances in Programming Languages integrado no FedCSIS (Federated Conference on Computer Science and Information Systems), Praga, Republica Checa, Setembro de 2017.}

\item {Novais D., Pereira M.J.V., Henriques P., {\bf{ Program Analysis for Clustering Programmers' Profile}}, short paper at WAPL 2017,- 6th Workshop on Advances in Programming Languages integrado no FedCSIS (Federated Conference on Computer Science and Information Systems), Praga, Republica Checa, Setembro de 2017. (SCOPUS and ISI)}

\item {Barros P., Pereira M.J.V., Henriques P., {\bf{ Applying Attribute Grammars to teach Linguistic Rules}}, SLATE 2017 - 6th Symposium on Languages, Applications and Technologies, OpenAccess Series in Informatics (OASIcs), ESMAD-Instituto Politécnico do Porto, Vila do Conde, Junho 2017. (SCOPUS)}

\item {Chuchulashvili M., Goziashvili N., Pereira M.J.V, Lopes R., {\bf{ Micro atividades para a Aprendizagem de Programação}}, CMEA 2016 - VII Congresso Mundial de Estilos de Aprendizagem, pp. 1503-1514, Instituto Politécnico de Bragança, Julho 2016. (\href{run:Publicacoes/publicacoes/91.pdf}{pdf 2.2.3.1}) (\url{http://hdl.handle.net/10198/12934}) (versão do editor em publicacoes2016)}
\item {Alves L., Balsa C., Pereira M.J.V., {\bf{ Simulador Gráfico de Algoritmos Matemáticos}}, CMEA 2016 - VII Congresso Mundial de Estilos de Aprendizagem, pp. 1553-1564, Instituto Politécnico de Bragança, Julho 2016. (\href{run:Publicacoes/publicacoes/90.pdf}{pdf 2.2.3.2}) (\url{http://hdl.handle.net/10198/12934}) (versão do editor em publicacoes2016)}
\item {Tavares P., Gomes E.F., Pereira M.J.V., Henriques P.R., {\bf{ Técnicas para aumentar o Envolvimento dos Alunos na Aprendizagem da Programação}}, CMEA 2016 - VII Congresso Mundial de Estilos de Aprendizagem, pp. 1565-1577, Instituto Politécnico de Bragança, Julho 2016. (\href{run:Publicacoes/publicacoes/89.pdf}{pdf 2.2.3.3}) (\url{http://hdl.handle.net/10198/12934}) (versão do editor em publicacoes2016)}
\item {Novais D., Pereira M.J.V., Henriques P.R., {\bf{ Profile detection through source code static analysis}}, SLATE 2016 - 5th Symposium on Languages, Applications and Technologies, OpenAccess Series in Informatics (OASIcs), vol 51, pp.1-13, June 2016. (SCOPUS) (\href{run:Publicacoes/publicacoes/88.pdf}{pdf 2.2.3.4})}
\item {Bernardis H., Bernardis E., Berón M., Riesco D., Henriques P.R., Pereira M.J.V.,{\bf{ Cálculo de Métricas para Medir el Grado de Entendimiento de una Descripción WSDL}}, WICC 2016 - XVIII Workshop de Investigadores en Ciencias de la Computación, Concordia, Entre Ríos, Argentina, Abril 2016. (\href{run:Publicacoes/publicacoes/87.pdf}{pdf 2.2.3.5})}
\item {Perez N.B., Berón M., Henriques P.R., Pereira M.J.V.,{\bf{ Comprensión de Sistemas Paralelos}}, WICC 2015 - XVII Workshop de Investigadores en Ciencias de la Computación, Salta, Argentina, Abril 2015. (\href{run:Publicacoes/publicacoes/80.pdf}{pdf 2.2.3.6})}
\item {Bernardis H., Bernardis E., Berón M., Riesco D., Henriques P.R., Pereira M.J.V., {\bf{ Técnicas y Estrategias para Compreender Procesos de Negocios Especificados en WS-BPEL}}, WICC 2015 - XVII Workshop de Investigadores en Ciencias de la Computación, Salta, Argentina, Abril 2015. (\href{run:Publicacoes/publicacoes/82.pdf}{pdf 2.2.3.7})}
\item {Azcurra J., Berón M., Montejano G., Farnese A., Henriques P.R., Pereira M.J.V., {\bf{ AId: Uma Ferramenta para Análise de Identificadores de Programas Java}}, CoNaIISI 2014 - 2º Congreso Nacional de Ingeniería Informática/ Sistemas de Información, San Luis, Argentina, Nov 2014. (\href{run:Publicacoes/publicacoes/77.pdf}{pdf 2.2.3.8})}
\item {Pereira N., Pereira M.J.V., Henriques P.R., {\bf{ Comment-based Concept Location over System Dependency Graphs}}, SLATE 2014 - 3rd Symposium on Languages, Applications and Technologies, Instituto Politécnico de Bragança, Junho 2014, OpenAccess Series in Informatics (OASIcs), pp. 51-58, vol 38, 2014. (ISBN 978-3-939897-68-2) (ISSN 2190-6807) (\href{run:Publicacoes/PublicacoesSCOPUS.pdf}{SCOPUS}, \href{run:Publicacoes/ComprovativosDBLP.pdf}{DBLP}) (\href{run:Publicacoes/publicacoes/75.pdf}{pdf 2.2.3.9})}
\item {Carvalho N.R., Almeida J.J., Pereira M.J.V., Henriques P.R., {\bf{ Conclave: Writing Programs to Understand Programs}}, SLATE 2014 - 3rd Symposium on Languages, Applications and Technologies, Instituto Politécnico de Bragança, Junho 2014, OpenAccess Series in Informatics (OASIcs),  pp.19-34, vol 38, 2014. (ISBN 978-3-939897-68-2) (ISSN 2190-6807) (\href{run:Publicacoes/PublicacoesSCOPUS.pdf}{SCOPUS}, \href{run:Publicacoes/ComprovativosDBLP.pdf}{DBLP}) (\href{run:Publicacoes/publicacoes/73.pdf}{pdf 2.2.3.10})}
\item {Fonseca J., Pereira M.J.V., Henriques P.R., {\bf{ Converting Ontologies into DSLs}}, SLATE 2014 - 3rd Symposium on Languages, Applications and Technologies, Instituto Politécnico de Bragança, Junho 2014, OpenAccess Series in Informatics (OASIcs), pp. 85-92, vol 38, 2014. (ISBN 978-3-939897-68-2) (ISSN 2190-6807) (\href{run:Publicacoes/PublicacoesSCOPUS.pdf}{SCOPUS}, \href{run:Publicacoes/ComprovativosDBLP.pdf}{DBLP}) (\href{run:Publicacoes/publicacoes/76.pdf}{pdf 2.2.3.11})}
\item {Pereira M.J.V., Oliveira N., Cruz D., Henriques P.R., {\bf{ Choosing Grammars to Support Language Processing Courses}}, SLATE 2013 - Symposium on Languages, Applications and Technologies, Faculdade de Ciências da Universidade do Porto, Junho de 2013, OpenAccess Series in Informatics (OASIcs), pp. 155-168, vol 29, 2013. (\href{run:Publicacoes/PublicacoesSCOPUS.pdf}{SCOPUS}, \href{run:Publicacoes/ComprovativosDBLP.pdf}{DBLP}) (\href{run:Publicacoes/publicacoes/71.pdf}{pdf 2.2.3.12})}
\item {Miranda E., Berón M., Montejano G., Pereira M.J.V., Henriques P.R., {\bf{ NESSy: a New Evaluator for Software Development Tools}}, SLATE 2013 - Symposium on Languages, Applications and Technologies, Faculdade de Ciências da Universidade do Porto, Portugal, Junho de 2013, OpenAccess Series in Informatics (OASIcs), pp. 21-37, vol 29, 2013. (\href{run:Publicacoes/PublicacoesSCOPUS.pdf}{SCOPUS}, \href{run:Publicacoes/ComprovativosDBLP.pdf}{DBLP}) (\href{run:Publicacoes/publicacoes/72.pdf}{pdf 2.2.3.13})}
\item {Oliveira N., Pereira M.J.V., Gancarski A.L., Henriques P.R., {\bf{ Learning Spaces for Knowledge Generation}}, SLATE 2012 - Symposium on Languages, Applications and Technologies, Universidade do Minho, Portugal, Junho de 2012. (\href{run:Publicacoes/ComprovativosDBLP.pdf}{DBLP}) (\href{run:Publicacoes/publicacoes/69.pdf}{pdf 2.2.3.14})}
\item {Pereira M.J.V., Berón M., Cruz D., Oliveira N., Henriques P.R., {\bf{ Problem Domain Oriented Approach for Program Comprehension}}, SLATE 2012 - Symposium on Languages, Applications and Technologies, Universidade do Minho, Portugal, Junho de 2012. (\href{run:Publicacoes/ComprovativosDBLP.pdf}{DBLP}) (\href{run:Publicacoes/publicacoes/68.pdf}{pdf 2.2.3.15})}
\item {Carvalho N.R., Almeida J.J., Pereira M.J.V., Henriques P.R., {\bf{ Probabilistic SynSet Based Concept Location}}, SLATE 2012 - Symposium on Languages, Applications and Technologies, Universidade do Minho, Portugal, Junho de 2012. (\href{run:Publicacoes/ComprovativosDBLP.pdf}{DBLP}) (\href{run:Publicacoes/publicacoes/64.pdf}{pdf 2.2.3.16})}
\item {Balsa C., Alves L., Pereira M.J.V., Rodrigues P.J. and Lopes R., {\bf{ Graphical Simulation of Numerical Algorithms, An approach based on code instrumentation and java technologies}}, CSEDU 2012 - 4th International Conference on Computer Supported Education, Porto, pp. 164-169, Abril 2012. (\href{run:Publicacoes/PublicacoesSCOPUS.pdf}{SCOPUS}) (\href{run:Publicacoes/publicacoes/66.pdf}{pdf 2.2.3.17})}
\item {Miranda E., Berón M., Montejano G., Peralta M., Pereira M.J.V., {\bf{ Visualización de Software Orientada a Comprensión de Programas}}, WICC 2012 - XIV Workshop de Investigadores en Ciencias de la Computación, Misiones, Argentina, Abril de 2012. (\href{run:Publicacoes/publicacoes/67.pdf}{pdf 2.2.3.18})}
\item {Azcurra J., Berón M., Henriques P.R., Pereira M.J.V., {\bf{ Análisis de Información Informal para Facilitar la Comprensión de Programas}}, WICC 2012 - XIV Workshop de Investigadores en Ciencias de la Computación, Misiones, Argentina, Abril de 2012. (\href{run:Publicacoes/publicacoes/65.pdf}{pdf 2.2.3.19})}
\item {Matkovic M., Berón M., Salgado C., Henriques P.R., Pereira M.J.V., {\bf{ Construcción de Representaciones Innovadoras del Dominio del Programa para Facilitar la Comprensión de Programas}}, WICC 2012 - XIV Workshop de Investigadores en Ciencias de la Computación, Misiones, Argentina, Abril de 2012. (\href{run:Publicacoes/publicacoes/70.pdf}{pdf 2.2.3.20})}
\item {Carvalho N.R., Simões A.M., Almeida J.J., Henriques P.R., Pereira M.J.V., {\bf{ PFTL: A Systematic Approach For Describing Filesystem Tree Processors}}, INForum'11 --- Simpósio de Informática (CoRTA'11 track), Universidade de Coimbra, Setembro 2011. (\href{run:Publicacoes/publicacoes/56.pdf}{pdf 2.2.3.21})}
\item {Ceh I., Crepinsek M., Kosar T., Mernik M., Henriques P.R., Pereira M.J.V., Cruz D., Oliveira N.,{\bf{ Tool supported building of {DSL}}s from {OWL} ontologies}, INForum'11 --- Simpósio de Informática (CoRTA'11 track), Universidade de Coimbra, Setembro 2011. (\href{run:Publicacoes/publicacoes/59.pdf}{pdf 2.2.3.22})}
\item {Aristiaran M.,  Berón M., Henriques P.R., Pereira M.J.V., {\bf{ Clasificaciones: un mecanismo de herencia múltiple para la construcción de modelos fáciles de comprender y mantener}} , WICC 2011 - XIII Workshop de Investigadores en Ciencias de la Computación, Rosario, Argentina, Maio 2011. (\href{run:Publicacoes/publicacoes/60.pdf}{pdf 2.2.3.23})}
\item {Albanes J.,  Berón M. , Henriques P.R., Pereira M.J.V., {\bf{ Estrategias para relacionar el dominio del problema con el dominio del programa para la comprensión de programas}} , WICC 2011 - XIII Workshop de Investigadores en Ciencias de la Computación, Rosario, Argentina, Maio 2011. (\href{run:Publicacoes/publicacoes/57.pdf}{pdf 2.2.3.24})}
\item {Miranda E., Berón M., Montejano G., Riesco D., Henriques P.R., Pereira M.J.V., {\bf{ Visualización de software: conceptos, métodos y técnicas para facilitar la comprensión de programas}} , WICC 2011 - XIII Workshop de Investigadores en Ciencias de la Computación, Rosario, Argentina, Maio 2011. (\href{run:Publicacoes/publicacoes/58.pdf}{pdf 2.2.3.25})}
\item {Bernardis H.,  Berón M. , Riesco D., Henriques P.R., Pereira M.J.V., {\bf{ Extracción de información dinámica en programación orientada a objetos (Java)}} , XIII Workshop de Investigadores en Ciencias de la Computación, Rosario, Argentina, Maio 2011. (\href{run:Publicacoes/publicacoes/62.pdf}{pdf 2.2.3.26})}
\item {El Kadre I.,  Berón M. , Salgado C., Peralta M., Henriques P.R., Pereira M.J.V., {\bf{ Construcción de representaciones del dominio del programa para facilitar la interconexión de dominios}} , WICC 2011 - XIII Workshop de Investigadores en Ciencias de la Computación, Rosario, Argentina, Maio 2011. (\href{run:Publicacoes/publicacoes/61.pdf}{pdf 2.2.3.27})}
\item {Luzza M.,  Berón M., Montejano G., Henriques P.R., Pereira M.J.V., {\bf{ Diseño y Construcción de Lenguajes Específicos del Dominio}} , WICC 2011 - XIII Workshop de Investigadores en Ciencias de la Computación, Rosario, Argentina, Maio 2011. (\href{run:Publicacoes/publicacoes/63.pdf}{pdf 2.2.3.28})}
\item {Balsa C., Alves L., Pereira M.J.V., Rodrigues P., {\bf{ Graphical simulator of mathematical algorithms (GraSMA)}}, Proceedings of IASK International Conference Teaching and Learning, pp. 594-600, Seville, Spain, Dec 2010. (\href{run:Publicacoes/publicacoes/84.pdf}{pdf 2.2.3.29})}
\item {Lukovic I., Pereira M.J.V., Oliveira N., Cruz D., Henriques P.R., {\bf{ An Attribute Grammar Specification of IIS*Case PIM Concepts}}, MDASD 2010 - Workshop on Model Driven Approaches in System Development integrado no ADBIS 2010 14th East-European Conference on Advances in Databases and Information Systems, Novi Sad, Setembro 2010, CEUR Workshop Proceedings, 639, pp. 110-124. (\href{run:Publicacoes/PublicacoesSCOPUS.pdf}{SCOPUS}, \href{run:Publicacoes/ComprovativosDBLP.pdf}{DBLP}) (\href{run:Publicacoes/publicacoes/50.pdf}{pdf 2.2.3.30})}
\item {Béron M., Pereira M.J.V., Oliveira N., Cruz D., {\bf{ SVS, BORS, SVSi: Three Strategies to relate Problem and Program Domains}}, ICPC 2010 - 18th IEEE International Conference on Program Comprehension, Braga, 2010. (\href{run:Publicacoes/PublicacoesSCOPUS.pdf}{SCOPUS}, \href{run:Publicacoes/ComprovativosDBLP.pdf}{DBLP}) (\href{run:Publicacoes/publicacoes/53.pdf}{pdf 2.2.3.31})}
\item {Oliveira N., Pereira M.J.V., Cruz D., Béron M., {\bf{ Influence of Synchronized Domain Visualizations on Program Comprehension}}, Working Session, ICPC 2010 - 18th IEEE International Conference on Program Comprehension, Braga, 2010. (\href{run:Publicacoes/PublicacoesSCOPUS.pdf}{SCOPUS}, \href{run:Publicacoes/ComprovativosDBLP.pdf}{DBLP}) (\href{run:Publicacoes/publicacoes/54.pdf}{pdf 2.2.3.32})}
\item {Berón M., Uzal R., Henriques P.R., Pereira M.J.V., {\bf{ Estrategias para Facilitar la Comprensión de Programas}}, WICC 2010 - XII Workshop de Investigadores en Ciencias de la Computación, Universidad Nacional de la Patagonia San Juan Bosco, Argentina. (\href{run:Publicacoes/publicacoes/51.pdf}{pdf 2.2.3.33})}
\item {Oliveira N., Henriques P.R.,Cruz D., Pereira M.J.V., Mernik M., Kosar T., Crepinsek M., {\bf{ Applying Program Comprehension Techniques to Karel Robot Programs}}, WAPL'09 - 2nd Workshop on Advances in Programming Languages, Mragowo, Poland, October 2009. (\href{run:Publicacoes/PublicacoesSCOPUS.pdf}{SCOPUS}, \href{run:Publicacoes/ComprovativosDBLP.pdf}{DBLP}) (\href{run:Publicacoes/publicacoes/39.pdf}{pdf 2.2.3.34})}
\item {Oliveira N., Henriques P.R., Cruz D., Pereira M.J.V., {\bf{ VisualLISA: Visual Programming Environment for Attribute Grammars Specification}}, IMCSIT, WAPL'09 - 2nd Workshop on Advances in Programming Languages, Mragowo, Poland, October 2009. (\href{run:Publicacoes/PublicacoesSCOPUS.pdf}{SCOPUS}, \href{run:Publicacoes/ComprovativosDBLP.pdf}{DBLP}) (\href{run:Publicacoes/publicacoes/44.pdf}{pdf 2.2.3.35})}
\item {Kosar T., Mernik M., Crepinsek M., Henriques P.R., Cruz D., Pereira M.J.V., Oliveira N., {\bf{ Influence of domain-specific notation to program understanding}}, IMCSIT, WAPL'09 - 2nd Workshop on Advances in Programming Languages, Mragowo, Poland, October 2009. (\href{run:Publicacoes/PublicacoesSCOPUS.pdf}{SCOPUS}, \href{run:Publicacoes/ComprovativosDBLP.pdf}{DBLP}) (Cs=2, AC=0) (\href{run:Publicacoes/publicacoes/43.pdf}{pdf2.2.3.36})}
\item {Béron M., Henriques P.R., Pereira M.J.V., Uzal R., {\bf{ Instrumentaciones de Programas Escritos en C para Interrelacionar las Vistas Comportamental y Operacional de los Sistemas de Software}}, CACIC'09 - XV Argentine Congress on Computer Science, Universidad Nacional de Jujuy, Argentina, October 2009. (\href{run:Publicacoes/publicacoes/42.pdf}{pdf 2.2.3.37})}
\item {Oliveira N., Pereira M.J.V., Henriques P.R., Cruz D., {\bf{ Visualization of Domain Specific Program's Behavior}}, 5th IEEE International Workshop on Visualizing Software for Understanding and Analysis (co-located with ICSM 2009), Edmouton, Canada, September 2009. (ISBN: 978-1-4244-5027-5) (\href{run:Publicacoes/ComprovativosISI.pdf}{ISI}, \href{run:Publicacoes/PublicacoesSCOPUS.pdf}{SCOPUS}, \href{run:Publicacoes/ComprovativosDBLP.pdf}{DBLP}) (Cs=1,AC=1)(\href{run:Publicacoes/publicacoes/48.pdf}{pdf 2.2.3.38})}
\item {Béron M., Henriques P.R., Pereira M.J.V., Uzal R., {\bf{ Simplificando la Comprensión de Programas a través de la Interconnexión de Dominios}}, CACIC'08 - XIV Argentine Congress on Computer Science, Universidad Nacional de Chilecito, La Rioja, Argentina, Outubro 2008. (\href{run:Publicacoes/publicacoes/36.pdf}{pdf2.2.3.39 })}
\item {Cruz D., Henriques P.R., Pereira M.J.V., {\bf{ Strategies for Program Inspection and Visualization}}, CSE'08 - International Scientific Conference on Computer Science and Engineering, Kosice, Slovakia, September 2008. (\href{run:Publicacoes/publicacoes/31.pdf}{pdf 2.2.3.40})}
\item {Fonseca R., Cruz D., Henriques P.R., Pereira M.J.V., {\bf{ How to interconnect operational and behavioral views of web applications}}, ICPC'08 - IEEE International Conference on Program Comprehension, June 2008. (ISBN: 978-0-7695-3176-2) (\href{run:Publicacoes/ComprovativosISI.pdf}{ISI}, \href{run:Publicacoes/PublicacoesSCOPUS.pdf}{SCOPUS}, \href{run:Publicacoes/ComprovativosDBLP.pdf}{DBLP}) (\href{run:Publicacoes/publicacoes/38.pdf}{pdf 2.2.3.41})}
\item {Berón M., Henriques P.R., Pereira M.J.V., Uzal R., {\bf{ Inspección de Código para relacionar los Dominios del Problema y Programa para la Comprensión de Programas}}, WICC 2008 - X Workshop de Investigadores en Ciencias de la Computación, La Pampa, Argentina, 2008. (\href{run:Publicacoes/publicacoes/32.pdf}{pdf 2.2.3.42})}
\item {Berón M., Henriques P.R., Pereira M.J.V., Uzal R., {\bf{ Program Inspection to Interconnect Behavioral and Operational View for Program Comprehension}}, York Doctoral Symposium on Computing on Computing, University of York, UK, 2007. (\href{run:Publicacoes/publicacoes/28.pdf}{pdf 2.2.3.43})}
\item {Béron M., Henriques P.R., Pereira M.J.V., Uzal R., {\bf{ PICS un Sistema de Comprensión e Inspección de Programas}} , CACIC'07 - XIII Argentine Congress on Computer Science, RedUNCI - Red de Universidades con Carreras en Informática, Universidad Nacional de San Luis, Argentina, 2007. (\href{run:Publicacoes/publicacoes/22.pdf}{pdf 2.2.3.44})}
\item {Cruz D., Henriques P.R., Pereira M.J.V., {\bf{ Pattern-based Program Visualization}}, WAPL'07 - 1st Workshop on Advances in Programming Languages integrado no International Multiconference on Computer Science and Information Technology, Wisla, Polónia, Outubro de 2007. (\href{run:Publicacoes/publicacoes/21.pdf}{pdf 2.2.3.45})}
\item {Béron M., Uzal R., Henriques P.R., Pereira M.J.V., {\bf{ Compreensión de Programas por Inspección Visual y Animación}}, WICC 2007 - IX Workshop de Investigadores en Ciencias de la Computación, Chubut, Argentina, 2007. (\href{run:Publicacoes/publicacoes/27.pdf}{pdf 2.2.3.46})}
\item {Béron M., Henriques P.R., Pereira M.J.V., Uzal R., {\bf{ Static and Dynamic Strategies to Understand C Programs by Code Annotation}} , OpenCert 2007 - 1st International Workshop on Fondations and Techniques for Open Source Software Certification (integrado no ETAPS-European Joint Conferences on Theory and Practice of Software), Braga, 2007. (\href{run:Publicacoes/publicacoes/24.pdf}{pdf 2.2.3.47})}
\item {Béron M., Henriques P.R., Pereira M.J.V., Uzal R., Montejano G., {\bf{ A Language Processing Tool for Program Comprehension}} , CACIC'06 - XII Argentine Congress on Computer Science, Universidad Nacional de San Luis, Argentina, 2006. (\href{run:Publicacoes/publicacoes/15.pdf}{pdf 2.2.3.48})}
\item {Béron M., Henriques P.R., Pereira M.J.V., Uzal R., {\bf{ Comprensión de Algoritmos de Ruteo}}, CLEI 2006 - XXXII Conferencia Latinoamericana de Informática, Santiago de Chile, 2006. (\href{run:Publicacoes/publicacoes/18.pdf}{pdf 2.2.3.49})}
\item {Rebernak D., Mernik M., Henriques P.R., Pereira M.J.V., Cruz D., {\bf{ Specifying Languages using Aspect-oriented Approach: AspectLISA}}, ITI'06 - 28th International Conference Information Technology Interfaces, Cavtat, Croacia, 2006. (ISSN: 1330-1136) (\href{run:Publicacoes/ComprovativosISI.pdf}{ISI}, \href{run:Publicacoes/PublicacoesSCOPUS.pdf}{SCOPUS}) (\href{run:Publicacoes/publicacoes/19.pdf}{pdf 2.2.3.50})}
\item {Béron M., Henriques P.R., Pereira M.J.V., Uzal R., {\bf{ Herramientas para la compresión de programas}}, WICC 2006 - VIII Workshop de Investigadores en Ciencias de la Computación, Universidad de Morón, Bs. As., Argentina, 2006. (\href{run:Publicacoes/publicacoes/17.pdf}{pdf 2.2.3.51})}
\item {Pereira M. J.V., Henriques P.R., {\bf{ Visualization / Animation of Programs in Alma: obtaining different results}}, VMSE2003 - Symposium on Visual and Multimedia Software Engineering integrado no HCC'03-IEEE Symposium Human Centric Computing Languages and Environments,pp. 260-262, Auckland, New Zealand, 2003. (\href{run:Publicacoes/ComprovativosISI.pdf}{ISI},DBLP, \href{run:Publicacoes/PublicacoesSCOPUS.pdf}{SCOPUS}, \href{run:Publicacoes/ComprovativosDBLP.pdf}{DBLP}) (\href{run:Publicacoes/publicacoes/9.pdf}{pdf 2.2.3.52})}
\item {Henriques P.R., Kosar T., Mernik M., Pereira M.J.V., Zumer V., {\bf{ Grammatical Approach to Problem Solving}}, ITI'03 - 25th International Conference on Information Technology Interfaces, Cavtat, Croatia, 2003. (\href{run:Publicacoes/ComprovativosISI.pdf}{ISI}, \href{run:Publicacoes/PublicacoesSCOPUS.pdf}{SCOPUS}) (Cs=3,AC=0) (\href{run:Publicacoes/publicacoes/7.pdf}{pdf 2.2.3.53})}
\item {Pereira M.J.V., Henriques P.R., {\bf{ Visualization / Animation of Programs based on Abstract Representations and Formal Mappings}}, HCC'01 - 2001 Symposia on Human-Centric Computing Languages and Environments,pp. 373-381, Stresa, Italy, 2001. (\href{run:Publicacoes/ComprovativosISI.pdf}{ISI}, \href{run:Publicacoes/PublicacoesSCOPUS.pdf}{SCOPUS}, \href{run:Publicacoes/ComprovativosDBLP.pdf}{DBLP}) (\href{run:Publicacoes/publicacoes/5.pdf}{pdf 2.2.3.54})}
\item {Pereira M.J.V., Henriques P.R., {\bf{ Visualização Sistemática de Programas}}, SBLP 2000 - IV Simpósio Brasileiro de Linguagens de Programação, Recife, Brasil, 2000. (\href{run:Publicacoes/publicacoes/4.pdf}{pdf 2.2.3.55})}
\item {Pereira M.J.V., Rocha J., Henriques P.R., {\bf{ Concepção, Especificação e Implementação de Processadores de Linguagens Visuais}}, SBLP 1997 - II Simpósio Brasileiro de Linguagens de Programação, Campinas, Brasil, 1997. (\href{run:Publicacoes/publicacoes/2.pdf}{pdf 2.2.3.56})}
\end{enumerate} 

\subsubsection{Publicações em Atas de Conferências Nacionais}
\begin{enumerate}

\item {Thermal-Based Nutritional Recommendations AquaVitae System, Henrique Marcuzzo, Maria João Varanda Pereira, Paulo Alves, Juliano Foleis, SASYRS 2023 - Symposium of Applied Science for Young Researchers, Julho 2023, Barcelos. (poster)}

\item {Oliveira N., Pereira M.J.V., Henriques P.R., Cruz D., {\bf{ Domain Specific Languages: A Theoretical Survey}}, CoRTA'09 (INForum'09 -Simpósio de Informática), Faculdade de Ciências da Universidade de Lisboa, pp. 35-46, Setembro 2009. (\href{run:Publicacoes/publicacoes/45.pdf}{pdf 2.2.4.1})}
\item {Oliveira N., Pereira M.J.V., Henriques P.R., Cruz D., Cramer B., {\bf{ VisualLISA: A Domain Specific Visual Language for Attribute Grammars}}, CoRTA'09 (INForum'09 --- Simpósio de Informática), Faculdade de Ciências da Universidade de Lisboa, pp. 155-167, Setembro 2009. (\href{run:Publicacoes/publicacoes/47.pdf}{pdf 2.2.4.2})}
\item {Mernik M., Kosar T., Crepinsek M., Henriques P.R., Cruz D., Pereira M.J.V., Oliveira N., {\bf{ Comparison of XAML and C\# Forms using Cognitive Dimensions Framework}}, CoRTA'09 (INForum'09 --- Simpósio de Informática), Faculdade de Ciências da Universidade de Lisboa, pp. 180-191, Setembro 2009. (\href{run:Publicacoes/publicacoes/41.pdf}{pdf 2.2.4.3})}
\item {Oliveira N., Henriques P.R., Cruz D., Pereira M.J.V., {\bf{ XAGra - An XML Dialect for Attribute Grammars}}, XATA'09 (INForum'09 -Simpósio de Informática), Faculdade de Ciências da Universidade de Lisboa, pp. 409-420, Setembro 2009. (\href{run:Publicacoes/publicacoes/46.pdf}{pdf 2.2.4.4})}
\item {Béron M., Cruz D., Pereira M.J.V., Henriques P.R., Uzal R., {\bf{ Evaluation Criteria of Software Visualization Systems used for Program Comprehension}}, Interacção'08 - 3ª Conferência Interacção Pessoa\-Máquina, Universidade de Évora, Outubro 2008. (\href{run:Publicacoes/publicacoes/37.pdf}{pdf 2.2.4.5})}
\item {Pereira M.J.V., Mernik M., Cruz D., Henriques P.R., {\bf{ Program Comprehension for Domain-Specific Languages}}, CoRTA'08 - Compilers, Related Technologies and Applications, Julho 2008. (\href{run:Publicacoes/publicacoes/29.pdf}{pdf 2.2.4.6})}
\item {Pereira M.J.V., Mernik M., Cruz D., Henriques P.R., {\bf{ VisualLISA: a Visual Interface for an Attribute Grammar based Compiler-Compiler}}, CoRTA'08 - Compilers, Related Technologies and Applications, Julho 2008. (\href{run:Publicacoes/publicacoes/35.pdf}{pdf 2.2.4.7})}
\item {Cruz D., Henriques P.R., Pereira M.J.V., {\bf{ Alma versus DDD}}, CoRTA'08 - Compilers, Related Technologies and Applications, Julho 2008. (\href{run:Publicacoes/publicacoes/92.pdf}{pdf 2.2.4.8})}
\item {Cruz D., Henriques P.R., Pereira M.J.V., {\bf{ Exploring and Visualizing the "Alma" of XML Documents}}, XATA 2008 - XML: Aplicações e Tecnologias Associadas, Universidade de Évora, Fevereiro 2008. (\href{run:Publicacoes/publicacoes/30.pdf}{pdf 2.2.4.9})}
\item {Berón M., Henriques P.R., Pereira M.J.V., Uzal R., {\bf{ Técnicas de Inspecção de Programas para Inter-Relacionar as Vistas Comportamental e Operacional}}, Encontro Português de Inteligência Artificial, Guimarães, 2007. (\href{run:Publicacoes/publicacoes/23.pdf}{pdf 2.2.4.10})}
\item {Cruz D., Pereira M.J.V., Berón M., Fonseca R. and Henriques P.R., {\bf{ Comparing Generators for Language-based Tools}}, CORTA'07 - Compiler, Related Technologies and Applications, Universidade da Beira Interior, Julho 2007. (\href{run:Publicacoes/publicacoes/26.pdf}{pdf 2.2.4.11}) }
\item {Oliveira E., Henriques P.R., Pereira M.J.V., {\bf{ Proposta de um Sistema para Compreensão de Aplicações Web}}, CAPSI 2007 - 7ª Conferência da Associação Portuguesa de Sistemas de Informação, Aveiro, Portugal, 2007. (\href{run:Publicacoes/publicacoes/16.pdf}{pdf 2.2.4.12})}
\item {Béron M., Henriques P.R., Pereira M.J.V., {\bf{ A System to Evaluate and Understand Routing Algorithms}}, Interacção'06 - Conferência Nacional em Interacção Pessoa-Máquina, Universidade do Minho, Braga, 2006. (\href{run:Publicacoes/publicacoes/13.pdf}{pdf 2.2.4.13})}
\item {Oliveira E., Pereira M.J.V., Henriques P.R., {\bf{ Compreensão de Aplicações Web: o processo e as ferramentas}}, CAPSI 2005 - 6ª Conferência da APSI (Conferência da Associação Portuguesa de Sistemas de Informação), Escola Superior de Tecnologia e Gestão de Bragança, 2005. (\href{run:Publicacoes/publicacoes/12.pdf}{pdf 2.2.4.14})}
\item {Pereira M.J.V., Henriques P.R., {\bf{ Animação de algoritmos tornada sistemática}}, 1ª Workshop Computação Gráfica, Multimédia e Ensino, Leiria, 1999. (\href{run:Publicacoes/publicacoes/3.pdf}{pdf 2.2.4.15})}
\end{enumerate}   

\subsubsection{Publicações em Livros de Resumos} 
\begin{enumerate}  
\item {Quintin K., Pereira M.J.V., Balsa C., {\bf{ Graphical Simulator of mathematical algorithms}}, III Encontro de Jovens Investigadores do IPB, Novembro de 2015. (\href{run:Publicacoes/GraSMA.pdf}{pdf 2.2.5})}
\end{enumerate}

\subsubsection{Publicações Internas de Relatórios técnicos} 
\begin{enumerate}  
\item {Report SPEET}
\item {Pereira M.J.V., Mernik M., Kosar T., Henriques P.R., {\bf{ Object-Oriented Attribute Grammar based Grammatical Approach to Problem Specification}}, Universidade do Minho, 2003.}
\item {Pereira M.J.V., Henriques P.R., {\bf{ Alma: a Generic Program Animation System}}, apresentado no Simpósio Doutoral da Universidade do Minho, 2003 (\href{run:Publicacoes/ArtigoSimDoutoral.pdf}{pdf 2.2.6}).}
\item {Barroca L., Henriques P.R., Pereira M.J.V., {\bf{ Language and Environment for the Pragmatic Application of Formal Methods: work report 2}}, Universidade do Minho, 1995. }
\item {Barroca L., Henriques P.R., Pereira M.J.V., {\bf{ Language and Environment for the Pragmatic Application of Formal Methods: work report 1}}, Universidade do Minho, 1995. }
\end{enumerate} 


\subsubsection{Comunicações Orais em Conferências Internacionais}

Nesta secção estão listados todos os artigos, por ordem cronológica inversa, que mereceram comunicação oral em conferência internacional.  Na maioria deles há uma ligação para o programa da conferência e nos casos em que a comunicação foi feita pela docente foi também colocada uma ligação para os {\em{ slides}} da apresentação.

\begin{enumerate}

\item {Alvaro Neto, Cristiana Araújo, Maria João Varanda, Pedro Henriques, Value-Focused Investigation into Programming Languages Affinity, Third International Computer Programming Education Conference (ICPEC 2022), pp. 1-12, Open Access Series in Informatics (OASIcs), vol. 102, 2022. (fui eu que apresentei)}

\item {Joana Miguel, Maria João Varanda Pereira, Pedro Rangel Henriques, {\bf{ PRIVAS- Automatic Anonymization of Databases}}, 16th International Conference on Applied Computing, Cagliari, Italy, Nov 2019. (IADIS) (fui eu que apresentei)}

\item {Novais D., Pereira M.J.V., Henriques P., {\bf{ Program Analysis for Clustering Programmers' Profile}}, short paper at WAPL 2017,- 6th Workshop on Advances in Programming Languages integrado no FedCSIS (Federated Conference on Computer Science and Information Systems), Praga, Republica Checa, Setembro de 2017. (Fui eu que apresentei)}

\item {Chuchulashvili M., Goziashvili N., Pereira M.J.V, Lopes R., {\bf{ Micro atividades para a Aprendizagem de Programação}}, CMEA 2016 - VII Congresso Mundial de Estilos de Aprendizagem, Instituto Politécnico de Bragança, Julho 2016. (comprovativo programa e slides em publicacoes2016)}

\item {Alves L., Balsa C., Pereira M.J.V., {\bf{ Simulador Gráfico de Algoritmos Matemáticos}}, CMEA 2016 - VII Congresso Mundial de Estilos de Aprendizagem, Instituto Politécnico de Bragança, Julho 2016. (comprovativo programa em publicacoes2016)}

\item {Tavares P., Gomes E.F., Pereira M.J.V., Henriques P.R., {\bf{ Técnicas para aumentar o Envolvimento dos Alunos na Aprendizagem da Programação}}, CMEA 2016 - VII Congresso Mundial de Estilos de Aprendizagem, Instituto Politécnico de Bragança, Julho 2016. (comprovativo programa em publicacoes2016)}

\item {Berón M., Bernardis H., Miranda E., Riesco D., Pereira M.J.V., Henriques P.R., {\bf{ WSDLUD: A Metric to Measure the Understanding Degree of WSDL Description}}, SLATE 2015 - Symposium on Languages, Applications and Technogies, Universidad Complutense de Madrid, Junho 2015. (\href{run:ComunicacoesOrais/programas/SLATE2015prog.pdf}{programa 2.2.7.1})(\href{run:ComunicacoesOrais/Slate2015apres.pdf}{slides 2.2.7.2})}

\item {Azcurra J., Berón M., Montejano G., Farnese A., Henriques P.R., Pereira M.J.V., {\bf{ AId: Uma Ferramenta para Análise de Identificadores de Programas Java}}, CoNaIISI 2014 - 2º Congreso Nacional de Ingeniería Informática/ Sistemas de Información, Argentina, Nov 2014. (\href{run:ComunicacoesOrais/programas/CoN2014prog.pdf}{programa 2.2.7.3})}

\item {Pereira N., Pereira M.J.V., Henriques P.R., {\bf{ Comment-based Concept Location over System Dependency Graphs}}, SLATE 2014 - 3rd Symposium on Languages, Applications and Technologies, Instituto Politécnico de Bragança, Junho 2014. (\href{run:ComunicacoesOrais/programas/SLATE2014prog.pdf}{programa 2.2.7.4})}

\item {Carvalho N.R., Almeida J.J., Pereira M.J.V., Henriques P.R., {\bf{ Conclave: Writing Programs to Understand Programs}},SLATE 2014 - 3rd Symposium on Languages, Applications and Technologies, Instituto Politécnico de Bragança, Junho 2014. (\href{run:ComunicacoesOrais/programas/SLATE2014prog.pdf}{programa 2.2.7.5})}

\item {Fonseca J., Pereira M.J.V., Henriques P.R., {\bf{ Converting Ontologies into DSLs}}, SLATE 2014 - 3rd Symposium on Languages, Applications and Technologies, Instituto Politécnico de Bragança, Junho 2014. (\href{run:ComunicacoesOrais/programas/SLATE2014prog.pdf}{programa 2.2.7.5})}

\item {Pereira M.J.V., Oliveira N., Cruz D., Henriques P.R., {\bf{ Choosing Grammars to Support Language Processing Courses}}, SLATE 2013 - Symposium on Languages, Applications and Technologies, Faculdade de Ciências da Universidade do Porto, Portugal, Junho de 2013. (\href{run:ComunicacoesOrais/programas/SLATE2013prog.pdf}{programa 2.2.7.6}) (\href{run:ComunicacoesOrais/programas/SLATE2013ChooComp.pdf}{comprovativo 2.2.7.7}) (\href{run:ComunicacoesOrais/Slate13apres.pdf}{slides 2.2.7.8})}

\item {Miranda E., Berón M., Montejano G., Pereira M.J.V., Henriques P.R., {\bf{ NESSy: a New Evaluator for Software Development Tools}}, SLATE 2013 - Symposium on Languages, Applications and Technologies, Faculdade de Ciências da Universidade do Porto, Portugal, Junho de 2013. (\href{run:ComunicacoesOrais/programas/SLATE2013prog.pdf}{programa 2.2.7.6}) (\href{run:ComunicacoesOrais/programas/SLATE2013NessyComp.pdf}{comprovativo 2.2.7.9}) (\href{run:ComunicacoesOrais/Slate13Nessyapres.pdf}{slides 2.2.7.10})}

\item {Oliveira N., Pereira M.J.V., Gancarski A.L., Henriques P.R., {\bf{ Learning Spaces for Knowledge Generation}}, SLATE 2012 - Symposium on Languages, Applications and Technologies, Universidade do Minho, Portugal, Junho de 2012. (\href{run:ComunicacoesOrais/programas/SLATE2012prog.pdf}{programa 2.2.7.11})}

\item {Pereira M.J.V., Berón M., Cruz D., Oliveira N., Henriques P.R., {\bf{ Problem Domain Oriented Approach for Program Comprehension}}, SLATE 2012 - Symposium on Languages, Applications and Technologies, Universidade do Minho, Portugal, Junho de 2012. (\href{run:ComunicacoesOrais/programas/SLATE2012prog2.pdf}{programa 2.2.7.12}) (\href{run:ComunicacoesOrais/Slate12apres.pdf}{slides 2.2.7.13})}

\item {Carvalho N.R., Almeida J.J., Pereira M.J.V., Henriques P.R., {\bf{ Probabilistic SynSet Based Concept Location}}, SLATE 2012 - Symposium on Languages, Applications and Technologies, Universidade do Minho, Portugal, Junho de 2012. (\href{run:ComunicacoesOrais/programas/SLATE2012prog2.pdf}{programa 2.2.7.14})}

\item {Balsa C., Alves L., Pereira M.J.V., Rodrigues P., Lopes R., {\bf{ Graphical Simulation of Numerical Algorithms, An approach based on code instrumentation and java technologies}}, CSEDU 2012 - 4th International Conference on Computer Supported Education, Porto, pp. 164-169, Abril 2012. (\href{run:ComunicacoesOrais/programas/CSEDU2012prog.pdf}{programa 2.2.7.15})}

\item {Carvalho N.R., Simões A.M., Almeida J.J., Henriques P.R., Pereira M.J.V., {\bf{ PFTL: A Systematic Approach For Describing Filesystem Tree Processors}},INForum'11 - Simpósio de Informática (CoRTA'11 track), Universidade de Coimbra, Setembro 2011. (\href{run:ComunicacoesOrais/programas/INForum2011prog.pdf}{programa 2.2.7.16})}

\item {Ceh I., Crepinsek M., Kosar T., Mernik M., Henriques P.R., Pereira M.J.V., Cruz D., Oliveira N.,{\bf{ Tool supported building of {DSL}}s from {OWL} ontologies}, INForum'11 -Simpósio de Informática (CoRTA'11 track), Universidade de Coimbra, Setembro 2011. (\href{run:ComunicacoesOrais/programas/INForum2011prog.pdf}{programa 2.2.7.16}) (\href{run:ComunicacoesOrais/Corta11apres.pdf}{slides 2.2.7.17})}

\item {Balsa C., Alves L., Pereira M.J.V., Rodrigues P., {\bf{ Graphical simulator of mathematical algorithms (GraSMA)}}, Proceedings of IASK International Conference Teaching and Learning, Seville, Spain, Dec 2010. (\href{run:ComunicacoesOrais/programas/IASK2010prog.pdf}{programa 2.2.7.18})}

\item  {Lukovic I., Pereira M.J.V., Oliveira N., Cruz D., Henriques P.R., {\bf{ An Attribute Grammar Specification of IIS*Case PIM Concepts}}, MDASD 2010 - Workshop on Model Driven Approaches in System Development integrado no ADBIS 2010 14th East-European Conference on Advances in Databases and Information Systems, Novi Sad, September 2010. (\href{run:ComunicacoesOrais/programas/adbis2010prog.pdf}{programa 2.2.7.19})}

\item {Oliveira N., Pereira M.J., Cruz D., Béron M., {\bf{ Influence of Synchronized Domain Visualizations on Program Comprehension}}, Working Session, ICPC 2010 - 18th IEEE International Conference on Program Comprehension, Braga, 2010. (\href{run:ComunicacoesOrais/programas/ICPC2010.pdf}{programa 2.2.7.20})}

\item {Béron M., Pereira M.J., Oliveira N., Cruz D., {\bf{ SVS, BORS, SVSi: Three Strategies to relate Problem and Program Domains}}, ICPC 2010 - 18th IEEE International Conference on Program Comprehension, Braga, 2010. (\href{run:ComunicacoesOrais/programas/ICPC2010.pdf}{programa 2.2.7.21}) (\href{run:ComunicacoesOrais/ICPC10apres.pdf}{slides 2.2.7.22})}

\item {Oliveira N., Henriques P.,Cruz D., Pereira M.J., Mernik M., Kosar T., Crepinsek M., {\bf{ Applying Program Comprehension Techniques to Karel Robot Programs}}, WAPL'09 - 2nd Workshop on Advances in Programming Languages, Mragowo, Poland, October 2009. (\href{run:ComunicacoesOrais/programas/WAPL2009.pdf}{programa 2.2.7.23})}

\item {Oliveira N., Henriques P., Cruz D., Pereira M.J., {\bf{ VisualLISA: Visual Programming Environment for Attribute Grammars Specification}}, IMCSIT, WAPL'09 - 2nd Workshop on Advances in Programming Languages, Mragowo, Poland, October 2009. (\href{run:ComunicacoesOrais/programas/WAPL2009.pdf}{programa 2.2.7.24})}

\item {Kosar T., Mernik M., Crepinsek M., Henriques P., Cruz D., Pereira M.J., Oliveira N., {\bf{ Influence of domain-specific notation to program understanding}}, IMCSIT, WAPL'09 - 2nd Workshop on Advances in Programming Languages, Mragowo, Poland, October 2009. (\href{run:ComunicacoesOrais/programas/WAPL2009.pdf}{programa 2.2.7.24})}

\item {Béron M., Henriques P., Pereira M.J., Uzal R., {\bf{ Instrumentaciones de Programas Escritos en C para Interrelacionar las Vistas Comportamental y Operacional de los Sistemas de Software}}, CACIC'09 - XV Argentine Congress on Computer Science, Universidad Nacional de Jujuy, Argentina, October 2009 (\href{run:ComunicacoesOrais/programas/CACIC2009.pdf}{programa 2.2.7.25})}

\item {Oliveira N., Pereira M.J., Henriques P., Cruz D., {\bf{ Visualization of Domain Specific Program's Behavior}}, 5th IEEE International Workshop on Visualizing Software for Understanding and Analysis (co-located with ICSM 2009), Edmouton, Canada, September 2009. (\href{run:ComunicacoesOrais/programas/VISSOFT2009.pdf}{programa 2.2.7.26})}

\item {Béron M., Henriques P., Pereira M.J., Uzal R., {\bf{ Simplificando la Comprensión de Programas a través de la Interconnexión de Dominios}}, CACIC'08 - XIV Argentine Congress on Computer Science, Universidad Nacional de Chilecito, La Rioja, Argentina, Outubro 2008.}

\item {Cruz D., Henriques P., Pereira M.J., {\bf{ Strategies for Program Inspection and Visualization}}, CSE'08 - International Scientific Conference on Computer Science and Engineering, September 2008. (\href{run:ComunicacoesOrais/CSE08apres.pdf}{slides 2.2.7.27})}

\item {Fonseca R., Cruz D.,Henriques P., Pereira M.J., {\bf{ How to interconnect operational and behavioral views of web applications}}, ICPC'08 - IEEE International Conference on Program Comprehension, June 2008. (\href{run:ComunicacoesOrais/programas/ICPC2008.pdf}{programa 2.2.7.28}) (\href{run:ComunicacoesOrais/ICPC08apres.pdf}{slides 2.2.7.29})}

\item {Cruz D., Henriques P., Pereira M.J., {\bf{ Pattern-based Program Visualization}}, WAPL'07 - 1st Workshop on Advances in Programming Languages integrado no International Multiconference on Computer Science and Information Technology, Wisla, Polónia, Outubro de 2007. (\href{run:ComunicacoesOrais/programas/WAPL2007.pdf}{programa 2.2.7.31}) (\href{run:ComunicacoesOrais/WAPL2007apres.pdf}{slides 2.2.7.31})}

\item {Béron M., Henriques P., Pereira M.J.,Uzal R., {\bf{ Static and Dynamic Strategies to Understand C Programs by Code Annotation}} , OpenCert 2007 - 1st International Workshop on Fondations and Techniques for Open Source Software Certification (integrado no ETAPS-European Joint Conferences on Theory and Practice of Software), Braga, 2007. (\href{run:ComunicacoesOrais/programas/OpenCert2007.pdf}{programa 2.2.7.32}) (\href{run:ComunicacoesOrais/OPENCERT07apres.pdf}{slides 2.2.7.33})}

\item {Béron M., Henriques P., Pereira M.J., Uzal R., {\bf{ PICS un Sistema de Comprensión e Inspección de Programas}} , CACIC'07 - XIII Argentine Congress on Computer Science, RedUNCI - Red de Universidades con Carreras en Informática, Universidad Nacional de San Luis, Argentina, 2007.}

\item {Berón M., Henriques P., Pereira M.J., Uzal R., {\bf{ Program Inspection to Interconnect Behavioral and Operational View for Program Comprehension}}, York Doctoral Symposium on Computing on Computing, University of York, UK, 2007.}

\item {Rebernak D., Mernik M., Henriques P., Pereira M.J., {\bf{ AspectLISA: an aspect-oriented compiler construction system based on attribute grammars}}, LDTA'06 - 6th Workshop on Languages Descriptions, Tools and Applications, abril 2006. (\href{run:ComunicacoesOrais/programas/LDTA2006prog.pdf}{programa 2.2.7.34})}

\item {Béron M., Henriques P., Pereira M.J., Uzal R., Montejano G., {\bf{ A Language Processing Tool for Program Comprehension}} , CACIC'06 - XII Argentine Congress on Computer Science, Universidad Nacional de San Luis, Argentina, 2006. (\href{run:ComunicacoesOrais/programas/Cacic2006prog.pdf}{programa 2.2.7.35})}

\item {Béron M., Henriques P., Pereira M.J., Uzal R., {\bf{ Comprensión de Algoritmos de Ruteo}}, CLEI 2006 - XXXII Conferencia Latinoamericana de Informática, Santiago de Chile, 2006. (\href{run:ComunicacoesOrais/programas/CLEI2006.pdf}{programa 2.2.7.36})}

\item {Rebernak D., Mernik M., Henriques P., Pereira M.J., Cruz D., {\bf{ Specifying Languages using Aspect-oriented Approach: AspectLISA}}, ITI'06 - 28th International Conference Information Technology Interfaces, Cavtat, Croacia, 2006. (\href{run:ComunicacoesOrais/programas/ITI2006presentation.pdf}{programa 2.2.7.37})}

\item {Lopes R., Raimundo N., Pereira M.J., Oliveira J., Roque V., {\bf{ Executable Graphics for PBNM, The 5th IEEE International Workshop on IP Operations \& Management O\&M Challenges in Next Generation Services and Networks}}, Barcelona, Spain, 2005. (\href{run:ComunicacoesOrais/programas/IPOM05.pdf}{programa 2.2.7.38})}

\item {Henriques P., Kosar T., Mernik M., Pereira M.J., Zumer V., {\bf{ Grammatical Approach to Problem Solving}}, ITI'03 - 25th International Conference on Information Technology Interfaces, Cavtat, Croatia, 2003.}

\item {Pereira M.J., Henriques P., {\bf{ Visualization / Animation of Programs based on Abstract Representations and Formal Mappings}}, HCC'01 - 2001 Symposia on Human-Centric Computing Languages and Environments, pp. 373-381, Stresa, Italy, 2001. (\href{run:ComunicacoesOrais/programas/HCC2001Program.pdf}{programa 2.2.7.39}) (\href{run:ComunicacoesOrais/HCC01apres.pdf}{slides 2.2.7.40})}

\item {Pereira M.J., Henriques P., {\bf{ Visualização Sistemática de Programas}}, IV Simpósio Brasileiro de Linguagens de Programação, Recife, Brasil, 2000. (\href{run:ComunicacoesOrais/SBLP00apres.pdf}{slides 2.2.7.41})}

\item {Pereira M.J., Rocha J., Henriques P., {\bf{ Concepção, Especificação e Implementação de Processadores de Linguagens Visuais}}, SBLP 1997 - II Simpósio Brasileiro de Linguagens de Programação, Campinas, Brasil, 1997. (\href{run:ComunicacoesOrais/programas/SBLP1997.pdf}{programa 2.2.7.42})}

\end{enumerate}


\subsubsection{Comunicações Orais em Conferências Nacionais}

Nesta secção estão listados todos os artigos, por ordem cronológica inversa, que mereceram comunicação oral em conferência nacional.  Na maioria deles há uma ligação para o programa da conferência e nos casos em que a comunicação foi feita pela docente foi também colocada uma ligação para os {\em{ slides}} da apresentação.

\begin{enumerate}

\item {Oliveira N., Pereira M.J., Henriques P., Cruz D., {\bf{ Domain Specific Languages: A Theoretical Survey}}, CoRTA'09 (INForum'09 - Simpósio de Informática), Faculdade de Ciências da Universidade de Lisboa, Setembro 2009. (\href{run:ComunicacoesOrais/programas/INForum2009.pdf}{programa 2.2.8.1})}

\item {Oliveira N. and Pereira M.J., Henriques P., Cruz D., Cramer B., {\bf{ VisualLISA: A Domain Specific Visual Language for Attribute Grammars}}, CoRTA'09 (INForum'09 - Simpósio de Informática), Faculdade de Ciências da Universidade de Lisboa, Setembro 2009. (\href{run:ComunicacoesOrais/programas/INForum2009.pdf}{programa 2.2.8.2})}

\item {Mernik M., Kosar T., Crepinsek M., Henriques P., Cruz D., Pereira M.J., Oliveira N., {\bf{ Comparison of XAML and C\# Forms using Cognitive Dimensions Framework}}, CoRTA'09 (INForum'09 - Simpósio de Informática), Faculdade de Ciências da Universidade de Lisboa, Setembro 2009 (\href{run:ComunicacoesOrais/programas/INForum2009.pdf}{programa 2.2.8.3}).}

\item {Oliveira N., Henriques P., Cruz D., Pereira M.J., {\bf{ XAGra - An XML Dialect for Attribute Grammars}}, XATA'09 (INForum'09 - Simpósio de Informática), Faculdade de Ciências da Universidade de Lisboa, Setembro 2009. (\href{run:ComunicacoesOrais/programas/INForum2009.pdf}{programa 2.2.8.4}) }

\item {Pereira M.J., Mernik M., Cruz D., Henriques P., {\bf{ Program Comprehension for Domain-Specific Languages}}, CoRTA'08 - Compilers, Related Technologies and Applications, Julho 2008. (\href{run:ComunicacoesOrais/programas/CoRTA08prog.pdf}{programa 2.2.8.5}) (\href{run:ComunicacoesOrais/Corta08DSLapres.pdf}{slides 2.2.8.6})}

\item {Pereira M.J., Mernik M., Cruz D., Henriques P., {\bf{ VisualLISA: a Visual Interface for an Attribute Grammar based Compiler-Compiler}}, CoRTA'08 - Compilers, Related Technologies and Applications, Julho 2008. (\href{run:ComunicacoesOrais/programas/CoRTA08prog.pdf}{programa 2.2.8.7}) (\href{run:ComunicacoesOrais/Corta08VLapres.pdf}{slides 2.2.8.8})}

\item {Cruz D., Henriques P., Pereira M.J., {\bf{ Alma versus DDD}}, CoRTA'08 - Compilers, Related Technologies and Applications, Julho 2008. (\href{run:ComunicacoesOrais/programas/CoRTA08prog.pdf}{programa 2.2.8.9})}

\item {Cruz D., Henriques P., Pereira M.J., {\bf{ Exploring and Visualizing the "Alma" of XML Documents}}, XATA 2008 - XML: Aplicações e Tecnologias Associadas, Universidade de Évora, Fevereiro 2008. (\href{run:ComunicacoesOrais/programas/XATA2008.pdf}{programa 2.2.8.10})}

\item {Berón M., Henriques P., Pereira M.J., Uzal R., {\bf{ Técnicas de Inspecção de Programas para Inter-Relacionar as Vistas Comportamental e Operacional}}, Encontro Português de Inteligência Artificial, Guimarães, 2007.}

\item {Cruz D., Pereira M.J., Berón M., Fonseca R. and Henriques P., {\bf{ Comparing Generators for Language-based Tools}}, CORTA'07 - Compiler, Related Technologies and Applications, Universidade da Beira Interior, Julho 2007. (\href{run:ComunicacoesOrais/programas/CoRTA2007prog.pdf}{programa 2.2.8.11})}

\item {Oliveira E., Henriques P., Pereira M.J., {\bf{ Proposta de um Sistema para Compreensão de Aplicações Web}}, CAPSI 2006- 7ªConferência da Associação Portuguesa de Sistemas de Informação, Aveiro, Portugal, 2007. (\href{run:ComunicacoesOrais/programas/CAPSI2006.pdf}{programa 2.2.8.12})}

\item {Oliveira E., Pereira M.J., Henriques P., {\bf{ Compreensão de Aplicações Web: o processo e as ferramentas}}, CAPSI 2005 - 6ª Conferência da APSI (Conferência da Associação Portuguesa de Sistemas de Informação), Escola Superior de Tecnologia e Gestão de Bragança, 2005. (\href{run:ComunicacoesOrais/programas/CAPSI2005.pdf}{programa 2.2.8.13})}

\item {Pereira M.J., Henriques P., {\bf{ Alma: a Generic Program Animation System}}, apresentado no Simpósio Doutoral da Universidade do Minho, 2003.  (\href{run:ComunicacoesOrais/SimDoutoralapres.pdf}{slides 2.2.8.14})}

\item {Pereira M.J., Henriques P., {\bf{ Animação de algoritmos tornada sistemática}}, 1ª Workshop Computação Gráfica, Multimédia e Ensino, Leiria, 1999. (\href{run:ComunicacoesOrais/programas/CGME1999.pdf}{programa 2.2.8.15}) (\href{run:ComunicacoesOrais/cgme99apres.pdf}{slides 2.2.8.16})}

\end{enumerate}


\subsubsection{Apresentações em poster}
\begin{enumerate}
\item {Kevin Quintin, Maria João Pereira e Carlos Balsa, III Encontro de Jovens Investigadores do IPB, {\bf{ Graphical Simulator of mathematical algorithms}}, Novembro de 2015. (\href{run:Publicacoes/CertificadoEJI.pdf}{Comprovativo 2.2.9})}

\item {Hernán Bernardis, Edgardo Bernardis, Mario Berón, Daniel E. Riesco, Pedro Rangel Henriques, Maria João V. Pereira,{\bf{ Cálculo de Métricas para Medir el Grado de Entendimiento de una Descripción WSDL}}, WICC 2016 - XVIII Workshop de Investigadores en Ciencias de la Computación,Concordia, Entre Ríos, Argentina, abril 2016. }

\item {Norma Beatriz Perez, Mario Berón, Pedro Rangel Henriques, Maria João Varanda Pereira,{\bf{ Comprensión de Sistemas Paralelos}}, WICC 2015 - XVII Workshop de Investigadores en Ciencias de la Computación,Salta, Argentina, abril 2015. }

\item {Hernán Bernardis, Edgardo Bernardis, Mario Berón, Daniel E. Riesco, Pedro Rangel Henriques, Maria João Varanda Pereira, {\bf{ Técnicas y Estrategias para Compreender Procesos de Negocios Especificados en WS-BPEL}}, WICC 2015 - XVII Workshop de Investigadores en Ciencias de la Computación,Salta, Argentina, abril 2015.}

\item {Enrique A. Miranda, Mario Berón, German Montejano, Mario Peralta, Maria J. Pereira, {\bf{ Visualización de Software Orientada a Comprensión de Programas}}, WICC 2012 - XIV Workshop de Investigadores en Ciencias de la Computación, Misiones, Argentina, Abril de 2012. }

\item {Javier Azcurra, Mario Berón, Pedro Rangel Henriques, Maria João V. Pereira, {\bf{ Análisis de Información Informal para Facilitar la Comprensión de Programas}}, WICC 2012 - XIV Workshop de Investigadores en Ciencias de la Computación, Misiones, Argentina, Abril de 2012. }

\item {Maria Matkovic, Mario Berón, Carlos Salgado, Pedro Rangel Henriques, Maria João V. Pereira, {\bf{ Construcción de Representaciones Innovadoras del Dominio del Programa para Facilitar la Comprensión de Programas}}, WICC 2012 - XIV Workshop de Investigadores en Ciencias de la Computación, Misiones, Argentina, Abril de 2012. }

\item {Martín Aristiaran,  Mario Berón , Pedro Henriques, Maria João Varanda Pereira, {\bf{ Clasificaciones: un mecanismo de herencia múltiple para la construcción de modelos fáciles de comprender y mantener}} , WICC 2011 - XIII Workshop de Investigadores en Ciencias de la Computación, Rosario, Argentina, Maio 2011. }

\item {José Albanes,  Mario Berón , Pedro Henriques, Maria João Varanda Pereira, {\bf{ Estrategias para relacionar el dominio del problema con el dominio del programa para la comprensión de programas}} , WICC 2011 - XIII Workshop de Investigadores en Ciencias de la Computación, Rosario, Argentina, Maio 2011. }

\item {Enrique Miranda, Mario Berón , Germán Montejano, Daniel Riesco, Pedro Henriques, Maria João Varanda Pereira, {\bf{ Visualización de software: conceptos, métodos y técnicas para facilitar la comprensión de programas}} , WICC 2011 - XIII Workshop de Investigadores en Ciencias de la Computación, Rosario, Argentina, Maio 2011.}

\item {Hernán Bernardis,  Mario Berón , Daniel Riesco, Pedro Henriques, Maria João Varanda Pereira, {\bf{ Extracción de información dinámica en programación orientada a objetos (Java)}} , XIII Workshop de Investigadores en Ciencias de la Computación, Rosario, Argentina, Maio 2011. }

\item {Ignacio El Kadre,  Mario Berón , Carlos Salgado, Mario Peralta, Pedro Henriques, Maria João Varanda Pereira, {\bf{ Construcción de representaciones del dominio del programa para facilitar la interconexión de dominios}} , WICC 2011 - XIII Workshop de Investigadores en Ciencias de la Computación, Rosario, Argentina, Maio 2011.}

\item {Mariano Luzza,  Mario Berón , Germán Montejano, Pedro Henriques, Maria João Varanda Pereira, {\bf{ Diseño y Construcción de Lenguajes Específicos del Dominio}} , WICC 2011 - XIII Workshop de Investigadores en Ciencias de la Computación, Rosario, Argentina, Maio 2011. }

\item {Mario Berón, Roberto Uzal, Pedro Henriques, Maria João Varanda Pereira, {\bf{ Estrategias para Facilitar la Comprensión de Programas}}, WICC 2010 - XII Workshop de Investigadores en Ciencias de la Computación, Universidad Nacional de la Patagonia San Juan Bosco, Argentina. }

\item {Mario Béron, Daniela da Cruz, Maria João Varanda Pereira, Pedro Henriques, Roberto Uzal, {\bf{ Evaluation Criteria of Software Visualization Systems used for Program Comprehension}}, Interacção'08 -- 3ª Conferência Interacção Pessoa\-Máquina, Universidade de Évora, Outubro 2008.}

\item {Mario Berón, Pedro Henriques, Maria João Varanda Pereira, Roberto Uzal, {\bf{ Inspección de Código para relacionar los Dominios del Problema y Programa para la Comprensión de Programas}}, WICC 2008 - X Workshop de Investigadores en Ciencias de la Computación, La Pampa, Argentina, 2008. }

\item {Mario Béron, Pedro Henriques, Maria João Varanda Pereira, Roberto Uzal, {\bf{ Herramientas para la compresión de programas}}, WICC 2006 - VIII Workshop de Investigadores en Ciencias de la Computación, Universidad de Morón, Bs. As., Argentina, 2006. }

\item {Mario Béron, Pedro Henriques, Maria João Varanda Pereira, {\bf{ A System to Evaluate and Understand Routing Algorithms}}, Interacção'06 - Conferência Nacional em Interacção Pessoa-Máquina, Universidade do Minho, Braga, 2006. }


\end{enumerate}


\subsubsection{Edição de Publicações científicas}
\begin{enumerate}
\item {João Cordeiro, Maria João Pereira, Nuno F. Rodrigues, Sebastião Pais, "`11th Symposium on Languages, applications and Technologies (SLATE 2022) Proceedings, Open Access Series in Informatics (OASIcs), Volume 104, 2022. (ISBN 978-3-95977-245-7)}
\item {Queir\'{o}s, Ricardo and Pinto, M\'{a}rio and Sim\~{o}es, Alberto and Portela, Filipe and Pereira, Maria Jo\~{a}o, 10th Symposium on Languages, Applications and Technologies (SLATE 2021) Proceedings, Open Access Series in Informatics (OASIcs), Volume 94, 2021.}
\item {Maria João Varanda Pereira, José Paulo Leal, Alberto Simões, {\bf{ 3rd Symposium on Languages, Applications and Technologies (SLATE 2014)}}, OASICS, vol 38. (\href{run:Publicacoes/ComprovativosDBLP.pdf}{DBLP}) (\href{run:Publicacoes/PROCslate2014.pdf}{pdf 2.2.10.1}).}
\item {Lukovic I., Mernik M., Slivnik B., Janousek J., Aycock J., Chen H., Henriques P.R., Horvath Z., Ivanovic M., Kardas G., Kollar J., Kosar T., Liu S.-H., Lukovic I., Mandreoli F., Martínez López P.E., Mernik M., Milasinovic B., Moessenboeck H., Papaspyrou N., Pereira M.J.V., Poruban J., Rodriguez J.L.S., Slivnik B., Splawski Z., Watson B., {\bf{ 4th Workshop on Advances in Programming Languages}}, 2013 Federated Conference on Computer Science and Information Systems (\href{run:Publicacoes/PublicacoesSCOPUS.pdf}{SCOPUS}) (\href{run:OutrasActCientif/EDWAPL2013.pdf}{Comprovativo 2.2.10.2}).}
\item {Ivan Lukovic, Mirjana Ivanovic, Maria João Varanda Pereira, {\bf{ Advances in Formal Languages, Modeling and Applications}}, ComSIS, Volume 8, Issue 2, 2011 (\href{run:Publicacoes/ComprovativosISI.pdf}{ISI},\href{run:Publicacoes/PublicacoesSCOPUS.pdf}{SCOPUS}) (\href{run:Publicacoes/ComSIS0802.pdf}{pdf 2.2.10.3}).}
\item {Maria João Varanda Pereira, Pedro Rangel Henriques, Simão Melo de Sousa, {\bf{ Compilers, Related Technologies and Applications (CoRTA'08)}}, 2008, IPB, ISBN: 978-972-745-096-1 (\href{run:Publicacoes/ActasCORTA08VFR.pdf}{pdf 2.2.10.4}).}
\end{enumerate}

\subsubsection{Citações}
Os seguintes dados bibliométricos foram retirados das fontes a 23 de maio de 2016 (\href{run:Publicacoes/citacoesISI.pdf}{Dados extraídos do ISI - 2.2.11}) :
\begin{itemize}
\item {ISI:}
\begin{itemize}
\item {Número de publicações ISI: 17}
\item {Número de publicações ISI não SCOPUS: 0}
\item {Número de citações: 83}
\item {Número de citação sem as próprias: 78}
\item {Número de citação ISI não SCOPUS: 5}
\end{itemize}
\item {SCOPUS:}
\begin{itemize}
\item {Número de publicações SCOPUS: 34}
\item {Número de citações: 173}
\item {Número de citação sem as próprias: 160}
\end{itemize}
\end{itemize}


\subsubsection{Participação como orador convidado em seminários}
\begin{itemize}
\item {Oradora convidada no "1st International Conference On Teaching Programming Languages", Escola Superior de Comunicação, Administração e Turismo do Instituto Politécnico de Bragança, Portugal (EsACT-IPB), 12 de abril de 2023.}
\item {Palestrante do Webinar CeDRI , 21 de abril com o título: "Programação no ensino vs ensino da programação".}
\item {Seminário de 2 horas aos alunos de ensino de Informática da UM sobre "Informática no ensino e Ensino da Informática" a convite do Prof. Pedro Henriques, novembro 2021.}
\item {Seminário de 2 horas ao Módulo M2- Projeto da UCE 15 do 2º ano do Mestrado em Informática da Universidade do Minho sobre Gestão de Projetos de Desenvolvimento de Software, Outubro de 2007 (\href{run:MissaoIPBoutros/comunicacaooralUM.pdf}{Comprovativo 2.2.12}). Esta experiência repetiu-se em outubro de 2008.}
\end{itemize}

\subsubsection{Organização de Congressos, Conferências e Seminários}
\noindent{Eventos Internacionais}
\begin{enumerate}

\item {Organização da reunião técnica do projeto Bacchustech, 27 e 28 de julho na ESTiG.}
\item {Organização do Seminário Transfronteiriço em Cibersegurança e Inteligência Artificial, dia 6 de Maio 2022, ESTiG}
\item {Chair de uma sessão do SLATE 2022 e HCL track chair do SLATE 2022 na Covilhã em Julho.}
\item {HCL Track-chair no SLATE 2021 online }

\item {General-chair do WAPL 2019- 7th Workshop on Advances in Programming Languages (\url{https://fedcsis.org/2019/wapl}), Organized within the framework of FedCSIS, the Federated Conference on Computer Science & Information Systems, Leipzig, Germany,  1-4 September, 2019 (\url{http://www.fedcsis.org/}).}
\item {Membro da comissão organizadora do SLATE 2018 - Symposium on Languages, Applications and Technologies que decorreu no Campus de Couros da Universidade do Minho nos dias 21 e 22 de junho de 2018.}
\item {Organização do {\em{ {multiplier event}}} do projecto SPEET no IPB.}
\item {Co-chair do WAPL 2017- 6th Workshop on Advances in Programming Languages, Praga, Republica Checa, Setembro de 2017.}
\item {Responsável pela organização do track de Human-Computer Interaction do SLATE 2017 - 6th Symposium on Languages, Applications and Technologies, ESMAD-IPP. Vila do Conde, Junho 2017.}
\item {Responsável pela organização do SLATE 2014 - 3rd Symposium on Languages, Applications and Technologies, Escola Superior de Tecnologia e Gestão do Instituto Politécnico de Bragança, Portugal, Junho de 2014 (\href{run:ComprovativosCOCP/COSLATE2014.pdf}{Comprovativo 2.2.13.1}). }
\item {Membro da comissão organizadora do SLATE 2015 - 4th Symposium on Languages, Applications and Technologies, Universidad Complutense de Madrid, Madrid, Espanha, Junho de 2015 (\href{run:ComprovativosCOCP/COSLATE2015.pdf}{Comprovativo 2.2.13.2}). }
\item {Membro da comissão organizadora do RAISE 2012- Workshop on Realizing AI Synergies in Software Engineering integrado no ICSE 2012 - 34th International Conference on Software Engineering, Zurich, Junho 2012 (\href{run:ComprovativosCOCP/COCPRAISE2012.pdf}{Comprovativo 2.2.13.3}).}
\item {Membro da comissão organizadora do ICPC 2010 - International Conference on Program Comprehension, Junho 2010, Braga (\href{run:ComprovativosCOCP/COICPC10.pdf}{Comprovativo 2.2.13.4}).}
\item {Membro da Comissão Organizadora da Escola de Verão em Programa\-ção Funcional Avançada que teve lugar na Universidade do Minho, em Setembro de 1998.}
\end{enumerate}
\noindent{Eventos Nacionais}
\begin{enumerate}
\item {Organização do Talk for Us (CeDRI) – Prof. Alexandre Cardoso, Junho 2023.}
\item {Organização do Seminário da Altice Labs sobre “Internet of Things”, 15 de maio de 2019 (Palestrante Fernado Morgado).}
\item {Responsável pela organização do CORTA'08 - Compilers, Related Technologies and Applications, Escola Superior de Tecnologia e Gestão do Instituto Politécnico de Bragança, Julho 2008 (\href{run:ComprovativosCOCP/COCPCoRTA08.pdf}{Comprovativo 2.2.13.5}).}
\item {Responsável pela organização da 8ª Edição do Concurso Nacional de Programação Lógica e Funcional de 2005, que decorreu nos dias 13,14 e 15 de Maio de 2005, na ESTiG (\href{run:ComprovativosCOCP/COCNPLF05.pdf}{Comprovativo 2.2.13.6}).}
\item {Membro da comissão organizadora do CORTA'07 - Compilers, Related Technologies and Applications, Departamento de Informática da Universidade da Beira Interior, Covilhã, Portugal, Julho 2007 (\href{run:ComprovativosCOCP/COCoRTA2007.pdf}{Comprovativo 2.2.13.7}).}
\item {Membro da comissão organizadora do Dia da Engenharia Informática (entre 2002 e 2006) e Dia da Engenharia Electrotécnica (entre 1998 e 2002), integrados no evento anual da Semana das Engenharias da ESTiG Bragança.}
\item {Membro do secretariado do 6º Encontro Português de Computação Gráfica, realizado na Universidade do Minho, Braga, em Fevereiro de 1994. }
\item {Membro da comissão organizadora do simpósio "Sistemas de Informação e a Empresa", que teve lugar na Universidade do Minho, em 1993. }
\end{enumerate}


\subsubsection{Comissões de programa}
\begin{enumerate}
\item {Membro da comissão de programa do SLATE 2023, ICPEC 2023, FedCSIS 2023 (WAPL steering commitee).}
\item {Membro da Comissão de Programa do ICPEC 2021.}
\item {Membro da Comissão de programa do MDASD 2022.}
\item {Membro da comissão de programa do SLATE 2022.}
\item {Membro da comissão de programa do SLATE 2021.}
\item {Membro da comissão de programa do Simpósio de Engenharia Informática (SEI), ISEP (\url{http://sei.dei.isep.ipp.pt/comissoes.html}), 2020, 2021 e 2022.}
\item {Membro da comissão de programa do SLATE 2019, 2020.}
\item {Membro da Comissão de Programa do WorldCIST 2019, 2020 e 2021.}
\item {Membro da Comissão de Programa do Computing 2018 da Georgia.}
\item {Membro da comissão de programa do CAPSI 2018.}
\item {Membro da Comissão de programa do MDASD 2018.}
\item {Membro da comissão de programa do SLATE 2018.}

\item {Membro da comissão de programa do WorldCIST 2018-6th World Conference on Information Systems and Technologies, Napoles, Itália, 27 - 29 Março 2018.}

\item {Membro da comissão de programa da CAPSI 2017 - 17ª Conferência da Associação Portuguesa de Sistemas de Informação, Universidade do Minho, Junho 2017. }

\item {Membro da comissão de programa do WAPL 2017- 6th Workshop on Advances in Programming Languages integrado no FedCSIS (Federated Conference on Computer Science and Information Systems), Praga, Republica Checa, Setembro de 2017.}

\item {Membro da comissão de programa do SLATE 2017 - 6th Symposium on Languages, Applications and Technologies, ESMAD-IPP. Vila do Conde, Junho 2017.}

\item {Membro da comissão de programa do MDASD 2016 - Workshop on Model Driven Approaches in System Development integrado no FedCSIS - Federated Conference on Computer Science and Information Systems, Gdansk, Polónia, Setembro 2016 (\href{run:ComprovativosCOCP/CPMDASD2016.pdf}{Comprovativo 2.2.14.1}).}
\item {Membro da Comissão Científica do VII Congresso Mundial de Estilos de Aprendizagem:Educação, Tecnologias e Inovação, julho de 2016, IPB (\href{run:CongressoDocencia/ComissaoCientificaCMEA2016.pdf}{Comprovativo 2.2.14.2}).}
\item {Membro da comissão de programa do SLATe 2016 - Symposium on Languages, Applications and Technologies, Universidade de Maribor, Eslovénia, Junho de 2016 (\href{run:ComprovativosCOCP/CPSLATE2016.pdf}{Comprovativo 2.2.14.3}). }
\item {Membro da comissão de programa do Workshop Semantics for Humanities Resources integrado no WorldCIST 2016 - 4th World Conference on Information Systems and Technologies, Recife, Brasil, Março 2016 (\href{run:ComprovativosCOCP/WorldCist'16.pdf}{Comprovativo 2.2.14.4}). }
\item {Membro da comissão de programa do WAPL 2015 - 5th Workshop on Advances in Programming Languages integrado no FedCSIS (Federated Conference on Computer Science and Information Systems), Polónia, Setembro 2015 (\href{run:ComprovativosCOCP/CPWAPL2015.pdf}{Comprovativo 2.2.14.5}).}
\item {Membro da comissão de programa do SLATe 2015 - Symposium on Languages, Applications and Technologies, Universidad Complutense de Madrid, Madrid, Espanha, Junho de 2015 (\href{run:ComprovativosCOCP/CPSLATE2015.pdf}{Comprovativo 2.2.14.6}). }
\item {Membro da comissão de programa do 2º Congreso Nacional de Ingeniería Informática / Sistemas de Información (CoNaIISI 2014), San Luis, Argentina, Novembro 2014 (\href{run:ComprovativosCOCP/CPCoNaIISI2014.pdf}{Comprovativo 2.2.14.7}).}
\item {Membro da comissão de programa do Kes-IDT-14 - 6th International Conference on Intelligent Decision Technologies, Grécia, Junho de 2014 (\href{run:ComprovativosCOCP/CPKesIDT14.pdf}{Comprovativo 2.2.14.8}).}
\item {Membro da comissão de programa do SLATe 2014 - Symposium on Languages, Applications and Technologies, Escola Superior de Tecnologia e Gestão do Instituto Politécnico de Bragança, Portugal, Junho de 2014 (\href{run:ComprovativosCOCP/CPSLATE2014.pdf}{Comprovativo 2.2.14.9}).}
\item {Membro da comissão de programa do QUATIC 2014 - 9th International Conference on the Quality of Information and Communications Technology, Guimarães, Setembro 2014 (\href{run:ComprovativosCOCP/CPQuatic2014.pdf}{Comprovativo 2.2.14.10}).}
\item {Membro da comissão de programa do MDASD 2014 - Workshop on Model Driven Approaches in System Development integrado no FedCSIS - Federated Conference on Computer Science and Information Systems, Wroclaw, Polónia, Setembro 2014 (\href{run:ComprovativosCOCP/CPMDASD2014.pdf}{Comprovativo 2.2.14.11}).}
\item {Membro da comissão de programa do SLATE 2013 - Symposium on Languages, Applications and Technologies, Faculdade de Ciências da Universidade do Porto, Portugal, Junho de 2013 (\href{run:ComprovativosCOCP/CPSLATE2013.pdf}{Comprovativo 2.2.14.12}).}
\item {Membro da comissão de programa do WAPL 2013 - 4th Workshop on Advances in Programming Languages integrado no FedCSIS (Federated Conference on Computer Science and Information Systems), Polónia, Setembro 2013 (\href{run:ComprovativosCOCP/CPWAPL2013.pdf}{Comprovativo 2.2.14.13}).}
\item {Membro da comissão de programa da CSE 2012 - International Scientific Conference on Computer Science and Engineering, Tecnical University of Kosice, Eslováquia, Outubro de 2012 (\href{run:ComprovativosCOCP/CPCSE2012.pdf}{Comprovativo 2.2.14.14}).}
\item {Membro da comissão de programa do SLATE 2012 - Symposium on Languages, Applications and Technologies, Universidade do Minho, Portugal, Junho de 2012 (\href{run:ComprovativosCOCP/CPSLATE2012.pdf}{Comprovativo 2.2.14.15}).}
\item {Membro da comissão de programa  do MDASD 2012 - Workshop on Model Driven Approaches in System Development integrado no FedCSIS - Federated Conference on Computer Science and Information Systems, Wroclaw, Polónia, Setembro 2012 (\href{run:ComprovativosCOCP/CPMDASD2012.pdf}{Comprovativo 2.2.14.16}).}
\item {Membro da comissão de programa do RAISE 2012- Workshop on Realizing AI Synergies in Software Engineering integrado no ICSE 2012 - 34th International Conference on Software Engineering, Zurich, Junho 2012 (\href{run:ComprovativosCOCP/COCPRAISE2012.pdf}{Comprovativo 2.2.14.17}).}
\item {Membro do \emph{steering committee} e da comissão de programa do CoRTA 2011 - Compilers, Related Technologies and Applications (integrado no Inforum 2011), Universidade de Coimbra, Coimbra (\href{run:ComprovativosCOCP/CPCoRTA2011.pdf}{Comprovativo 2.2.14.18}).}
\item  {Membro da comissão de programa do WAPL 2011 - 3rd Workshop on Advances in Programming Languages integrado no FedCSIS (Federated Conference on Computer Science and Information Systems), Polónia, Setembro 2011 (\href{run:ComprovativosCOCP/CPWAPL2011.pdf}{Comprovativo 2.2.14.19}).}
\item {Membro da comissão de programa do MDASD (Model Driven Approaches in System Development) integrado no ADBIS 2010 - 14th East-European Conference on Advances in Databases and Information Systems, Novi Sad, September 20 - 24, 2010 (\href{run:ComprovativosCOCP/CPMDASD2010.pdf}{Comprovativo 2.2.14.20}).}
\item {Membro da comissão de programa do CoRTA 2010 - Compilers, Related Technologies and Applications (integrado no Inforum 2010), Braga, Universidade do Minho (\href{run:ComprovativosCOCP/CPCoRTA2010.pdf}{Comprovativo 2.2.14.21}).}
\item {Membro da comissão de programa da CSE 2010 - International Scientific Conference on Computer Science and Engineering, Tecnical University of Kosice, Eslováquia, Setembro de 2010 (\href{run:ComprovativosCOCP/CPCSE2010.pdf}{Comprovativo 2.2.14.22}).}
\item {Membro da comissão de programa do CoRTA'09 - Compilers, Related Technologies and Applications (integrado no Inforum 2009), Lisboa, Setembro 2009 (\href{run:ComprovativosCOCP/CPCoRTA2009.pdf}{Comprovativo 2.2.14.23}).}
\item {Membro da comissão de programa do WAPL'09 - 2nd Workshop on Advances in Programming Languages (WAPL'09) integrado no "International Multiconference on Computer Science and Information Technology" (\href{run:ComprovativosCOCP/CPWAPL09.pdf}{Comprovativo 2.2.14.24}).}
\item {Membro da comissão de programa da CSE'08 - International Scientific Conference on Computer Science and Engineering, Tecnical University of Kosice, High Tatras, Eslováquia, Setembro de 2008 (\href{run:ComprovativosCOCP/CPCSE2008.pdf}{Comprovativo 2.2.14.25}).}
\item {Membro da comissão de programa do CoRTA'08 - Compilers, Related Technologies and Applications, Instituto Politécnico de Bragança, ESTiG, Julho de 2008 (\href{run:ComprovativosCOCP/COCPCoRTA08.pdf}{Comprovativo 2.2.14.26}).}
\item {Membro da comissão de programa do WAPL'07 - 1st Workshop on Advances in Programming Languages (WAPL'07) integrado no "International Multiconference on Computer Science and Information Technology", Wisla, Poland, Outubro 2007 (\href{run:ComprovativosCOCP/CPWAPL07.pdf}{Comprovativo 2.2.14.27}).}
\item{Membro da comissão de programa da 6ª Conferência da Associação Portuguesa de Sistemas de Informação, ESTiG, Bragança, 2005 (\href{run:ComprovativosCOCP/CPAPSI05.pdf}{Comprovativo 2.2.14.28}).}
\end{enumerate}

\subsubsection{Avaliador de artigos científicos submetidas a revistas e conferências}
\noindent{Revisão de artigos em Revistas Indexadas}\\
\begin{itemize}
\item {Revisão de 1 artigo IPSI Transactions on Internet Research, Abril de 2023.}
\item {Revisão de 1 artigo MDPI journal, Agosto 2022.}
\item {Revisão de 1 artigo para a revista COLA (Journal of Computer Languages), 2021.}
\item {Revisão de 1 artigo para a revista Jordanian Journal of Computers and Information Technologies, 2020.}
\item {Revisão de 1 artigo para a revista COLA (Journal of Computer Languages),2020.}
\item {Revisão de 1 artigo para IEEE 2020.}

\item {Revisão de 1 artigo para a revista ACM Computing Surveys ( Manuscript CSUR-2019-125).}
\item {Revisão de 2 artigos para a revista Computer Languages, Systems \& Structures, 2018.}
\item {Revisão de um artigo para a revista Information (open access journal by MDPI), 2017.}
\item {Revisão de um artigo para a revista NLE (Natural Language Engineering), 2017.}
\item {Revisão de 1 artigo para a revista MDPI (), 2017.}
\item {Revisão de 1 artigo para a revista Computer Languages, Systems \& Structures, 2017.}
\item {Revisão de 1 artigo para a revista Computer Languages, Systems \& Structures, 2016 (\href{run:ComprovativosCOCP/revisoes/ComLang2016.pdf}{Comprovativo 2.2.15.1.1}). }
\item {Revisão de 1 artigo para a revista Software Quality Journal (artigo vindo do Quatic 2014) (\href{run:ComprovativosCOCP/revisoes/SQJ2014.pdf}{Comprovativo 2.2.15.1.2}).}
\item {Revisão de 1 artigo para a revista JIS - Journal of Information Science, Julho 2013 (\href{run:ComprovativosCOCP/revisoes/JIS2013.pdf}{Comprovativo 2.2.15.1.3}).}
\item {Revisão de 1 artigo para a revista ComSIS - Computer Science and Information Systems Journal, Faculty of Technical Sciences, Novi Sad, Serbia, Fevereiro 2013 (\href{run:ComprovativosCOCP/revisoes/ComSIS2013.pdf}{Comprovativo 2.2.15.1.4}).}
\item {Revisão de 2 artigos para a revista ComSIS - Computer Science and Information Systems Journal, Faculty of Technical Sciences, Novi Sad, Serbia, 2011 (\href{run:ComprovativosCOCP/revisoes/ComSIS2011.pdf}{Comprovativo 2.2.15.1.5}).}
\item {Revisão de 1 artigo para o Jornal ComSIS - Computer Science and Information Systems Journal, No. 135-0812, Faculty of Technical Sciences, Novi Sad, Serbia, 2010 (\href{run:ComprovativosCOCP/revisoes/ComSIS2010.pdf}{Comprovativo 2.2.15.1.6}).}
\item {Revisão de 1 artigo para o Jornal ComSIS - Computer Science and Information Systems Journal, Special Issue on Advances in Programming Languages, Faculty of Technical Sciences, Novi Sad, Serbia, volume 4, number 2, Dezembro de 2007 (\href{run:ComprovativosCOCP/revisoes/ComSIS2007.pdf}{Comprovativo 2.2.15.1.7}).}
\end{itemize}

\noindent{Revisão de artigos em Conferências Indexadas}\\
\begin{itemize}
\item {revisão de um artigo para o SLATE 2023.}
\item {revisão de um artigo para o ICPEC 2023.}
\item {Revisão de um artigo para o FedCSIS 2023.}
\item {Revisão de um full paper e de um position paper para FedCSIS 2022.}
\item {Revisão de 1 artigo para o SLATE 2022.}
\item {Revisão de 1 artigo para a CAPSI 2022.}
\item {Revisão de 1 artigo para o ICPEC 2021 - Second International Computer Programming Education Conference.}
\item {Revisão de 2 artigos para o WorldCIST 2021.}
\item {Revisão de 2 artigos para o SEI 2021.}
\item {Revisão de artigo para a workshop ASSE integrada no FedCSIS 2020.}
\item {Revisão de 1 artigo para o SEI 2020.}

\item {Revisão de 2 artigos para o SLATe 2019 - Symposium on Languages, Applications and Technologies, Coimbra, Portugal, Junho de 2019.}
\item {Revisão de 3 artigos para o WorldCIST 2019 - 7th World Conference on Information Systems and Technologies, La Toja, Spain, March 2019.}
\item {Revisão de 1 artigos para o INDIN 2018 - IEEE 16TH International Conference of Industrial Informatics, Porto, 2018.}
\item {Revisão de 1 artigos para o JAIIO 2018 - Jornadas Argentinas de Informátcia, Argentina, 2018.}
\item {Revisão de 1 artigo para o CAPSI 2018, Conferência da Associação Portuguesa de Sistemas de Informação, Santarém, Outubro de 2018.}
\item {Revisão de 2 artigos para o SLATe 2018 - Symposium on Languages, Applications and Technologies, Guimarães, Portugal, Junho de 2018.}
\item {Revisão de 4 artigos para o WorldCIST 2018 - 6th World Conference on Information Systems and Technologies, Naples, Italy, March 2018.}
\item {Revisão de 2 artigos para o WAPL 2017 - 6th Workshop on Advances in Programming Languages integrado no FedCSIS (Federated Conference on Computer Science and Information Systems), Praga, Setembro 2017.}
\item {Revisão de 1 artigo para o CAPSI 2017, Conferência da Associação Portuguesa de Sistemas de Informação, Universidade do Minho, Junho de 2017.}
\item {Revisão de 3 artigos para o SLATE 2017, Symposium on Languages, Applications and Technologies, Vila do Conde, Portugal, Junho de 2017.}
\item {Revisão de 1 artigo para o MDASD 2016 - Workshop on Model Driven Approaches in System Development integrado no FedCSIS - Federated Conference on Computer Science and Information Systems, Gdansk, Polónia, Setembro 2016 (\href{run:ComprovativosCOCP/revisoes/MDASD2016.pdf}{Comprovativo 2.2.15.2.1}). }
\item {Revisão de 2 artigo para o SLATE 2016, Symposium on Languages, Applications and Technologies, Universidade de Maribor, Eslovénia, Junho de 2016 (\href{run:ComprovativosCOCP/revisoes/SLATE2016.pdf}{Comprovativo 2.2.15.2.2}). }
\item {Revisão de 1 artigo para o WorldCIST 2016 - 4th World Conference on Information Systems and Technologies (Workshop on Semantics for Humanities Resources), Recife, Brasil, Março 2016 (\href{run:ComprovativosCOCP/revisoes/WorldCisT2016.pdf}{Comprovativo 2.2.15.2.3}).}
\item {Revisão de 3 artigos para o WAPL 2015 - 5th Workshop on Advances in Programming Languages integrado no FedCSIS (Federated Conference on Computer Science and Information Systems), Polónia, Setembro 2015 (\href{run:ComprovativosCOCP/revisoes/WAPL2015.pdf}{Comprovativo 2.2.15.2.4}).}
\item {Revisão de 3 artigos para o SLATe 2015 - Symposium on Languages, Applications and Technologies, Universidad Complutense de Madrid, Madrid, Espanha, Junho de 2015 (\href{run:ComprovativosCOCP/revisoes/SLATE2015.pdf}{Comprovativo 2.2.15.2.5}).}
\item {Revisão de 2 artigos para o Kes-IDT-14 - 6th International Conference on Intelligent Decision Technologies, Grécia, Junho de 2014 (\href{run:ComprovativosCOCP/revisoes/kesidt2014.pdf}{Comprovativo 2.2.15.2.6}).}
\item {Revisão de 3 artigos para o QUATIC 2014 - 9th International Conference on the Quality of Information and Communications Technology, Guimarães, Setembro 2014 (\href{run:ComprovativosCOCP/revisoes/QUATIC2014.pdf}{Comprovativo 2.2.15.2.7}).}
\item {Revisão de 1 artigo para o MDASD 2014 - Workshop on Model Driven Approaches in System Development integrado no FedCSIS - Federated Conference on Computer Science and Information Systems, Wroclaw, Polónia, Setembro 2014 (\href{run:ComprovativosCOCP/revisoes/MDASD2014.pdf}{Comprovativo 2.2.15.2.8}).}
\item {Revisão de 3 artigo para o WAPL 2013 - 4th Workshop on Advances in Programming Languages integrado no FedCSIS (Federated Conference on Computer Science and Information Systems), Polónia, Setembro 2013 (\href{run:ComprovativosCOCP/revisoes/WAPL2013.pdf}{Comprovativo 2.2.15.2.9}).}
\item {Revisão de 2 artigos para o SLATE 2013 - Symposium on Languages, Applications and Technologies, Faculdade de Ciências da Universidade do Porto, Portugal, Junho de 2013 (\href{run:ComprovativosCOCP/revisoes/SLATE2013.pdf}{Comprovativo 2.2.15.2.10}).}
\item {Revisão de 1 artigo para o SLATE 2012 - Symposium on Languages, Applications and Technologies, Universidade do Minho, Portugal, Junho de 2012 (\href{run:ComprovativosCOCP/revisoes/SLATE2012.pdf}{Comprovativo 2.2.15.2.11}).}
\item {Revisão de 2 artigos para o MDASD 2012 - Workshop on Model Driven Approaches in System Development integrado no FedCSIS - Federated Conference on Computer Science and Information Systems, Wroclaw, Polónia, Setembro 2012 (\href{run:ComprovativosCOCP/revisoes/MDASD2012.pdf}{Comprovativo 2.2.15.2.12}).}
\item {Revisão de 2 artigos para o RAISE 2012- Workshop on REalizing AI Synergies in Software Engineering integrado no ICSE 2012 - 34th International Conference on Software Engineering, Zurich, Junho 2012 (\href{run:ComprovativosCOCP/revisoes/RAISE2012.pdf}{Comprovativo 2.2.15.2.13}).}
\item {Revisão de 3 artigos para o WAPL 2011 - 3rd Workshop on Advances in Programming Languages integrado no FedCSIS (Federated Conference on Computer Science and Information Systems), Polónia, Setembro 2011 (\href{run:ComprovativosCOCP/revisoes/WAPL2011.pdf}{Comprovativo 2.2.15.2.14}).}
\item {Revisão de 2 artigos para o MDASD 2010 - Workshop on Model Driven Approaches in System Development integrado no ADBIS 2010 - 14th East-European Conference on Advances in Databases and Information Systems, Novi Sad, Setembro 2010 (\href{run:ComprovativosCOCP/revisoes/MDASD2010.pdf}{Comprovativo 2.2.15.2.15}).}
\end{itemize}

\noindent{Revisão de artigos em Conferências Não Indexadas}\\
\begin{itemize}
\item {Revisão de 11 resumos e 8 artigos para o CMEA 2016 - Congresso Mundial de Estilos de Aprendizagem, Instituto Politécnico de Bragança, Julho 2016 (\href{run:ComprovativosCOCP/revisoes/CMEA2016.pdf}{Comprovativo 2.2.15.3.1})}
\item {Revisão de 2 artigos para o CORTA 2011 - Compilers, Related Technologies and Applications (integrado no INForum 2011 - Simpósio de Informática), Universidade de Coimbra, Setembro 2011 (\href{run:ComprovativosCOCP/revisoes/CORTA2011.pdf}{Comprovativo 2.2.15.3.2})}
\item {Revisão de 2 artigos para o CORTA 2010 - Compilers, Related Technologies and Applications (integrado no INForum 2010 - Simpósio de Informática), Universidade do Minho, Setembro 2010 (\href{run:ComprovativosCOCP/revisoes/CORTA2010.pdf}{Comprovativo 2.2.15.3.3})}
\item {Revisão de 2 artigos para o CoRTA'08- Compilers, Related Technologies and Applications, Bragança, Julho 2008 (\href{run:ComprovativosCOCP/revisoes/CORTA2008.pdf}{Comprovativo 2.2.15.3.4})}
\item {Revisão de 2 artigos para a conferência WAPL'07 - 1st Workshop on Advances in Programming Languages, que decorreu em Wisla, Polónia, Outubro de 2007 (\href{run:ComprovativosCOCP/revisoes/WAPL2007.pdf}{Comprovativo 2.2.15.3.5})}
\item {Revisão de 5 artigos para a conferência CAPSI - 6ª Conferência da Associação Portuguesa de Sistemas de Informação, que decorreu em Bragança, de 26 a 28 de Outubro de 2005 (\href{run:ComprovativosCOCP/revisoes/CAPSI2005.pdf}{Compro{}vativo 2.2.15.3.6})}
\item {Revisão de 1 artigo para a conferência VDA - Visualization and Data Analysis, que decorreu em San Jose, California, em 18-20 Janeiro 2004 (\href{run:ComprovativosCOCP/revisoes/VDA2004.pdf}{Comprovativo 2.2.15.3.7})}
\end{itemize}

\subsubsection{Membro de Organizações Científicas Nacionais}
\begin{itemize}
\item {Membro colaborador do Grupo Algoritmi da Universidade do Minho desde janeiro de 2018.}
\item {Membro integrado do CeDRI - Centro de Investigação em Digitalização e Robótica Inteligente, IPB, desde Janeiro de 2018.}
\item {Membro integrado do Grupo Algoritmi da Universidade do Minho desde janeiro de 2014 (\href{run:OutrasActCientif/algoritmi.pdf}{Comprovativo 2.2.16.1}).}
\item {Membro integrado do CCTC (Centro de Ciências e Tecnologias de Computação) da Universidade do Minho, desde 2007 até 2014 (\href{run:OutrasActCientif/cctc.pdf}{Comprovativo 2.2.16.2}).}
\item {Membro do Grupo de Especificação e Processamento de Linguagens da Universidade do Minho desde 1994 (\href{run:OutrasActCientif/gepl.pdf}{Comprovativo 2.2.16.3}).}
\end{itemize}

\subsubsection{Outras Atividades Científicas} 
\begin{itemize}

\item {Participação na reunião do projeto OLEAF4VALUE em Ede, Holanda, 19 a 22 de Junho 2023.}
\item {Participação na reunião do projeto OLEAF4VALUE em IZOLA, Eslovénia, 9 e 10 de Fevereiro 2023.}
\item {Mobilidade ICM à Universidade Donja Gorica em Podgorica, Montenegro.}
\item {Mobilidade Erasmus à Universidade Autonoma de Barcelona (UAB),  Computer Vision Center (CVC) e Artificial Intelligence Research Institute (IIIA).}
\item {Participação da reunião técnica do projeto OLEAF4VALUE em Hervás , 23 e 24 de junho 2022.}
\item {Participação no Encontro Ciência 2022 , Centro de Congressos de Lisboa, 16-18 de Maio de 2022.}
\item {Chair de uma keynote no SLATE 2020.}
\item {Chair of the session "Intelligent Systems" at the 16th International Conference on Applied Computing, Cagliari, Italy, Nov 2019. (IADIS)}

\item {Participação no seminário “O futuro do ensino superior de qualidade é blended e flipped: experiências com o modelo de sala de aula invertida na Universidade de Alcalá – Madrid e a extensão do modelo flipped às universidades espanholas”, dinamizado pelo Professor Doutor Alfredo Prieto Martín da Universidade de Alcalá – Madrid, no dia 29 de maio de 2019, na Escola Superior de Tecnologia e Gestão do Instituto Politécnico de Bragança.}
\item {Participação no IV Seminário em Inovação Docente em Automação que decorreu nos dias 10 a 12 de Janeiro de 2018 na Universidade de León.}
\item {Co-autora do best paper do track Human Computer Languages do SLATE 2017.}
\item {Chair de uma sessão do SLATE 2017, 6th Symposium on Languages, Applications and Technologies, Vila do Conde, Junho 2017.}
\item {Prestação de provas públicas no âmbito de um concurso para Prof. Coordenador para o Departamento de Informática e Comunicações do Instituto Politécnico de Bragança em 2007 tendo como resultado o reconhecimento de mérito (\href{run:OutrasActCientif/provas.pdf}{Comprovativo 2.2.17.1}).}
\item {{\em{ General Chair}} da conferência SLATE 2014- 3rd Symposium on Languages, Applications and Technologies, Instituto Politécnico de Bragança, junho 2014 (\href{run:OutrasActCientif/GCslate2014.pdf}{Comprovativo 2.2.17.2}).}
\item {{\em{ General Chair}} da conferência CoRTA 2008- Compilers, Related Technologies and Applications, Escola Superior de Tecnologia e Gestão do Instituto Politécnico de Bragança, Julho 2008 (\href{run:OutrasActCientif/EDcorta08.pdf}{Comprovativo 2.2.17.3}).}
\item {{\em{ Chair}} de uma sessão do SLATE 2015- 4th Symposium on Languages, Appli cations and Technologies, Universidade Complutense de Madrid, junho 2015 (\href{run:OutrasActCientif/CHslate2015.pdf}{Comprovativo 2.2.17.4}).}
\item {{\em{ Chair}} de uma das sessões do CoRTA 2010 integrado no INForum'2010 --- Simpósio de Informática, Universidade do Minho, Braga, 2010 (\href{run:OutrasActCientif/CHCoRTA2010.pdf}{Comprovativo 2.2.17.5}).}
\item {{\em{ Chair}} de uma Working Session sobre "Influence of Synchronized Domain Visualizations on Program Comprehension" no ICPC 2010 - 18th IEEE International Conference on Program Comprehension, Braga, 2010 (\href{run:OutrasActCientif/WSicpc2010.pdf}{Comprovativo 2.2.17.6}).}
\item {Moderação de uma sessão de artigos científicos da CAPSI sobre "Construção de Software", Bragança, 2005 (\href{run:OutrasActCientif/ModeradorCapsi05.pdf}{Comprovativo 2.2.17.7}).}
\item {Participação num curso de formação em C\# (plataforma .net) dado pela Microsoft na Universidade do Minho em Guimarães em Junho de 2005 (5 dias de formação) (\href{run:OutrasActCientif/cursoGuimaraes.pdf}{Comprovativo 2.2.17.8}).}
\end{itemize} 

\subsection{Qualidade de projetos e contratos de investigação}
\subsubsection{Responsável de projectos de investigação e desenvolvimento nacionais}
\begin{itemize}
\item {Responsável pelo projeto {\em{ Program Comprehension by Visual Inspection and Animation}}, financiado pela FCT (POSC/EIA/57662/2004), de julho de 2005 a dezembro de 2007. A equipa é formada também por elementos da Universidade do Minho: Professor Doutor Pedro Henriques, aluna de mestrado Eva Oliveira, aluno de doutoramento Mario Béron, tarefeira Daniela Cruz e um bolseiro (\href{run:Projectos/ComprovativoPCVIA.pdf}{Comprovativo 2.3.1}). Resultaram deste projeto as publicações 10, 12 e 13 da secção 2.2.2 e 37 a 49 da secção 2.2.3.}
\end{itemize}
\subsubsection{Responsável de projectos de investigação e desenvolvimento internacionais}
\begin{itemize}
\item {Responsável Português do projecto bilateral de Cooperação Transnacional entre o Instituto Politécnico de Bragança e a Universidad de San Luis, Argentina, com o título “S3IR - Reforço da segurança dos sistemas de software através de métodos, técnicas e ferramentas de engenharia reversa”, a decorrer de 2018 a 2019.}
\end{itemize}
\subsubsection{Membro de projectos de investigação e desenvolvimento nacionais}
\begin{itemize}
\item {BacchusTech - Integrated Approach for the Valorisation of Winemaking Residues (2021-2023) (\url{http://bacchustech.ipb.pt/})}

\item {Participação no projeto {\em{ Design of Simulated Moving Bed Related Techniques for the Separation of High Added Value Products}}, financiado pela FCT (POSI/EIA/59738/2004), junho de 2005 a junho de 2008. A equipa é formada por docentes da ESTiG de Bragança: elementos do Departamento de Tecnologia Química com a colaboração de elementos do Departamento de Informática e Comunicações e do Departamento de Electrotecnia (\href{run:Projectos/ComprovativoDSMBRT.pdf}{Comprovativo 2.3.2}).}
\end{itemize}
\subsubsection{Membro de projetos de investigação e desenvolvimento internacionais}
\begin{itemize}


\item {Participação no projecto bilateral entre a Universidade de Novi Sad e a Universidade do Minho com o título {\em{ Data Mining based Evaluation of IT Teaching pPractice in Portugal and Serbia}} para os anos 2018 e 2019.}
\item {Participação no projecto {\em{ Construction, Exploration and Transformation of Learning Object Repositories in Specialized Domains}} com a referência TIN2017-88092-R, cujo responsável é o Professor José Luis Sierra Rodríguez da Universidade Complutense de Madrid, financiado pelo governo espanhol para 3 anos de 2018 a 2020. }
\item {Participação como investigador responsável do Instituto Politécnico de Bragança no projeto Erasmus+ 2016-1-ES01-KA203-025452 da Universidade Autónoma de Barcelona, com o título {\em{ Student Profile for Enhancing Engineering Tutoring (SPEET)}}, a decorrer de 2016 a 2018.}
\item {Colaboração no Projeto {\em{ Repositories for Education with Dynamic Reconfigurability in the Humanities}} cujo responsável é o Professor José Luis Sierra Rodríguez da Universidade Complutense de Madrid, financiado pelo governo espanhol para 3 anos de 2015 a 2017 (\href{run:Projectos/Memoria.pdf}{Comprovativo 2.3.3.1}).
Até momento a participação neste projeto consistiu em reuniões periódicas para discutir a criação de repositórios reconfiguráveis, com conteúdos dinâmicos e métodos de consulta baseadas em ontologias.}
\item {Participação no projeto bilateral entre a Universidade do Minho e a Universidade de Maribor (Eslovénia)com o título {\em{ Avaliação da Compreensão de Programas para Domínios Específicos}}, financiado pela FCT, para os anos de 2010 e 2011 (\href{run:Projectos/eslov20102011.pdf}{Comprovativo 2.3.3.2}). Resultaram deste projeto as publicações 8 da secção 2.2.2 e 22 da secção 2.2.3.}
\item {Participação no projeto bilateral entre a Universidade do Minho e a Universidad Nacional de San Luis, (Argentina) com o título {\em{ QUIXOTE - Desenvolvimento de Modelos do Domínio do Problema para inter-relacionar as Vistas Comportamental e Operacional em Sistemas de Software}}, financiado pela FCT, para os anos de 2010 e 2011 (\href{run:Projectos/Quixote.pdf}{Comprovativo 2.3.3.3}). Resultaram deste projeto as publicações 23 a 28, 31 e 33 da secção 2.2.3.}
\item {Participação no projeto bilateral entre a Universidade do Minho e a Universidade de Maribor (Eslovénia)com o título {\em{ Comprehension of Domain Specific Languages}}, financiado pelo CRICES (Ministério da Ciência e do Ensino Superior) para o ano de 2008 e 2009 (\href{run:Projectos/declaracoesUM.pdf}{Comprovativo 2.3.3.4}). Resultaram deste projeto as publicações 11 da secção 2.2.2 e 36 da secção 2.2.3.}
\item {Participação no projeto {\em{ Grammar-based Systems}}, financiado pelo CRICES (Ministério da Ciência e do Ensino Superior) para o ano de 2004, 2005 e 2006, com a colaboração de elementos da Universidade de Maribor, Eslovénia (\href{run:Projectos/GBS.pdf}{Comprovativo 2.3.3.5}). Resultaram deste projeto as publicações 14, 15 e 17 da secção 2.2.2.}
\item {Participação no projeto {\em{ Automatic Generation of Language-based Tools}}, em colaboração com um grupo de investigação da Universidade de Maribor, Eslovénia, liderado pelo Prof. Marjan Mernik. Este projecto decorreu de 2001 a 2003 (3 anos) e teve o financiamento do Ministério da Ciência e do Ensino Superior (\href{run:Projectos/declaracoesUM.pdf}{Comprovativo 2.3.3.6}). Resultaram deste projeto as publicações 18 e 19 da secção 2.2.2 e 53 da secção 2.2.3 e um contributo válido para a conclusão da tese de doutoramento.}
\item {Participação no projeto {\em{ Language and Environment for the Pragmatic Application of Formal Methods}}, em colaboração com um grupo da Open University (U.K.) liderado pela Dr. Leonor Barroca, projeto esse que foi financiado ao abrigo do Protocolo British Council / JNICT para o ano letivo 1995/96 (\href{run:Projectos/declaracoesUM.pdf}{Comprovativo 2.3.3.7}). Esta colaboração surgiu no âmbito do desenvolvimento da tese de mestrado.}
\end{itemize}


\subsubsection{Participação em projectos europeus}
\begin{itemize}
\item {OLEAF4VALUE -Olive leaf multi-product cascade based biorefinery (2021-2024) (\url{https://oleaf4value.eu/})}
\end{itemize}

\subsubsection{Participação em projectos mobilizadores e de co-promoção}
\begin{itemize}
\item {Cognita (JCANAO) -Descriptive and predictive analytics of organizational data to foster new decision support and quality management approaches (2018-2020).}
\item {Bioma -Integrated BIOeconomy solutions for the Mobilization of the Agri-food chain (2020-2023) (\url{https://morecolab.pt/df480-neve-home/projetos/projetos-bioma/})}
\item {Nanostim - Nanomaterials for wearable-based inte-grated biostimulation (2020-2023) (\url{https://nanostim.pt/})}
\item {AquaVITAE - Thermal Water As a Source of Life and Health (2021-2023)}
\end{itemize}

\subsubsection{Próximos projetos aprovados}
\begin{itemize}
\item {Projeto Erasmus+ - ThinkGame - Cooperation to implement Creative Thinking and Gamification for intelligent online training of engineering students Consortium members will cooperate to identify, share and improve online methods with creative thinking and gamification techniques to stimulate learning (Digital Innovation for Social Change) and to support teachers in the development of new learning materials and tools and enhance the LMS adopted by each institution. }
\item {Projeto CETP -2022 ( Clean Energy Transition Partnership) A Business Model-Oriented Platform with Applications for Developing Local Electricity Markets and Accelerating Clean Energy Transition (SMART-LEM) }
\end{itemize}


\subsubsection{Candidaturas a Projetos de Investigação não aprovadas para financiamento mas com avaliação positiva}
\begin{itemize}

\item {Candidaturas a decorrer: projeto Horizon Europe sobre modelos de poupança de energia com a Roménia (Simona); projeto europeu sobre gamificação no ensino de alunos STEM com a Roménia (Adina Cocu); projeto OliveCoFree (José Alberto); projeto Lillian sobre plantas de baixo valor comercial; INTEREG - digitalização de empresas da região (Corunha) com colegas de Mirandela especialistas em conteúdos audiovisuais;}

\item {{\em{ NAU - Exploration of the Problem Domain and creation of a map to enable the navigation between Program \& Problem Models improving Program Comprehension}}, 2008, como investigador responsável (\href{run:Projectos/CandidaturasNaoAprovadas/NAU.pdf}{Comprovativo 2.3.4.1}).}
\item {{\em{ NAU - Formal Exploration of the Problem Domain for Program Comprehension Improvement}}, 2009, como investigador responsável (\href{run:Projectos/CandidaturasNaoAprovadas/NAU2.pdf}{Comprovativo 2.3.4.2}).}
\item {{\em{ Distributed and Intelligent Disturbance Handling in Manufacturin Systems}}, 2008, cujo investigador responsável seria o Prof. Paulo Leitão do Instituto Politécnico de Bragança (\href{run:Projectos/CandidaturasNaoAprovadas/PauloLeitao.pdf}{Comprovativo 2.3.4.3}).}
\item {{\em{ PEM: Policy-based Energy Management}}, 2008, cujo investigador responsável seria o Prof. Rui Pedro Lopes do Instituto Politécnico de Bragança. (\href{run:Projectos/CandidaturasNaoAprovadas/RuiPedro.pdf}{Comprovativo 2.3.4.4}).}
\item {{\em{ COnVEM - Creating Ontology-based Virtual Learning Spaces from Multiple Data Sources}}, 2010, cujo investigador responsável seria o Prof. Pedro Rangel Henriques da Universidade do Minho. (\href{run:Projectos/CandidaturasNaoAprovadas/convem2010.pdf}{Comprovativo 2.3.4.5}).}
\item {{\em{ GENY - Ontology-based Generation of Dymanic Learning Spaces}}, 2012, cujo investigador responsável seria o Prof. Pedro Rangel Henriques da Universidade do Minho. (\href{run:Projectos/CandidaturasNaoAprovadas/GENY2012.pdf}{Comprovativo 2.3.4.6}).}
\item {{\em{ Desenvolvimento de metodologias inovadoras para a melhoria dos projetos educativos do ensino superior politécnico ao nível do ensino e da aprendizagem}} (Processo Nº 138209), submetido à Fundação Calouste Gulbenkian no âmbito dos apoios concedidos, em 2015, para projetos de desenvolvimento do ensino superior - projetos inovadores no domínio educativo, cujo investigador responsável seria o Prof. Vicente Leite do Instituto Politécnico de Bragança. (\href{run:Projectos/CandidaturasNaoAprovadas/Gulbenkian2015.pdf}{Comprovativo 2.3.4.7}).}
\item {{\em{ Plataforma de Inovação Curricular e Pedagógica do IPB}} (Processo Nº 142788), submetido à Fundação Calouste Gulbenkian no âmbito dos apoios a conceder em 2016, para projetos de desenvolvimento do ensino superior - projetos inovadores no domínio educativo, cujo investigador responsável seria o Prof. Vicente Leite do Instituto Politécnico de Bragança. (\href{run:Projectos/CandidaturasNaoAprovadas/Gulbenkian2016.pdf}{Comprovativo 2.3.4.8}).}
\end{itemize}

\subsection{Orientação de trabalhos académicos}
\begin{itemize}
\item {Coorientação da tese de doutoramento \emph{An Ontology Toolkit for Problem Domain Concept Location in Program Comprehension} do aluno Nuno Carvalho da Universidade do Minho, que decorreu durante os anos de 2011 a 2014 e foi defendida em fevereiro de 2015. O outro coorientador desta tese foi o Prof. José João Almeida da Universidade do Minho. (\href{run:CoOrientDout/DecMJVarndaMAPi.pdf}{Comprovativo 2.4}).}
\item {Coorientação da tese de doutoramento \emph{DSL based Automatica Generation of QA Systems}, do aluno Renato Preigschadt de Azevedo.}
\item {Coorientação da tese de doutoramento do aluno Álvaro Neto da Universidade do Minho (receitas de coktails de linguagens de programação).}
\end{itemize}

\subsection{Transferência de Conhecimento}
Foram desenvolvidos projetos em parceria com algumas instituições públicas, privadas e grupos recreativos:
\begin{itemize}
\item {Colaboração com a empresa {\em{ Cosmetek}} de Vila Real no desenvolvimento de uma aplicação de gestão do Laboratório de Análises Químicas, com os alunos Aires Rocha e António Sousa de Engenharia Informática (\href{run:CoOrientTrabalhos/emdesenvolvimento/cosmetek.pdf}{Comprovativo 2.5.1}).}
\item {Colaboração com a Drª Ana Afonso (Diretora do Arquivo Distrital) no desenvolvimento de um sítio dinâmico de apoio ao Arquivo Distrital de Bragança, com o aluno Bruno Queirós de Engenharia Informática, 2007 (\href{run:CoOrientTrabalhos/projFimCursoEI.pdf}{Comprovativo 2.5.2}).}
\item {Construção de uma Aplicação Web para Gestão do Grupo Folclórico de Professores de Braga, com os alunos Ana Daniel e Daniel Carvalho de Engenharia Informática, 2008 (\href{run:CoOrientTrabalhos/projFimCursoEI.pdf}{Comprovativo 2.5.3}).}
\item {Informatização da Gestão da Revista Cultural da Assembleia Distrital a pedido da Drª Ana Afonso (Diretora do Museu Abade Baçal), com o aluno Carlos Matos de Engenharia Informática, 2010 (\href{run:CoOrientTrabalhos/projFimCursoEI.pdf}{Comprovativo 2.5.4}).}
\item {Implementação de um portal para  Museu Abade Baçal (a pedido da Drª Ana Afonso), com a aluna Íris Péres de Engenharia Informática, 2010 (\href{run:CoOrientTrabalhos/projFimCursoEI.pdf}{Comprovativo 2.5.5}).}
\end{itemize}


\subsection{Prémios, bolsas e distinções}
\begin{itemize}
\item {Bolsa de Mestrado da JNICT (FCT) (\href{run:Bolsas/fctMest.pdf}{Comprovativo 2.6.1}).}
\item {Bolsa PRODEP de doutoramento (\href{run:Bolsas/prodepPHD.pdf}{Comprovativo 2.6.2}).}
\end{itemize}

\section{Desempenho Pedagógico}

\subsection{Funções docentes}

\subsubsection{Experiência e qualidade do trabalho pedagógico}

Desde 1995 até hoje, contam-se 35 semestres de experiência letiva e 16 unidades curriculares diferentes lecionadas.

A lista de disciplinas lecionadas pode ser consultada em: \href{run:Disciplinas/Disciplinas.pdf}{Lista de Disciplinas 3.1.1.1}.
A tabela \ref{disciplinas1} resume as disciplinas lecionadas antes da dispensa de serviço docente ao abrigo da Bolsa 1999-2001 para conclusão de doutoramento (\href{run:Bolsas/DispensaProdep.pdf}{Comprovativo 3.1.1.2}).
Da tabela \ref{disciplinas2} constam as disciplinas lecionadas após esse período até à eleição para integrar o Conselho Diretivo da ESTiG.
A tabela \ref{disciplinas3} representa o serviço docente enquanto vice-presidente e subdiretora da ESTiG apesar de estar dispensada desta componente (\href{run:Bolsas/DispensaSubdiretora.pdf}{Comprovativo 3.1.1.3}).\\
A letra R significa responsável pela disciplina, as letras TP representam a lecionação de aulas teórica e práticas e a letra P corresponde a apenas aulas práticas.
A letra O significa orientação de trabalhos. 

\begin{table}[h]
\scriptsize
\centering
    \begin{tabular}[h]{|l||c|c|c|c|}
      \hline
& 95/96 & 96/97 & 97/98 & 98/99\\
\hline
\hline
Processamento de Linguagens (EI) & & & & \href{run:Disciplinas/Fichas/PL9899.pdf}{R} \\
\hline
Algoritmos e Estruturas de Dados (IG e EE) & & & & \href{run:Disciplinas/Fichas/AED9899EEP.pdf}{P}\\
\hline
Introdução à Informática (EM) & & & \href{run:Disciplinas/Fichas/II9798P.pdf}{P} &\\
\hline
Paradigmas da Programação (IG) & & \href{run:Disciplinas/Fichas/PP9697.pdf}{R} & \href{run:Disciplinas/Fichas/PP9798.pdf}{R} &\\
\hline
Aplicações Informáticas (IG) & \href{run:Disciplinas/Fichas/APLICINF9596IG.pdf}{R} & & &\\
\hline
Complementos de Programação(ACG) & & & \href{run:Disciplinas/Fichas/CP9798.pdf}{R} &\\
\hline
Aplicações Informáticas(ACG) & \href{run:Disciplinas/Fichas/AplicInf9596.pdf}{R} & \href{run:Disciplinas/Fichas/AplicInf9697.pdf}{R} & &\\
\hline
Linguagens da Programação II (IG) & \href{run:Disciplinas/Fichas/LPII9596.pdf}{R} & \href{run:Disciplinas/Fichas/LPII9697.pdf}{R} & &\\
\hline
\end{tabular}
    \caption{Lecionação de disciplinas antes da dispensa letiva}
    \label{disciplinas1}
\normalsize
\end{table}

\begin{table}[h]
\scriptsize
\centering
    \begin{tabular}[h]{|l||c|c|c|c|c|c|}
      \hline
& 01/02 & 02/03 & 03/04 & 04/05 & 05/06 & 06/07\\
\hline
\hline
Processamento de Linguagens (EI) & \href{run:Disciplinas/Fichas/PL0102.pdf}{R} & \href{run:Disciplinas/Fichas/PL0203.pdf}{R} & \href{run:Disciplinas/Fichas/PL0304.pdf}{R} & \href{run:Disciplinas/Fichas/PL0405.pdf}{R} & \href{run:Disciplinas/Fichas/PL0506.pdf}{R} & \href{run:Disciplinas/Fichas/PL0607.pdf}{R} \\
\hline
Programação I (TSI ou IG) & & & & \href{run:Disciplinas/Fichas/ProgI0405.pdf}{R} & \href{run:Disciplinas/Fichas/ProgI0506.pdf}{R} & \href{run:Disciplinas/Fichas/ProgI0607.pdf}{R}\\
\hline
Programação II (IG) & & & & & \href{run:Disciplinas/Fichas/ProgII0506.pdf}{R} & \\
\hline
Programação (EE) & & & & & & \href{run:Disciplinas/Fichas/ProgEE0607.pdf}{R}\\
\hline
Projecto Integrado(EI) & & O & & O & O & \\
\hline
Técnicas de Programação (IG e EI) & \href{run:Disciplinas/Fichas/TP0102.pdf}{R} & \href{run:Disciplinas/Fichas/TP0203.pdf}{R} & \href{run:Disciplinas/Fichas/TP0304.pdf}{R} & & \href{run:Disciplinas/Fichas/TP0506P.pdf}{TP} & \\
\hline
Linguagens de Programação (EI) & & \href{run:Disciplinas/Fichas/LP0203.pdf}{TP} & & & & \\ 
\hline
Investigação em Sistemas de Informação (CETSI) & & & & \href{run:Disciplinas/Fichas/ISICETSI0405.pdf}{TP} & & \\ 
\hline
Gestão de Projetos de Desenvolvimento de Software (CETSI) & & & & \href{run:Disciplinas/Fichas/GPDSWCETSI0405.pdf}{R} & & \\ 
\hline
\end{tabular}
    \caption{Lecionação de disciplinas após dispensa letiva}
    \label{disciplinas2}
\normalsize
\end{table}

\begin{table}[h]
\scriptsize
\centering
    \begin{tabular}[h]{|l||c|c|c|c|c|c|c|c|c|}
      \hline
 & 07/08 & 08/09 & 09/10 & 10/11 & 11/12 & 12/13 & 13/14 & 14/15 & 15/16\\
\hline
\hline
Processamento de Linguagens (EI) & \href{run:Disciplinas/Fichas/PL0708.pdf}{R} & \href{run:Disciplinas/Fichas/PL0809.pdf}{R} & & & & & & &\\
\hline
Projecto Integrado(EI) & O & O & O & O & O & O & & & O\\
\hline
Projecto de Informática (IG) & O & & & & & & & &\\
\hline
Dissertação (MSI) & & & & O & & & O & & O\\
\hline
Gestão de Projetos (MSI) & & \href{run:Disciplinas/Fichas/GestaoProjetos.pdf}{R} & & & & & & &\\ 
\hline
Paradigmas da Programação (MSI) & & & \href{run:Disciplinas/Fichas/PP0910.pdf}{R} & \href{run:Disciplinas/Fichas/PP1011.pdf}{R} & \href{run:Disciplinas/Fichas/PP1112.pdf}{R} & \href{run:Disciplinas/Fichas/PP1213.pdf}{R} & \href{run:Disciplinas/Fichas/PP134.pdf}{R} & \href{run:Disciplinas/Fichas/PP1415.pdf}{R} & \href{run:Disciplinas/Fichas/PP1516.pdf}{R}\\
\hline
\end{tabular}
    \caption{Lecionação de disciplinas enquanto subdiretora}
    \label{disciplinas3}
\normalsize
\end{table}


\subsubsection{Qualidade dos elementos elaborados no âmbito das unidades curriculares}
Para as seguintes unidades curriculares foram disponibilizados aos alunos via plataformas eletrónicas os materiais indicados: 
\begin{enumerate}
\item {Programação e Programação I aos cursos de Engenharia Eletrotécnica e Informática de Gestão (\href{run:Disciplinas/Apontamentos/ApontamentosAvaliacao/ProgI/exercicios.pdf}{Exercícios 3.1.2.1});}
\item {Técnicas de Programação aos cursos de Engenharia Informática e Informática de Gestão (\href{run:Disciplinas/Apontamentos/ApontamentosAvaliacao/TP/exerciciosTP.pdf}{Exercícios 3.1.2.2});}
\item {Processamento de Linguagens ao Curso de Engenharia Informática (\href{run:Disciplinas/Apontamentos/ApontamentosAvaliacao/ProcessamentoLinguagens/pling06TEO.pdf}{Sebenta Teórica 3.1.2.3}, \href{run:Disciplinas/Apontamentos/ApontamentosAvaliacao/ProcessamentoLinguagens/plingprof06.pdf}{Sebenta Prática 3.1.2.4});}
\item {Gestão de Projetos de Desenvolvimento de Software (CETSI - Curso de Especialização em Tecnologias e Sistemas de Informação) (\href{run:Disciplinas/Apontamentos/ApontamentosAvaliacao/GPCETSI/acetatosGPSW.pdf}{Slides 3.1.2.5});}
\item {Gestão de Projetos (Mestrado em Sistemas de Informação) (\href{run:Disciplinas/Apontamentos/acetatosGP.pdf}{Slides 3.1.2.6});}
\item {Paradigmas da Programação (Mestrado em Sistemas de Informação) (\href{run:Disciplinas/Apontamentos/PPSebenta2015.pdf}{Sebenta 3.1.2.7}, \href{run:Disciplinas/Apontamentos/PPingles.pdf}{Slides em Inglês 3.1.2.8})}
\item {Curso de Curta Duração em MSProject (\href{run:Disciplinas/Apontamentos/MSProject2010.pdf}{Guião 3.1.2.9});}
\item {Curso de Curta Duração em SQL (\href{run:Disciplinas/Apontamentos/ApontamentosAvaliacao/CursoSQLZamora/SQLPlus.pdf}{Guião 3.1.2.10});}
\end{enumerate}

\subsubsection{Participação na elaboração de conteúdos programáticos}
A letra 'R' nas tabelas \ref{disciplinas1},\ref{disciplinas2} e \ref{disciplinas3} corresponde a fichas de unidade curricular efetuadas.
As disciplinas para quais foi feita uma ficha de unidade curricular diferente pela primeira vez (\href{run:Disciplinas/FUCs.pdf}{Declaração 3.1.3}) são:
\begin{enumerate}
\item {Aplicações Informáticas (Licenciatura de Informática de Gestão - 95/06) (\href{run:Disciplinas/Fichas/APLICINF9596IG.pdf}{FUC 3.1.3.1})}
\item {Aplicações Informáticas (CESE-Auditoria e Controlo de Gestão - 95/96) (\href{run:Disciplinas/Fichas/AplicInf9596.pdf}{FUC 3.1.3.2}) }
\item {Linguagens da Programação II (Licenciatura de Informática de Gestão - 95/96) (\href{run:Disciplinas/Fichas/LPII9596.pdf}{FUC 3.1.3.3}) }
\item {Paradigmas da Programação (Licenciatura de Informática de Gestão - 96/97) (\href{run:Disciplinas/Fichas/PP9697.pdf}{FUC 3.1.3.4})}
\item {Complementos de Programação (CESE-Auditoria e Controlo de Gestão - 97/08) (\href{run:Disciplinas/Fichas/CP9798.pdf}{FUC 3.1.3.5}) }
\item {Processamento de Linguagens (Licenciatura em Engenharia Informática - 98/99) (\href{run:Disciplinas/Fichas/PL9899.pdf}{FUC 3.1.3.6})}
\item {Técnicas de Programação (Licenciaturas de Informática de Gestão e Engenharia Informática - 01/02) (\href{run:Disciplinas/Fichas/TP0102.pdf}{FUC 3.1.3.7})}
\item {Programação I (Licenciatura de Informática de Gestão - 04/05) (\href{run:Disciplinas/Fichas/ProgI0405.pdf}{FUC 3.1.3.8}) }
\item {Gestão de Projetos de Desenvolvimento de Software (Curso de Especialização em Tecnologias e Sistemas de Informação - 04/05) (\href{run:Disciplinas/Fichas/GPDSWCETSI0405.pdf}{FUC 3.1.3.9})}
\item {Programação II (Licenciatura de Informática de Gestão - 05/06) (\href{run:Disciplinas/Fichas/ProgII0506.pdf}{FUC 3.1.3.10}) }
\item {Programação (Licenciatura em Engenharia Eletrotécnica - 06/07) (\href{run:Disciplinas/Fichas/ProgEE0607.pdf}{FUC 3.1.3.11})}
\item {Gestão de Projetos (Mestrado em Sistemas de Informação - 08/09) (\href{run:Disciplinas/Fichas/GestaoProjetos.pdf}{FUC 3.1.3.12})}
\item {Paradigmas da Programação (Mestrado em Sistemas de Informação - 09/10) (\href{run:Disciplinas/Fichas/PP0910.pdf}{FUC 3.1.3.13})}
\end{enumerate}

\subsubsection{Participação na elaboração de planos curriculares}
\begin{enumerate}
\item {Licenciatura em Engenharia Informática em 2004 (\href{run:MissaoIPBoutros/desenhoCursos.pdf}{Comprovativo 3.1.4.1}).}
\item {Curso de Especialização Tecnológica em Sistemas de Informação (CETSI 2º ciclo) em 2004 (\href{run:MissaoIPBoutros/desenhoCursos.pdf}{Comprovativo 3.1.4.2}).}
\item {Mestrado em Informática em 2008 (\href{run:MissaoIPBoutros/desenhoCursos.pdf}{Comprovativo 3.1.4.3}).}
\item {Mestrado em Sistemas de Informação em 2009 (\href{run:MissaoIPBoutros/desenhoCursos.pdf}{Comprovativo 3.1.4.4}).}
\item {Cursos de Especialização Tecnológica da ESTiG em 2007 e 2008 (\href{run:MissaoIPBCargos/tarefasSub.pdf}{Comprovativo 3.1.4.5}).}
\item {Cursos Técnicos Superiores Profissionais da ESTiG em 2015 (\href{run:MissaoIPBCargos/tarefasSub.pdf}{Comprovativo 3.1.4.6}).}
\end{enumerate}

\subsubsection{Publicação e disponibilização de materiais didáticos}
O material pedagógico das disciplinas foi disponibilizado aos alunos numa primeira fase numa página pessoal (\href{run:Virtual/usoPagPessoal.pdf}{Comprovativo 3.1.5.1}) e numa segunda fase, numa plataforma de intranet (\href{run:Virtual/intranet.pdf}{Comprovativo 3.1.5.2}). A partir de 2009 passou a ser usada após a atual plataforma, IPB virtual (\href{run:Virtual/usoIPBVirtual.pdf}{Comprovativo 3.1.5.3}).
Todos os anos o material é actualizado e disponibilizado por estes meios.


\subsubsection{Inovação Pedagógica}

Dado o carácter prático das disciplinas de programação tem sido, desde sempre, adotada uma metodologia de {\em{ problem-based learning}} no sentido das matérias serem introduzidas através da realização de exercícios práticos promovendo em alturas oportunas o recurso a suporte teórico. A avaliação, seguindo o mesmo raciocínio, baseia-se numa metodologia de {\em{ project-based learning}} onde alguns dos projetos propostos correspondem a solicitações externas à disciplina potenciando a interdisciplinariedade.
No caso da disciplina de Processamento de Linguagens a metodologia adotada pela docente deu origem à publicação de um capítulo em livro e um artigo em conferência internacional:
\begin{itemize}
\item {Pereira M.J.V., Oliveira N., Da Cruz D., Henriques P.R., {\bf{ An effective way to teaching Language Processing Courses}} para o livro {\bf{ Innovative teaching strategies and New Learning Paradigms in Computer Programming}}, pp. 131-152, IGI Global, 2014(\href{run:Publicacoes/PublicacoesSCOPUS.pdf}{SCOPUS})(\href{run:Publicacoes/publicacoes/78.pdf}{pdf 2.2.1}).}
\item {Pereira M.J.V., Oliveira N., Cruz D., Henriques P.R., {\bf{ Choosing Grammars to Support Language Processing Courses}}, SLATE 2013 - Symposium on Languages, Applications and Technologies, Faculdade de Ciências da Universidade do Porto, Junho de 2013, OpenAccess Series in Informatics (OASIcs), pp. 155-168, vol 29, 2013 (\href{run:Publicacoes/PublicacoesSCOPUS.pdf}{SCOPUS},\href{run:Publicacoes/ComprovativosDBLP.pdf}{DBLP}) (\href{run:Publicacoes/publicacoes/71.pdf}{pdf 2.2.3.12})}
\end{itemize}

No âmbito das disciplinas de programação alguns trabalhos foram desenvolvidos pela docente por forma a aumentar a motivação dos alunos e consequentemente o seu sucesso escolar. Esse esforço foi feito em disciplinas de programação (dadas por outros docentes) aplicando metodologias publicadas em:
\begin{itemize}
\item {Cruz D., Henriques P.R., Pereira M.J.V., {\bf{ Constructing program animations using a pattern-based approach}}, ComSIS - Computer Science and Information Systems Journal, Special Issue on Advances in Programming Languages, Faculty of Technical Sciences, Novi Sad, Serbia, Volume 4, Number 2, pp. 99-116, Dec 2007. (ISSN: 1820-0214) (\href{run:Publicacoes/ComprovativosDBLP.pdf}{DBLP}) (\href{run:Publicacoes/publicacoes/25.pdf}{pdf 2.2.2.13})}
\item {Chuchulashvili M., Goziashvili N., Pereira M.J.V, Lopes R., {\bf{ Micro atividades para a Aprendizagem de Programação}}, CMEA 2016 - VII Congresso Mundial de Estilos de Aprendizagem, Instituto Politécnico de Bragança, Julho 2016. (\href{run:Publicacoes/publicacoes/91.pdf}{pdf 2.2.3.1})}
\item {Tavares P., Gomes E.F., Pereira M.J.V., Henriques P.R., {\bf{ Técnicas para aumentar o Envolvimento dos Alunos na Aprendizagem da Programação}}, CMEA 2016 - VII Congresso Mundial de Estilos de Aprendizagem, Instituto Politécnico de Bragança, Julho 2016. (\href{run:Publicacoes/publicacoes/89.pdf}{pdf 2.2.3.3})}
\end{itemize}
No âmbito da disciplina de Métodos Numéricos dada por docentes do departamento de Matemática foram realizadas experiências baseadas no sistema publicado em:
\begin{itemize}
\item {Alves L., Balsa C., Pereira M.J.V., {\bf{ Simulador Gráfico de Algoritmos Matemáticos}}, CMEA 2016 - VII Congresso Mundial de Estilos de Aprendizagem, Instituto Politécnico de Bragança, Julho 2016. (\href{run:Publicacoes/publicacoes/90.pdf}{pdf 2.2.3.2})}
\item {Balsa C., Alves L., Pereira M.J.V., Rodrigues P.J. and Lopes R., {\bf{ Graphical Simulation of Numerical Algorithms, An approach based on code instrumentation and java technologies}}, CSEDU 2012 - 4th International Conference on Computer Supported Education, Porto, pp. 164-169, Abril 2012. (\href{run:Publicacoes/PublicacoesSCOPUS.pdf}{SCOPUS}) (\href{run:Publicacoes/publicacoes/66.pdf}{pdf 2.2.3.17})}
\item {Balsa C., Alves L., Pereira M.J.V., Rodrigues P., {\bf{ Graphical simulator of mathematical algorithms (GraSMA)}}, Proceedings of IASK International Conference Teaching and Learning, pp. 594-600, Seville, Spain, Dec 2010. (\href{run:Publicacoes/publicacoes/84.pdf}{pdf 2.2.3.29})}
\end{itemize}

No âmbito da disciplina de Paradigmas da Programação do Mestrado em Sistemas de Informação, lecionada pela docente nos últimos anos, foram feitas algumas experiências inovadoras baseadas em inquéritos que permitem distinguir linguagens de programação de domínio geral e linguagens de programação de domínio específico, conforme o artigo:
\begin{itemize}
\item {Kosar T., Oliveira N., Mernik M., Pereira M.J.V., Crepinsek M., Cruz D., Henriques P.R., {\bf{ Comparing General-Purpose and Domain-Specific Languages: An Empirical Study}}, ComSIS - Computer Science and Information Systems Journal, Special Issue on Compilers, Related Technologies and Applications, Vol. 7, Number 2, pp. 247-264, April 2010. (ISSN: 1820-0214) (\href{run:Publicacoes/ComprovativosISI.pdf}{ISI}, \href{run:Publicacoes/PublicacoesSCOPUS.pdf}{SCOPUS},\href{run:Publicacoes/ComprovativosDBLP.pdf}{DBLP}) (Ci=35, AC=1) (Cs=67, AC=1) (\href{run:Publicacoes/publicacoes/49.pdf}{pdf 2.2.2.8})}
\end{itemize}
Foi também possível dar oportunidade aos alunos de experimentarem o uso de linguagens visuais usando o sistema publicado em:
\begin{itemize}
\item {Oliveira N., Pereira M.J.V., Henriques P.R., Cruz D., Cramer B., {\bf{ VisualLISA: A Visual Environment to Develop Attribute Grammars}}, ComSIS - Computer Science and Information Systems Journal, Special Issue on Compilers, Related Technologies and Applications, Vol. 7, Number 2, pp. 265-290, April 2010. (ISSN: 1820-0214) (\href{run:Publicacoes/ComprovativosISI.pdf}{ISI}, \href{run:Publicacoes/PublicacoesSCOPUS.pdf}{SCOPUS},\href{run:Publicacoes/ComprovativosDBLP.pdf}{DBLP}) (Ci=3, AC=1) (Cs=10, AC=2) (\href{run:Publicacoes/publicacoes/52.pdf}{pdf 2.2.2.9})}
\end{itemize}



\subsection{Participação em júris de provas}
\subsubsection{Arguente de Provas de Doutoramento}
\begin{enumerate}
\item {Arguente das Provas de Doutoramento de Leandro Oliveira Freitas da Universidade do Minho em Março de 2022 com o título "Uncertainty and Incompleteness Handling in Context-Aware Systems" e orientação do Prof. Pedro Henriques.}
\item {Arguente das Provas de Doutoramento de Daniel Rodríguez Cerezo da "Facultad de Informática de la Universidad Complutense de Madrid" em Fevereiro de 2016 (\href{run:JuriProvas/TesisDaniel.pdf}{Comprovativo 3.2.1.1}).}
\item {Arguente das Provas de Doutoramento requeridas pelo Mestre Paulo Jorge Teixeira Matos na Universidade do Minho em Maio de 2005 (\href{run:JuriProvas/JuriDoutoramentoPauloMatos.pdf}{Comprovativo 3.2.1.2}).}
\end{enumerate}
\subsubsection{Arguente de Provas de Mestrado}
\begin{enumerate}
\item {Arguência da tese de mestrado do aluno Lucas Guedes Barboza de dupla diplomação UTFPR-IPB (com o título Comparing Sentiment Analysis Tools on GitHub Project Discussions), Junho 2023.}
\item {Arguência da tese de mestrado do aluno Pedro Mimoso da Universidade do Minho (com o título Automatic Generation of ASTs from a Programming Language Grammar e orientador Pedro Henriques), Janeiro de 2023.(zoom 14:30 e 16 valores)}
\item {Arguência da tese de mestrado do aluno Carlos Barbosa da Universidade do Minho (com o título Meta Data Migrator e orientador José Carlos Ramalho), Abril 2022.}
\item {Arguente das Provas de Mestrado do aluno Tiago Santos (IPB) - "Ontologias para detecção de fraude em meio académico", dezembro de 2021.}
\item {Arguente das Provas de Mestrado do aluno Alexandre Dias (UM) - "ONTODL+ - An ontology description language and its compiler", Universidade do Minho, Setembro 2021.}
\item {Arguente das Provas de Mestrado do aluno Tarlon Gomes (UTFPR) - "Modelo de previsão do sucesso académico utilizando métodos de Learning Analytics", Mestrado em Informática, IPB, Fevereiro 2021.}
\item {Arguente das Provas de Mestrado do aluno Cassiano Andrade (UTFPR) - "“Web Application for the Mentoring Academy Program”", Mestrado em sistemas de Informação, Instituto Politécnico de Bragança, Fevereiro de 2020.}
\item {Arguente das Provas de Mestrado de Pedro Miguel Lopes Pereira, "Automatic fix of Source Code Vulnerabilities", aluno da Universidade do Minho, dezembro de 2019.}
\item {Arguente das Provas de Mestrado do Tiago Sanches Franco (UTFPR)- "Big Data Analytics para a classificação do risco de abandono escolar em cursos do ensino superior", Mestrado em sistemas de Informação, Instituto Politécnico de Bragança, Março de 2019.}
\item {Arguente das Provas de Mestrado de João Carlos Alves da Cruz, aluno da Universidade do Minho, dezembro de 2015 (\href{run:JuriProvas/ArguenteJoaoCruz.pdf}{Comprovativo 3.2.2.1}).}
\item {Arguente das Provas de Mestrado de Pedro José Ribeiro Moreira, aluno da Universidade do Minho, dezembro de 2014 (\href{run:JuriProvas/ArguentePedroMoreira.pdf}{Comprovativo 3.2.2.2}).}
\item {Arguente das Provas de Mestrado da aluna Sónia Pires Fernandes, do Mestrado em Sistemas de Informação 2009/2010, com o título "As TIC nas Juntas de Freguesia dos Conselhos de Bragança", 2010 (\href{run:JuriProvas/ArguenteSoniaPires.pdf}{Comprovativo 3.2.2.3}).}
\end{enumerate}
\subsubsection{Membro do júri em Provas de Doutoramento}
\begin{enumerate}
\item {Membro do júri de avaliação de pré$-$tese da aluna de doutoramento da Universidade do Minho, Paula Tavares (docente do ISEP), reunido a 28 de Fevereiro de 2014 (\href{run:JuriProvas/ArguentePaulaTavares.pdf}{Comprovativo 3.2.}3.1).}
\item {Membro do júri para reconhecimento de habilitações do Grau de Doutor de Mario Marcelo Beron, reunido a 25 de Maio de 2012 na Universidade do Minho (\href{run:JuriProvas/RecMarioBeron.pdf}{Comprovativo 3.2.3.2}).}
\end{enumerate}
\subsubsection{Membro do júri em Provas de Mestrado}
\begin{enumerate}
\item {Presidente de 3 júris de provas de defesa de tese de Mestrado em Informática, Junho 2023. }
\item {Membro do júri das Provas de Mestrado na qualidade de orientadora do aluno Henrique Marcuzzo de dupla diplomação UTFPR-IPB, junho 2023.}
\item {Presidente do júri de 1 prova de defesa de Tese de Mestrado em Informática, IPB, dezembro 2022. (Aluno: Afonso Rocha ) }
\item {Presidente do júri de 1 prova de defesa de Tese de Mestrado em Informática, IPB, dezembro 2022. (Aluno: André Matos) }
\item {Presidente do júri de 1 prova de defesa de Tese de Mestrado em Informática, IPB, dezembro 2022. (Aluno: Seifeldien Sameh Soliman Mahmoud Soliman) }
\item {Presidente do júri de 1 prova de defesa de Tese de Mestrado em Informática, IPB, julho 2022. (Aluno: João Paulo Baptista Pereira)}
\item {Presidente do júri de 1 prova de defesa de Tese de Mestrado em Informática, IPB, dezembro 2021.}
\item {Presidente do júri de 2 provas de defesa de Teses de Mestrado em Contabilidade e Finanças, IPB, dezembro 2021.}
\item {Presidente do júri de 1 prova de defesa de Tese de Mestrado em Informática, IPB, junho 2021.}
\item {Presidente do júri de 3 provas de defesa de Teses de Mestrado em Sistemas de Informação, IPB, abril e julho de 2020.}
\item {Presidente do júri de 2 provas de defesa de Teses de Mestrado em Contabilidade e Finanças, IPB, dezembro 2019.}
\item {Presidente do júri de 3 provas de defesa de Teses de Mestrado em Sistemas de Informação, setembro 2019.}
\item {Membro do júri das Provas de Mestrado na qualidade de orientadora da aluna Joana Miguel da Universidade do Minho, dezembro 2019.}
\item {Membro do júri das Provas de Mestrado na qualidade de orientadora do aluno Martinho Aragão da Universidade do Minho, dezembro 2019.}
\item {Março de 2019 - Membro do júri de Reconhecimento de grau de mestre de Nilton Hideki Takagi do Programa de Pós-graduação em Informática da Universidade Federal de Paraíba; de Cristiane Maria Santos Ferreira do Programa de Pós-graduação em Computação da Universidade Federa Fluminense e de Nidhi Sharma do Master of Computer Applications da Rajasthan Technical University, India (este último foi indeferido).}
\item {Membro do júri das Provas de Mestrado  na qualidade de orientadora da aluna Gohar Tomeyan da Arménia, Julho de 2017.}
\item {Membro do júri das Provas de Mestrado na qualidade de orientadora do aluno Marcos Ramos da Universidade do Minho, maio 2017.}
\item {Membro do júri das Provas de Mestrado na qualidade de orientadora do aluno Daniel Novais da Universidade do Minho, dezembro 2016.}
\item {Membro do júri das Provas de Mestrado na qualidade de orientadora do aluno  da Universidade do Minho, dezembro 2014 }
\item {Presidente do júri da Tese de Mestrado da aluna Elena Gudorlava (Georgia) do Mestrado em Sistemas de Informação do IPB, julho 2016. (comprovativos em Investigação2016/TemasMSI)}
\item {Membro do júri das Provas de Mestrado na qualidade de orientadora da aluna Mariami Chuchulashvili do Mestrado em Sistemas de Informação do IPB, julho 2016.(comprovativos em Investigação2016/TemasMSI)}
\item {Membro do júri das Provas de Mestrado na qualidade de orientadora da aluna Nino Goziashvili do Mestrado em Sistemas de Informação do IPB, julho 2016.(comprovativos em Investigação2016/TemasMSI)}
\item {Presidente do júri da defesa de uma Tese do Mestrado em Engenharia da Construção, dezembro 2015 (\href{run:JuriProvas/PresidenteMEC.pdf}{Comprovativo 3.2.4.1}).}
\item {Presidente do júri de 18 Provas de Mestrado no âmbito do Mestrado em Sistemas de Informação do IPB desde 2010 até hoje (\href{run:JuriProvas/TeseMSI.pdf}{Comprovativo 3.2.4.2}).}
\item {Membro do júri das Provas de Mestrado na qualidade de orientadora do aluno Nuno Pereira da Universidade do Minho, dezembro 2014 (\href{run:JuriProvas/juriNunoPereira.pdf}{Comprovativo 3.2.4.3}).}
\item {Membro do júri das Provas de Mestrado na qualidade de orientadora do aluno João Fonseca da Universidade do Minho, outubro 2014 (\href{run:JuriProvas/juriJoaoFonseca.pdf}{Comprovativo 3.2.4.4}).}
\item {Membro do júri de Provas de Mestrado requeridas pela Licenciada Eva Oliveira na Universidade do Minho em Setembro de 2006 (\href{run:JuriProvas/JuriMestradoEva.pdf}{Comprovativo 3.2.4.5}).}
\end{enumerate}

\subsection{Congressos e Conferências sobre docência}
\begin{itemize}
\item {Algumas palestras organizadas pelo mentoring academy do IPB.}
\item {Formação pedagógica “Testes online e integridade académica”, Junho 2020.}
\item {Formação pedagógica “Apresentação e análise dos resultados do inquérito de avaliação do modelo de ensino remoto do IPB”, Junho 2020.}
\item {Formação Pedagógica para docentes "Criação de apresentações visuais de alto impacto I", 2 horas, ESTiG-IPB, Junho 2019.}
\item {Membro da Comissão Organizadora do VII Congresso Mundial de Estilos de Aprendizagem:Educação, Tecnologias e Inovação, julho de 2016, Instituto Politécnico de Bragança (\href{run:CongressoDocencia/cmea2016.pdf}{Comprovativo 3.3.1}).}
\item {Participação no XXI Encontro da Associação das Universidades de Língua Portuguesa (AULP), junho de 2011, Instituto Politécnico de Bragança (\href{run:CongressoDocencia/AULP.pdf}{Comprovativo 3.3.2}).}
\end{itemize}

\subsection{Dedicação e qualidade das atividades docentes}
\subsubsection{Inquéritos}
\begin{itemize}
\item {Resultado dos inquéritos pedagógicos até 2009 (\href{run:Inqueritos/inqueritosATE2009.pdf}{Comprovativo 3.4.1.1}).}
\item {Resultado dos inquéritos pedagógicos 2010 (\href{run:Inqueritos/relatorioPedagogico2010.pdf}{Comprovativo 3.4.1.2}).}
\item {Resultado dos inquéritos pedagógicos 2011 (\href{run:Inqueritos/relatorioPedagogico2011.pdf}{Comprovativo3.4.1.3}).}
\item {Resultado dos inquéritos pedagógicos 2012 (\href{run:Inqueritos/relatorioPedagogico2012.pdf}{Comprovativo3.4.1.4}).}
\item {Resultado dos inquéritos pedagógicos 2013 (\href{run:Inqueritos/relatorioPedagogico2013.pdf}{Comprovativo3.4.1.5}).}
\item {Resultado dos inquéritos pedagógicos 2014 (\href{run:Inqueritos/relatorioPedagogico2014.pdf}{Comprovativo3.4.1.6}).}
\end{itemize}

\subsubsection{Utilização de ferramentas de elearning}
A plataforma IPB virtual (\url{https://virtual.ipb.pt}) é baseada em áreas às quais está associada uma lista de participantes o que permite ser usada como ferramenta de elearning. No caso das áreas correspondentes a disciplinas, o professor e os alunos constituem o grupo de participantes dentro do qual é feita a partilha de informação. Em cada uma dessas áreas é possível fazer o upload de recursos, o envio e receção de mensagens entre participantes, o depósito de documentos no cacifo digital por parte dos alunos (entrega de trabalhos) entre outras facilidades (\href{run:Virtual/mensagensvirtual.pdf}{Comprovativo 3.4.2.1}).\\
Os recursos que normalmente são disponibilizados consistem em conteúdos teóricos, fichas de trabalho, enunciados de trabalhos práticos, diapositivos e outras ligações importantes. 
A comunicação entre a docente e os alunos em contexto fora da sala de aula foi sempre assegurada por email pessoal mas também por mensagens e avisos nesta plataforma de elearning.
A plataforma dá também acesso à plataforma de sumários o que permite aos alunos terem acesso ao plano da aula em antecipação. 
As fichas de disciplina são atualizadas e depositadas no Guia ECTS (\url{https://apps.ipb.pt/guia$-$ects}) no início de cada ano letivo. Os horários de atendimentos são disponibilizados na página da ESTiG no início de cada semestre.\\
A disponibilização de toda esta informação e a facilidade de comunicação entre participantes permite que alunos que não têm acesso a aulas presenciais possam acompanhar a disciplina à distância.
Mais se acrescenta que estas plataformas são usadas pela docente desde a sua implementação (comprovativos na secção 3.1.5), o mesmo acontece com a plataforma de pautas eletrónicas desde o ano letivo 2008/2009 e a plataforma de sumários desde 2009/2010  (\href{run:Virtual/pautassumarios.pdf}{Declaração 3.4.2.2}).

\subsubsection{Internacionalização da atividade pedagógica}
\begin{itemize}
\item {Lecionação de módulos de SQL (3,5 horas) ao Curso de Verão em Bases de Dados para a Web na Fundação Rei Afonso Henriques, em Zamora, em Julho durante os anos de 2007, 2008, 2009 (\href{run:MissaoIPBoutros/AfonsoHenriques.pdf}{Comprovativo 3.4.3.1}).}
\item {Lecionação de um módulo de SQL (3 horas) ao Curso de Verão em Bases de Dados para a Web na Fundação Rei Afonso Henriques, em Bragança, Julho de 2010 (\href{run:MissaoIPBoutros/AfonsoHenriques2.pdf}{Comprovativo 3.4.3.2}).}
\item {Orientação de estágios de verão no âmbito de um protocolo entre o Instituto Politécnico de Bragança e o B.K. Birla Institute of Engineering and Technology (Pilani, Índia) no ano letivo 2010/2011 (\href{run:CoOrientTrabalhos/AlunoIndiano.pdf}{Comprovativo 3.4.3.3}).}
\item {Orientação de estágios de verão no âmbito de um protocolo entre o Instituto Politécnico de Bragança e a École Nationale Supérieur d'Electrotechnique, d'Eletronique, d'Informatique,
d'Hydraulique et de Télècommunications (ENSEEIHT) de l'Institut Nationale Polytechnique de Toulouse (ENSEEIHT-INPT) no ano letivo 2014/2015 (\href{run:CoOrientTrabalhos/AlunosFrancesesBalsa.pdf}{Comprovativo 3.4.3.4}).}
\item {Participação no Projeto Erasmus+ International Credit Mobility (ICM) concretizada pela participação em reuniões com as Instituições Parceiras incluindo visitas a duas Universidades Georgianas para divulgação do projeto e cooperação na área da Engenharia Informática e do Mestrado em Sistemas de Informação, nomeadamente na coorientação de duas teses de mestrado em Sistemas de Informação, abril 2016 (\href{run:MissaoIPBoutros/ICM.pdf}{Comprovativo 3.4.3.5}).}
\item {Participação no Projeto Erasmus+ International Credit Mobility (ICM) concretizada pela participação em reuniões com as Instituições Parceiras incluindo visitas às várias faculdades da Universidade Nacional da Arménia para divulgação do projeto e cooperação na área da Engenharia Informática e do Mestrado em Sistemas de Informação, nomeadamente na coorientação de uma tese de mestrado em Sistemas de Informação, maio 2017.}
\item {Visita aos campos da UTFPR de Curitiba e de Ponta Grossa para promoção de investigação conjunta no âmbito do CEDRI e estabelecimento de novos protocolos de dupla diplomação em especial para os cursos de Engenharia Informática e Mestrado em Sistemas de Informação, Junho 2018.}
\item {Participação no Projeto Erasmus+ International Credit Mobility (ICM) concretizada pela participação em reuniões com as Instituições Parceiras incluindo visitas à Universidade de Biskek, Quirguistão para divulgação do projeto e cooperação na área da Engenharia Informática e do Mestrado em Sistemas de Informação, Março 2019.}
\end{itemize}

\subsection{Orientação de dissertações e trabalhos conducentes a grau académico}

\subsubsection{Teses de Mestrado}
\begin{enumerate}
\item {Coorientação da Tese de Mestrado de Eva Oliveira, com o título \emph{Características de um Sistema de Visualização para Compreensão de Programas Web}, no Departamento de Informática da Universidade do Minho, defendido em Setembro de 2006 (\href{run:CoOrientTrabalhos/OrientacaoEva.pdf}{Comprovativo 3.5.1.1}).}
\item {Coorientação da Tese de Mestrado de Nuno Oliveira, com o título \emph{Improving Program Comprehension Tools for DSLs} no Departamento de Informática da Universidade do Minho, defendido em Dezembro de 2009 (\href{run:CoOrientTrabalhos/NunoOliveira.pdf}{Comprovativo 3.5.1.2}).}
\item {Coorientação da Tese de Mestrado da aluna Albertina Neto do Mestrado em Sistemas de Informação do IPB, com o título \emph{O uso das TIC nas escolas do 1\º ciclo do Ensino Básico do Distrito de Bragança}, defendido em Dezembro de 2010 (\href{run:CoOrientTrabalhos/TesesMSI.pdf}{Comprovativo 3.5.1.3}).}
\item {Coorientação da Tese de Mestrado da aluna Ermelinda Afonso Gonçalves do Mestrado em Sistemas de Informação do IPB, com o título \emph{Utilização de Ferramentas Web pelos professores do Ensino Secundário para Acompanhamento Escolar dos Alunos em Contexto Fora da Sala de Aula}, defendido em Dezembro de 2013 (\href{run:CoOrientTrabalhos/TesesMSI.pdf}{Comprovativo 3.5.1.4}).}
\item {Coorientação da Tese de Mestrado do aluno Nuno Pereira do Departamento de Informática da Universidade do Minho, com o título {\em{ Concept location over the system dependency graph}}, defendida em Dezembro de 2015 (\href{run:CoOrientTrabalhos/OrientNunoPereira.pdf}{Comprovativo 3.5.1.5}).}
\item {Coorientação da Tese de Mestrado do aluno João Fonseca do Departamento de Informática da Universidade do Minho, com o título {\em{ Criação de DSL com base em ontologias}}, defendida em Outubro de 2015(\href{run:CoOrientTrabalhos/OrientJoaoFonseca.pdf}{Comprovativo 3.5.1.6}).}
\item {Coorientação da Tese de Mestrado da aluna Mariami Tchutchulashvili da Georgia, no âmbito do Projeto ICM (Erasmus+) do Instituto Politécnico de Bragança, com o tema {\em{ Design and implementation of an online pedagogical methodology for Java programmer student beginners}}, defesa prevista para Julho de 2016 (\href{run:CoOrientTrabalhos/emdesenvolvimento/Mariami.pdf}{Comprovativo 3.5.1.7}).}
\item {Coorientação da Tese de Mestrado da aluna Nino Godziashvili da Georgia, no âmbito do Projeto ICM (Erasmus+) do Instituto Politécnico de Bragança, com o tema {\em{ Design and implementation of an online pedagogical methodology for C programmer student beginners}}, Julho de 2016 (\href{run:CoOrientTrabalhos/emdesenvolvimento/Nino.pdf}{Comprovativo 3.5.1.8}).}
\item {Coorientação da Tese de Mestrado do aluno Marcos Ramos do Departamento de Informática da Universidade do Minho, com tema, {\em{ Querying and Answering Systems for Programming Languages}}, maio de 2017 (\href{run:CoOrientTrabalhos/emdesenvolvimento/MarcosRamos.pdf}{Comprovativo 3.5.1.9}).}
\item {Coorientação da Tese de Mestrado do aluno Daniel Novais do Departamento de Informática da Universidade do Minho, com tema, {\em{ Programmer Profiling through Code Analysis}},dezembro de 2016 (\href{run:CoOrientTrabalhos/emdesenvolvimento/DanielNovais.pdf}{Comprovativo 3.5.1.10}).}
\item {Orientação da Tese de Mestrado da aluna Gohar Tomeyan da Arménia, no âmbito do Projeto ICM (Erasmus+) do Instituto Politécnico de Bragança, com o tema {\em{ A text uniqueness checking system for Armenian Language}}, Julho de 2017. }
\item {Coorientação da Tese de Mestrado do aluno Johnny Lima (UTFPR), no âmbito do projeto de dupla diplomação do Mestrado em Sistemas de Informação, com o título {\em{ Modelo para Classificação do Risco de Abandono Escolar em Cursos de Engenharia com Base em Métodos de Academic Analytics}}, Junho 2018.}
\item {Coorientação da Tese de Mestrado do aluno Martinho Aragão do Departamento de Informática da Universidade do Minho, com tema, {\em{ Intelligent feedback system for programmer’s profile improvement}},dezembro de 2019.}
\item {Coorientação da Tese de Mestrado da aluna Joana Margarida Miguel do Departamento de Informática da Universidade do Minho, com tema, {\em{ Privas: assuring the privacy in database exploring systems}},dezembro de 2019.}
\item {Coorientação da Tese de Mestrado do aluno João Gris (UTFPR), no âmbito do projeto de dupla diplomação do Mestrado em Sistemas de Informação, com o título {\em{ Course Direction Support Information System}}, dezembro 2019.}
\item {Coorientação da Tese de Mestrado do aluno João Reis do Departamento de Informática da Universidade do Minho, com tema, {\em{ GSD: A Web Application for Teacher Timetable Management}}, Março de 2020.}
\item {Coorientação da Tese de Mestrado da aluna Marcela Almeida (CEFET-Minas Gerais), no âmbito do projeto de dupla diplomação do Mestrado em Sistemas de Informação, com o título {\em{ EasyCoding - Metodologia para apoiar o aprendizado de programação}}, Junho 2020.}
\item {Orientação do Caio Nakai (IPB), com o título {\em{ Digitalização de Processos de Gestão de Espaços}}, defendida em julho 2020.}
\item {Coorientação do Diogo Soares (UM), com o título {\em{ Python-Tutor on Program Comprehension}}, defendida em dezembro de 2020.}
\item {Coorientação do Manuel Sousa (UM), com o título {\em{ Applying Attribute Grammars to teach Linguistic Rules}}, defendida em julho de 2021.}
\item {Orientação do Benarous Farouq (IPB), com o título {\em{ Almond Variety Detection using Deep Learning}}, defendida em dezembro 2020.}
\item {Orientação da Rafaela Pinho (UM), com o título {\em{ Biometric Analysis of Behaviours in Serious Games}}, defendida em Abril de 2022.}
\item {Orientação da tese de mestrado do Mestrado em Informática do aluno Beka Kokhodze da Georgia com o titulo “Marketplace for Circular Bioeconomy”, defendida em Julho de 2022.}
\item {Coorientação da tese de mestrado da aluna Mariana de Oliveira Pereira  da Universidade do Minho, defendida a 3 de Março de 2023, com o título Avaliação Automática de Testes de Atenção e de Acuidade Visual (protótipo da mariana \url{https://daisy.epl.di.uminho.pt/}).}
\item {Orientação da tese de mestrado do Henrique Marcuzzo (UTFPR - IPB), com o título {\em{ Implementation of a Thermal-based Food Recommendation System}}, Junho 2023 (18 valores). }
\item {Orientação do Marco Silva (IPB) (em curso - desenvolvimento de uma palicação web de auto-avaliação de sustentabilidade ambiental para empresas vinícolas).}
\item {Orientação do Paulo Pereira (IPB) (em curso).}
\item {Orientação do Daniel Farina (UTFPR-IPB) (em curso).}
\item {Orientação do Guilherme Tonello (UTFPR-IPB) (em curso).}
\item {Orientação do Eduardo Silva (UM).}
\item {Orientação do Gustavo Lourença (UM).}
\item {Orientação do Tiago Freitas (UM).}
\item {Orientação do Martim Bento (UM).}


\end{enumerate}
\subsubsection{Projetos de Fim de Curso}
\begin{enumerate}
\item {Orientação de um estágio da Licenciatura em Engenharia Informática, na empresa IT Sector em 2023 (Ricardo Vieira).}
\item {Projeto de fim de curso de Engenharia Informática da Universidade do Minho, João Guilherme Rodrigues Reis, João Carlos Viana Pereira Marques, João Domingos Pereira Barbosa, Nuno Alexandre Pereira Machado, Desenvolvimento de uma aplicação para gestão de viaturas oficiais de uma instituição do ensino superior.
(uniauto.di.uminho.pt/ endereço:gestor@mail.com e password: gestor)}
\item {Coorientadora do projeto de fim de curso com o título "Compreensão de programas em Visual-Basic" do aluno Moisés Ramires da Licenciatura em Ciências da Computação da Universidade do Minho, no ano letivo de 2018/2019.}
\item {Coorientadora do projeto de fim de curso com o título "Animador de Algoritmos Específicos"dos alunos José Simões e Pedro Fernandes da Licenciatura em Ciências da Computação da Universidade do Minho, no ano letivo de 2015/2016.}
\item {Coorientadora do projeto de fim de curso com o título "Animador de Programas em C (semelhante ao Jeliot)"dos alunos António Oliveira e Pedro Lemos da Licenciatura em Ciências da Computação da Universidade do Minho, no ano letivo de 2015/2016.}
\item {Orientadora de projeto de fim de curso de Engenharia Informática tendo por objetivo o desenvolvimento de um "Sistema cliente/servidor em Java para Avaliação Automática de Alunos", Lúcia Torres Teixeira e Gina Rodrigues, durante o ano letivo 2004/2005 (\href{run:CoOrientTrabalhos/projFimCursoEI.pdf}{Comprovativo 3.5.2.1}).}
\item {Orientadora de projeto de fim de curso de Tecnologias e Sistemas de Informação cujo objetivo foi o desenvolvimento de um "Sistema de Interfaces para um Simulador de Processos de Polimerização", Márcia Sousa e Elisabete Freitas durante o ano letivo 2005/2006 (\href{run:CoOrientTrabalhos/projFimCursoTSI.pdf}{Comprovativo 3.5.2.2}).}
\item {Orientadora de projeto de fim de curso de Engenharia Informática cujo objetivo foi a "Construção e teste de uso de um animador de programas no ensino da programação em C", Pedro Franco, durante o ano letivo 2006/2007 (\href{run:CoOrientTrabalhos/projFimCursoEI.pdf}{Comprovativo 3.5.2.3}).}
\item {Orientadora de projeto de fim de curso de Engenharia Informática cujo objetivo foi a construção de um "Jogo de Matemática Interactivo (Geometria)", Nuno Afonso, durante o ano letivo 2006/2007 (\href{run:CoOrientTrabalhos/projFimCursoEI.pdf}{Comprovativo 3.5.2.4}).}
\item {Orientadora de projeto de fim de curso de Engenharia Informática cujo objetivo foi a construção de um "Jogo de Matemática Interactivo (Cálculo de Operações para os anos 1 e 2)", Tiago Martins, durante o ano letivo 2006/2007 (\href{run:CoOrientTrabalhos/projFimCursoEI.pdf}{Comprovativo 3.5.2.5}).}
\item {Orientadora de projeto de fim de curso de Engenharia Informática cujo objetivo foi a construção de um "Jogo de Matemática Interactivo (Cálculo de Operações para os anos 3 e 4)", Ana Silva e Francisco Gonçalves, durante o ano letivo 2007/2008 (\href{run:CoOrientTrabalhos/projFimCursoEI.pdf}{Comprovativo 3.5.2.6}).}
\item {Orientadora de projeto de fim de curso de Engenharia Informática cujo objetivo foi a construção de um "Portal Web - Juventude S.Pedro,  Liliana Fernandes e Ricardo Cruz, durante o ano letivo 2007/2008 (\href{run:CoOrientTrabalhos/projFimCursoEI.pdf}{Comprovativo 3.5.2.7}).}
\item {Orientadora de projeto de fim de curso de Engenharia Informática cujo objetivo foi a implementação de uma aplicação de "Balanced Scored Card", Teresa Vaz, durante o ano letivo 2008/2009 (\href{run:CoOrientTrabalhos/projFimCursoEI.pdf}{Comprovativo 3.5.2.8}).}
\end{enumerate}
\subsubsection{Orientação de Estagiários e Bolseiros}
\begin{enumerate}
\item {Orientação do bolseiro Gustavo Quieregato do projeto Aquae Vitae.}
\item {Orientadora responsável pela estância do Iván Arias Rodriguez da Universidade Complutense de Madrid no âmbito da sua tese de doutoramento sobre redes neuronais.}
\item {Coorientadora de um estágio de uma aluna da Licenciatura em Matemática e Ciências da Computação da Universidade do Minho, com o título {\em{ LISS, A linguagem e o ambiente de programação}}, Daniela Cruz, durante o ano letivo 2006/2007  (\href{run:Projectos/declaracoesUM.pdf}{Comprovativo 3.5.3.1}).}
\item {Coorientadora de um estágio de um aluno da Licenciatura em Engenharia de Sistemas e Informática da Universidade do Minho, com o título {\em{ Implementação do WebAppViewer: uma ferramenta para compreender aplicações Web}}, Ruben Fonseca, durante o ano letivo 2006/2007 (\href{run:Projectos/declaracoesUM.pdf}{Comprovativo 3.5.3.2}).}
\item {Coorientadora de um projecto do Mestrado de Informática da UM para desenvolvimento do sistema {\em{ VisualLISA}} do aluno Nuno Oliveira, durante o ano letivo 2008/2009 (\href{run:Projectos/declaracoesUM.pdf}{Comprovativo 3.5.3.3}).}
\item {Coorientadora de um bolseiro de iniciação à atividade científica (Pedro Faria) da Universidade do Minho cujo trabalho se intitula {\em{ Visual Languages: Comparative study of generators}}, durante o ano letivo 2009/2010 (\href{run:Projectos/declaracoesUM.pdf}{Comprovativo 3.5.3.4}).}
\item {Orientação de um estágio a decorrer na secretaria de alunos da ESTiG de uma aluna do Curso de Especialização Tecnológica de Secretariado e Assessoria Administrativa em 2013 (\href{run:CoOrientTrabalhos/estagiosESE.pdf}{Comprovativo 3.5.3.5}).}
\item {Orientação de dois estágios (um a decorrer na secretaria de alunos da ESTiG e outro na Gabinete de Relações com o Exterior da ESTiG) de duas alunas do Curso de Especialização Tecnológica de Secretariado e Assessoria Administrativa em 2014 (\href{run:CoOrientTrabalhos/estagiosESE.pdf}{Comprovativo 3.5.3.6}).}
\item {Orientação de um estágio a decorrer no Gabinete de Relações com o Exterior da ESTiG de uma aluna do Curso de Especialização Tecnológica de Secretariado e Assessoria Administrativa em 2015 (\href{run:CoOrientTrabalhos/estagiosESE.pdf}{Comprovativo 3.5.}3.7).}
\end{enumerate}

\section{Outras atividades consideradas relevantes para a missão do IPB}

\subsection{Exercícios de cargos e funções académicas}
\subsubsection{Participação em órgãos colegiais}
\begin{itemize}
\item {Membro do primeiro CTC (Conselho Técnico-científico) da ESTiG desde Julho de 2009 até hoje (\href{run:MissaoIPBCargos/ctc.pdf}{Comprovativo 4.1.1.1}). }
\item {Vice-presidente do Conselho Diretivo da Escola Superior de Tecnologia e Gestão de Bragança, desde Março de 2007 até Junho de 2009 (\href{run:MissaoIPBCargos/subdiretora.pdf}{Comprovativo 4.1.1.2}).}
\item {Coordenadora do Departamento de Informática e Comunicações da ESTiG, desde Setembro de 2003 até Setembro de 2005 (\href{run:MissaoIPBCargos/coordDIC.pdf}{Comprovativo 4.1.1.3}).}
\item {Membro da Comissão Científica do Mestrado em Sistemas de Informação desde a primeira edição (novembro de 2008) até hoje (\href{run:MissaoIPBCargos/msi.pdf}{Comprovativo 4.1.1.4}).}
\item {Membro do Colégio Eleitoral para a eleição do Presidente do Instituto Politécnico de Bragança, desde Novembro de 2005 a Fevereiro de 2006 (\href{run:MissaoIPBCargos/colegioeleitoral.pdf}{Comprovativo 4.1.1.5}).}
\item {Membro do Conselho Pedagógico da ESTiG de dezembro 1998 até hoje (\href{run:MissaoIPBCargos/CP.pdf}{Comprovativo 4.1.1.6}).  }
\item {Membro da Comissão de Curso do CTESP em Desenvolvimento de Software.}
\item {Vogal da Comissão Executiva do Conselho Pedagógico da ESTiG desde dezembro de 2002 até novembro de 2004 (\href{run:MissaoIPBCargos/CP.pdf}{Comprovativo 4.1.1.7}).}
\item {Membro e Vice-presidente da Assembleia de Representantes da ESTiG, desde Julho de 2005 até Maio de 2009 (\href{run:MissaoIPBCargos/AssembleiaRepresentantes.pdf}{Comprovativo 4.1.1.8}).}
\item {Comissão Diretiva do CETSI (Curso de Especialização em Tecnologias e Sistemas de Informação) durante o ano letivo 2004-2005 (\href{run:MissaoIPBCargos/avaliacaoReestruturacao.pdf}{Comprovativo 4.1.1.9}).}
\item {Membro da Comissão de Curso de Engenharia Informática da ESTiG desde dezembro de 2002 até dezembro de 2006 (\href{run:MissaoIPBCargos/CP.pdf}{Comprovativo 4.1.1.10}).}
\item {Representante da área de informática no Comissão de Curso de Engenharia Eletrotécnica da ESTiG entre dezembro de 1998 e dezembro de 2002 (\href{run:MissaoIPBCargos/CP.pdf}{Comprovativo 4.1.1.11}).}
\item {Representante da área de Informática no Conselho de Curso de Auditoria e Controlo de Gestão da ESTiG (96/97 e 97/98) (\href{run:MissaoIPBCargos/InfACG.pdf}{Comprovativo 4.1.1.12}).}
\item {Membro da Comissão de Autoavaliação da Licenciatura Bietápica em Engenharia Informática desde abril de 2004 a dezembro de 2004 (\href{run:MissaoIPBCargos/CALEI.pdf}{Comprovativo 4.1.1.13}).}
\item {Membro por inerência do Conselho Científico da ESTiG de junho de 1999 até julho de 2009 (\href{run:MissaoIPBCargos/ConselhoCientifico.pdf}{Comprovativo 4.1.1.14}). }
\end{itemize}

\subsubsection{Outros cargos e funções por designação}
\begin{itemize}
\item {Subdiretora (por nomeação) da ESTiG desde 2009 até novembro de 2022 (\href{run:MissaoIPBCargos/subdiretora.pdf}{Comprovativo 4.1.2.1}).}
\item {Membro da Comissão de Horários durante o ano letivo 1996/1997 (\href{run:MissaoIPBoutros/comissaoHorarios.pdf}{Comprovativo 4.1.2.2}).}
\item {Membro do júri de provas de maiores de 23 em 2006 (\href{run:JuriProvas/Maiores23.pdf}{Comprovativo 4.1.2.3}).}
\item {Membro do júri de creditação de experiência profissional em 2008 (\href{run:JuriProvas/JuriExpProf.pdf}{Comprovativo 4.1.2.4}).}
\item {Membro do júri de atribuição de uma bolsa para Técnico de Investigação na área de Informática em 2010 (\href{run:JuriProvas/BolsaCIMO.pdf}{Comprovativo 4.1.2.5}).}
\item {Presidente do júri de atribuição de uma bolsa de Gestão de Ciência e Tecnologia na área de Informática de Gestão em 2012 (\href{run:JuriProvas/BolsaIG.pdf}{Comprovativo 4.1.2.6}).}
\item {Presidente do júri de atribuição de uma bolsa de Gestão de Ciência e Tecnologia na área de Engenharia Mecânica em 2015 (\href{run:JuriProvas/BolsaFabLab.pdf}{Comprovativo 4.1.2.7}).}
\item {Presidente do Júri do processo concursal para dirigente intermédio do 3º grau para os Serviços Académicos do IPB, julho de 2017.}
\item {Presidente do Júri de quatro concursos de recrutamento de professores coordenadores para as áreas de Informática, Engenharia Química, Engenharia Civil e Engenharia Eletrotécnica e Computadores.}
\item {Presidente da Comissão Eleitoral para eleição do primeiro coordenador do Centro de Investigação em Digitalização e Robótica Inteligente (CeDRI).}
\item {Membro do júri para contratação de um bolseiro de doutoramento (FCT) para o CEDRI, 2020.}
\item {Membro do júri para recrutamento de Prof. Adjunto do IPCA, 2021.}
\item {Presidente do Júri do processo concursal para dirigente intermédio do 3º grau para os Serviços Académicos do IPB, Janeiro 2022.}
\item {Membro do júri para contratação de um bolseiro de doutoramento (FCT) para o CEDRI, 2022.}
\item {Membro do júri de um concurso internacional para recrutamento de Prof. Adjunto para a Escola Superior de Tecnologia e Gestão de Águeda (ESTGA) da Universidade de Aveiro, 2023.}
\item {Presidente do júri do concurso para recrutamento de quatro professores coordenadores para a área de Engenharia Civil da ESTiG, IPB, 2023 (a decorrer).}
\item {Membro do júri do concurso para Prof. Adjunto do DIC, 2023 (a decorrer).}
\item {Membro do júri do concurso para recrutamento de professor adjunto para o ISEP, 2023 (a decorrer).}
\item {Membro do júri do concurso para recrutamento de professor adjunto para o IPVC, 2023 (a decorrer).}
\item {Membro de júris de contratação de bolseiros para os projetos Aquae Vitae e Bacchustech.}
\end{itemize}

\subsection{Funções como membro da Direção da ESTiG}
Como membro da Direção da ESTiG foram produzidos relatórios de atividades (\href{run:MissaoIPBCargos/RelActividadesMJ.pdf}{Comprovativo 4.2.1}) mas as principais tarefas realizadas todos os anos desde 2007 estão descritas abaixo (\href{run:MissaoIPBCargos/tarefasSub.pdf}{Declaração 4.2.2}) :
\begin{itemize}
\item {Responsável pela Gestão Pedagógica de todos os Cursos da ESTiG desde 2007.}
\item {Responsável pela Gestão Pedagógica do CET em Contabilidade e Gestão, deslocalizado em Mogadouro, em 2010/2011.}
\item {Responsável pela Gestão Pedagógica do CET em Condução de Obra, deslocalizado em Mogadouro, em 2011/2012 e 2012/201}
\item {Responsável pela Gestão Pedagógica do CET em Contabilidade e Gestão, deslocalizado em Penafiel, em 2012/2013, 2013/2014 e 2014/2015.}
\item {Responsável pela Gestão Pedagógica do CET em Energias Renováveis, deslocalizado em Amarante, em 2012/2013.}
\item {Responsável pela Gestão Pedagógica do CET em Energias Renováveis, deslocalizado em Chaves, em 2013/2014 e 2014/2015.}
\item {Responsável pela Gestão Pedagógica do CET em Contabilidade e Gestão, deslocalizado em Chaves, em 2014/2015.}
\item {Responsável pela Gestão Pedagógica do CTESP em Energias Renováveis e Instalações Elétricas, deslocalizado em Chaves, em 2015/2016.}
\item {Responsável pela Gestão Pedagógica do CTESP em Prospeção Mineral e Geotécnica, deslocalizado em Torre de Moncorvo, em 2015/2016.}
\item {Responsável pela seriação dos alunos candidatos aos Cursos de Especialização Tecnológica (CET), desde 2007 até 2014.}
\item {Responsável pela seriação dos alunos candidatos aos Cursos Técnicos Superiores Profissionais (CTESP), desde 2015. }
\item {Responsável pela Elaboração dos Horários Escolares, desde 2007. }
\item {Responsável pelo funcionamento da Secretaria de Alunos da ESTiG, desde 2007. }
\item {Membro da Comissão de Avaliação do Desempenho do Pessoal não docente da ESTiG, desde 2007. }
\item {Responsável pela validação das férias dos docentes da ESTiG, desde 2014. }
\item {Membro das Comissões de Autoavaliação dos cursos da ESTiG em avaliação, desde 2007. }
\item {Responsável pela implementação de estratégias de combate ao insucesso escolar nos cursos da ESTiG, em especial dos CTESPs, desde 2015. }
\item {Responsável pelas propostas dos Novos Cursos Técnicos Superiores Profissionais, cujos dossiers que acompanham os pedidos de registo foram enviados para a DGES durante o ano de 2015. }
\item {Participação na elaboração de propostas de Cursos de Especialização Tecnológica da ESTiG em 2007 e 2008.}
\item {Responsável pela organização das reuniões das Comissões Externas de Avaliação (CAE), aquando das avaliações dos cursos da ESTiG, desde 2011. }
\item {Membro de 21 júris de seleção de Pessoal Docente Especialmente Contratado (PDEC), desde 2010. }
\item {Participação no Desenho do curso Engenharia Informática para a Universidade Politécnica Fernando Marcelino no Huambo-Angola em 2007.}
\item {Responsável pela candidatura dos cursos de Engenharias da ESTiG para inclusão no índice FEANI, no ano 2012/2013. }
\item {Participação na organização da STG (Semana da Tecnologia e Gestão), desde 2013. }
\item {Responsável pelo acolhimento de alunos internacionais e Erasmus da ESTiG no que diz respeito a facilitar contacto direto com os professores, horário escolar e na identificação de disciplinas a lecionar em língua inglesa, desde 2011.}
\item {Vogal do Júri do Concurso para Técnico de 2ª Classe estagiário, área Administrativa, conforme Edital nº 618/2007, publicado no Diário da República, 2ª Série, nº 145, em Julho de 2007 (\href{run:MissaoIPBoutros/Tecnico2Classe.pdf}{Comprovativo 4.2.3}).}
\item {Presidente do Júri do Concurso Interno de Acesso Limitado para o preenchimento de dois lugares de Operário Altamente Qualificado Principal, conforme aviso nº 8/2007 (\href{run:MissaoIPBoutros/Operario.pdf}{Comprovativo 4.2.4}).}
\item {Presidente do Júri do Concurso Interno de Acesso Limitado para um lugar de Técnico Principal da Carreira Técnica do IPB, conforme o aviso nº 11/2008 (\href{run:MissaoIPBoutros/TecnicoPrincipal.pdf}{Comprovativo 4.2.5}).}
\item {Presidente do Júri do Concurso Interno de Acesso Limitado para dois lugares de Assistente Administrativo Especialista da Carreira Administrativa do IPB, conforme o aviso nº 14/2008 (\href{run:MissaoIPBoutros/AssAdmin.pdf}{Comprovativo 4.2.6}).}
\item {Responsável pela comunicação de novos planos de estudos à DGES na sequênica da avaliação da A3ES}
\item {Responsável pela gestão das candidaturas de CTESP, Mestrados e Internacionais.}
\item {Responsável pela interação com os serviços académicos}
\item {Membro da equipa de elaboração de inquéritos online aos docentes e alunos na pandemia COVID}
\item {Responsável pela tabela de creditações CTESP -Licenciaturas}
\item {Responsável pela envio da reformulação de CTESPs, Licenciaturas e Mestrados à DGES como consequência de novos financiamentos de CTESPs e da avaliação da A3ES de Licenciaturas e Mestrados.}

\end{itemize}

\subsection{Atividades de Extensão}
\begin{itemize}
\item {Emissão de parecer sobre o período experimental de uma Prof. Adjunta do IPVC.}
\item {Participação no Projecto Cognita para apoio científico ao desenvolvimento de software da empresa JCANAO de Viana do Castelo durante o ano de 2019.}
\item {Responsável pela interação da ESTiG com a Altice Labs com o objectivo de dinamizar o protocolo estabelecido entre as duas entidades em novembro de 2018.}
\item {Responsável pela interação da ESTiG com a empresa Faurecia durante os anos letivos 2014-2015 e
2015-2016 (\href{run:MissaoIPBoutros/FAURECIA.pdf}{Declaração 4.3.1}), no que diz respeito às seguintes atividades:}
\begin{itemize}
\item {realização de reuniões preparatórias para recolha de possíveis projetos a serem desenvolvidos em parceria;}
\item {realização de sessões de divulgação desses projetos aos alunos da ESTiG;}
\item {atribuição desses projetos a disciplinas/docentes ou a alunos finalistas para serem desenvolvidos como projetos de fim de curso;}
\item {estabelecimento de acordos de formação, planos de trabalhos e horários de entrada e saída da fábrica;}
\item {realização de reuniões periódicas para verificação do estado de desenvolvimento dos projetos;}
\item {organização de sessões na ESTiG e na Faurecia de apresentação de trabalhos;}
\item {receção de pessoal técnico da Faurecia para realização de palestras no âmbito de eventos realizados pela ESTiG;}
\item {receção de pessoal técnico da Faurecia para lecionação de aulas no âmbito de disciplinas específicas;}
\item {divulgação de estágios profissionais a realizar na Faurecia.}
\end{itemize}
\item {Responsável pela organização do terceiro dia da Semana da Tecnologia e Gestão (STG 2015) da ESTiG com o tema {\em{ Cooperação com empresas e Empregabilidade}} aberto ao público em geral em 23 de Abril de 2015 (\href{run:MissaoIPBoutros/STG2015.pdf}{Declaração 4.3.2}).}
\end{itemize}

\subsection{Atividades de Serviço à Comunidade}
\begin{itemize}
\item {Lecionação de aulas de Introdução à Informática à Universidade Sénior de Bragança durante o segundo semestre do ano letivo 2006/2007 (30h), na Escola Superior de Tecnologia e Gestão de Bragança (\href{run:MissaoIPBoutros/univSenior.pdf}{Comprovativo 4.4}).}
\end{itemize} 

\subsection{Atividades de formação contínua de profissionais}
\begin{itemize}
\item {Lecionação de um curso de Ms-Project - Gestão de Projetos (15 horas) na Escola Superior de Tecnologia e Gestão de Bragança, Dezembro de 2012 (\href{run:MissaoIPBoutros/msproject2012.pdf}{Comprovativo 4.5.1}).}
\item {Lecionação de um curso de Ms-Project - Gestão de Projetos (10 horas) na Escola Superior de Tecnologia e Gestão de Bragança, Novembro 2008 (\href{run:MissaoIPBoutros/msproject2008.pdf}{Comprovativo 4.5.2}).}
\item {Lecionação de um Módulo de Ms-Project - Gestão de Projetos (15 horas) a um grupo de engenheiros da empresa FAURECIA, em 2005, na Escola Superior de Tecnologia e Gestão de Bragança (\href{run:MissaoIPBoutros/CursoFaurecia.pdf}{Comprovativo 4.5.3}).}
\item {Lecionação de um Mini-curso de Java (4h) aos professores de informática do 3o ciclo (grupo 39) em profissionalização em serviço na Universidade do Minho, Instituto de Estudos da Criança, Maio de 1998.}
\end{itemize}

\subsection{Atividades de participação em projetos e ações de interesse social}
\begin{itemize}
\item {Participação no projeto Ciência@Bragança, nº 16911, financiado pelo Ciência Viva - Agência Nacional para a Cultura Científica e Tecnológica (\href{run:MissaoIPBoutros/CienciaBraganca.pdf}{Comprovativo 4.6.1}).}
\item {Participação em Atividades de Divulgação do IPB em Escolas Secundárias em 2007 e 2009 (\href{run:MissaoIPBoutros/divulgacao.pdf}{Comprovativo 4.6.2}).}
\end{itemize}

\end{document} 