
\documentclass[11pt]{article}
\usepackage{portuges}
\usepackage[pdftex]{hyperref}
\hypersetup{pdfnewwindow=true}
\usepackage[latin1]{inputenc}
\title{CURRICULUM VITAE \\
 
\vspace{3cm}
\vspace{3cm}}
\author{Maria João Tinoco Varanda Pereira}
\date{
\today} 
\begin{document} 
\maketitle 
\newpage
\section{Desempenho Pedagógico}
\subsection{Funções docentes}
\subsubsection{Experiência e qualidade do trabalho pedagógico}Desde 1995 até hoje, contam-se 35 semestres de experiência letiva e 16 unidades curriculares diferentes lecionadas.A lista de disciplinas lecionadas pode ser consultada em: 
\href{run:Disciplinas/Disciplinas.pdf}{Lista de Disciplinas 3.1.1.1}.A tabela 
\ref{disciplinas1} resume as disciplinas lecionadas antes da dispensa de serviço docente ao abrigo da Bolsa 1999-2001 para conclusão de doutoramento (
\href{run:Bolsas/DispensaProdep.pdf}{Comprovativo 3.1.1.2}).Da tabela 
\ref{disciplinas2} constam as disciplinas lecionadas após esse período até à eleição para integrar o Conselho Diretivo da ESTiG.A tabela 
\ref{disciplinas3} representa o serviço docente enquanto vice-presidente e subdiretora da ESTiG apesar de estar dispensada desta componente (
\href{run:Bolsas/DispensaSubdiretora.pdf}{Comprovativo 3.1.1.3}). \\
A letra R significa responsável pela disciplina, as letras TP representam a lecionação de aulas teórica e práticas e a letra P corresponde a apenas aulas práticas.A letra O significa orientação de trabalhos. 
\begin{table}[h]
\scriptsize
\centering
\begin{tabular}[h]{|l||c|c|c|c|}
\hline& 95/96 & 96/97 & 97/98 & 98/99 \\

\hline
\hlineProcessamento de Linguagens (EI) & & & & 
\href{run:Disciplinas/Fichas/PL9899.pdf}{R}  \\

\hlineAlgoritmos e Estruturas de Dados (IG e EE) & & & & 
\href{run:Disciplinas/Fichas/AED9899EEP.pdf}{P} \\

\hlineIntrodução à Informática (EM) & & & 
\href{run:Disciplinas/Fichas/II9798P.pdf}{P} & \\

\hlineParadigmas da Programação (IG) & & 
\href{run:Disciplinas/Fichas/PP9697.pdf}{R} & 
\href{run:Disciplinas/Fichas/PP9798.pdf}{R} & \\

\hlineAplicações Informáticas (IG) & 
\href{run:Disciplinas/Fichas/APLICINF9596IG.pdf}{R} & & & \\

\hlineComplementos de Programação(ACG) & & & 
\href{run:Disciplinas/Fichas/CP9798.pdf}{R} & \\

\hlineAplicações Informáticas(ACG) & 
\href{run:Disciplinas/Fichas/AplicInf9596.pdf}{R} & 
\href{run:Disciplinas/Fichas/AplicInf9697.pdf}{R} & & \\

\hlineLinguagens da Programação II (IG) & 
\href{run:Disciplinas/Fichas/LPII9596.pdf}{R} & 
\href{run:Disciplinas/Fichas/LPII9697.pdf}{R} & & \\

\hline
\end{tabular}
\caption{Lecionação de disciplinas antes da dispensa letiva}
\label{disciplinas1}
\normalsize
\end{table}
\begin{table}[h]
\scriptsize
\centering
\begin{tabular}[h]{|l||c|c|c|c|c|c|}
\hline& 01/02 & 02/03 & 03/04 & 04/05 & 05/06 & 06/07 \\

\hline
\hlineProcessamento de Linguagens (EI) & 
\href{run:Disciplinas/Fichas/PL0102.pdf}{R} & 
\href{run:Disciplinas/Fichas/PL0203.pdf}{R} & 
\href{run:Disciplinas/Fichas/PL0304.pdf}{R} & 
\href{run:Disciplinas/Fichas/PL0405.pdf}{R} & 
\href{run:Disciplinas/Fichas/PL0506.pdf}{R} & 
\href{run:Disciplinas/Fichas/PL0607.pdf}{R}  \\

\hlineProgramação I (TSI ou IG) & & & & 
\href{run:Disciplinas/Fichas/ProgI0405.pdf}{R} & 
\href{run:Disciplinas/Fichas/ProgI0506.pdf}{R} & 
\href{run:Disciplinas/Fichas/ProgI0607.pdf}{R} \\

\hlineProgramação II (IG) & & & & & 
\href{run:Disciplinas/Fichas/ProgII0506.pdf}{R} &  \\

\hlineProgramação (EE) & & & & & & 
\href{run:Disciplinas/Fichas/ProgEE0607.pdf}{R} \\

\hlineProjecto Integrado(EI) & & O & & O & O &  \\

\hlineTécnicas de Programação (IG e EI) & 
\href{run:Disciplinas/Fichas/TP0102.pdf}{R} & 
\href{run:Disciplinas/Fichas/TP0203.pdf}{R} & 
\href{run:Disciplinas/Fichas/TP0304.pdf}{R} & & 
\href{run:Disciplinas/Fichas/TP0506P.pdf}{TP} &  \\

\hlineLinguagens de Programação (EI) & & 
\href{run:Disciplinas/Fichas/LP0203.pdf}{TP} & & & &  \\
 
\hlineInvestigação em Sistemas de Informação (CETSI) & & & & 
\href{run:Disciplinas/Fichas/ISICETSI0405.pdf}{TP} & &  \\
 
\hlineGestão de Projetos de Desenvolvimento de Software (CETSI) & & & & 
\href{run:Disciplinas/Fichas/GPDSWCETSI0405.pdf}{R} & &  \\
 
\hline
\end{tabular}
\caption{Lecionação de disciplinas após dispensa letiva}
\label{disciplinas2}
\normalsize
\end{table}
\begin{table}[h]
\scriptsize
\centering
\begin{tabular}[h]{|l||c|c|c|c|c|c|c|c|c|}
\hline& 07/08 & 08/09 & 09/10 & 10/11 & 11/12 & 12/13 & 13/14 & 14/15 & 15/16 \\

\hline
\hlineProcessamento de Linguagens (EI) & 
\href{run:Disciplinas/Fichas/PL0708.pdf}{R} & 
\href{run:Disciplinas/Fichas/PL0809.pdf}{R} & & & & & & & \\

\hlineProjecto Integrado(EI) & O & O & O & O & O & O & & & O \\

\hlineProjecto de Informática (IG) & O & & & & & & & & \\

\hlineDissertação (MSI) & & & & O & & & O & & O \\

\hlineGestão de Projetos (MSI) & & 
\href{run:Disciplinas/Fichas/GestaoProjetos.pdf}{R} & & & & & & & \\
 
\hlineParadigmas da Programação (MSI) & & & 
\href{run:Disciplinas/Fichas/PP0910.pdf}{R} & 
\href{run:Disciplinas/Fichas/PP1011.pdf}{R} & 
\href{run:Disciplinas/Fichas/PP1112.pdf}{R} & 
\href{run:Disciplinas/Fichas/PP1213.pdf}{R} & 
\href{run:Disciplinas/Fichas/PP134.pdf}{R} & 
\href{run:Disciplinas/Fichas/PP1415.pdf}{R} & 
\href{run:Disciplinas/Fichas/PP1516.pdf}{R} \\

\hline
\end{tabular}
\caption{Lecionação de disciplinas enquanto subdiretora}
\label{disciplinas3}
\normalsize
\end{table}
\subsubsection{Qualidade dos elementos elaborados no âmbito das unidades curriculares}Para as seguintes unidades curriculares foram disponibilizados aos alunos via plataformas eletrónicas os materiais indicados: 
\begin{enumerate}
\item{Programação e Programação I aos cursos de Engenharia Eletrotécnica e Informática de Gestão (
\href{run:Disciplinas/Apontamentos/ApontamentosAvaliacao/ProgI/exercicios.pdf}{Exercícios 3.1.2.1});}
\item{Técnicas de Programação aos cursos de Engenharia Informática e Informática de Gestão (
\href{run:Disciplinas/Apontamentos/ApontamentosAvaliacao/TP/exerciciosTP.pdf}{Exercícios 3.1.2.2});}
\item{Processamento de Linguagens ao Curso de Engenharia Informática (
\href{run:Disciplinas/Apontamentos/ApontamentosAvaliacao/ProcessamentoLinguagens/pling06TEO.pdf}{Sebenta Teórica 3.1.2.3}, 
\href{run:Disciplinas/Apontamentos/ApontamentosAvaliacao/ProcessamentoLinguagens/plingprof06.pdf}{Sebenta Prática 3.1.2.4});}
\item{Gestão de Projetos de Desenvolvimento de Software (CETSI - Curso de Especialização em Tecnologias e Sistemas de Informação) (
\href{run:Disciplinas/Apontamentos/ApontamentosAvaliacao/GPCETSI/acetatosGPSW.pdf}{Slides 3.1.2.5});}
\item{Gestão de Projetos (Mestrado em Sistemas de Informação) (
\href{run:Disciplinas/Apontamentos/acetatosGP.pdf}{Slides 3.1.2.6});}
\item{Paradigmas da Programação (Mestrado em Sistemas de Informação) (
\href{run:Disciplinas/Apontamentos/PPSebenta2015.pdf}{Sebenta 3.1.2.7}, 
\href{run:Disciplinas/Apontamentos/PPingles.pdf}{Slides em Inglês 3.1.2.8})}
\item{Curso de Curta Duração em MSProject (
\href{run:Disciplinas/Apontamentos/MSProject2010.pdf}{Guião 3.1.2.9});}
\item{Curso de Curta Duração em SQL (
\href{run:Disciplinas/Apontamentos/ApontamentosAvaliacao/CursoSQLZamora/SQLPlus.pdf}{Guião 3.1.2.10});}
\end{enumerate}
\subsubsection{Participação na elaboração de conteúdos programáticos}A letra 'R' nas tabelas 
\ref{disciplinas1},
\ref{disciplinas2} e 
\ref{disciplinas3} corresponde a fichas de unidade curricular efetuadas.As disciplinas para quais foi feita uma ficha de unidade curricular diferente pela primeira vez (
\href{run:Disciplinas/FUCs.pdf}{Declaração 3.1.3}) são:
\begin{enumerate}
\item{Aplicações Informáticas (Licenciatura de Informática de Gestão - 95/06) (
\href{run:Disciplinas/Fichas/APLICINF9596IG.pdf}{FUC 3.1.3.1})}
\item{Aplicações Informáticas (CESE-Auditoria e Controlo de Gestão - 95/96) (
\href{run:Disciplinas/Fichas/AplicInf9596.pdf}{FUC 3.1.3.2}) }
\item{Linguagens da Programação II (Licenciatura de Informática de Gestão - 95/96) (
\href{run:Disciplinas/Fichas/LPII9596.pdf}{FUC 3.1.3.3}) }
\item{Paradigmas da Programação (Licenciatura de Informática de Gestão - 96/97) (
\href{run:Disciplinas/Fichas/PP9697.pdf}{FUC 3.1.3.4})}
\item{Complementos de Programação (CESE-Auditoria e Controlo de Gestão - 97/08) (
\href{run:Disciplinas/Fichas/CP9798.pdf}{FUC 3.1.3.5}) }
\item{Processamento de Linguagens (Licenciatura em Engenharia Informática - 98/99) (
\href{run:Disciplinas/Fichas/PL9899.pdf}{FUC 3.1.3.6})}
\item{Técnicas de Programação (Licenciaturas de Informática de Gestão e Engenharia Informática - 01/02) (
\href{run:Disciplinas/Fichas/TP0102.pdf}{FUC 3.1.3.7})}
\item{Programação I (Licenciatura de Informática de Gestão - 04/05) (
\href{run:Disciplinas/Fichas/ProgI0405.pdf}{FUC 3.1.3.8}) }
\item{Gestão de Projetos de Desenvolvimento de Software (Curso de Especialização em Tecnologias e Sistemas de Informação - 04/05) (
\href{run:Disciplinas/Fichas/GPDSWCETSI0405.pdf}{FUC 3.1.3.9})}
\item{Programação II (Licenciatura de Informática de Gestão - 05/06) (
\href{run:Disciplinas/Fichas/ProgII0506.pdf}{FUC 3.1.3.10}) }
\item{Programação (Licenciatura em Engenharia Eletrotécnica - 06/07) (
\href{run:Disciplinas/Fichas/ProgEE0607.pdf}{FUC 3.1.3.11})}
\item{Gestão de Projetos (Mestrado em Sistemas de Informação - 08/09) (
\href{run:Disciplinas/Fichas/GestaoProjetos.pdf}{FUC 3.1.3.12})}
\item{Paradigmas da Programação (Mestrado em Sistemas de Informação - 09/10) (
\href{run:Disciplinas/Fichas/PP0910.pdf}{FUC 3.1.3.13})}
\end{enumerate}
\subsubsection{Participação na elaboração de planos curriculares}
\begin{enumerate}
\item{Licenciatura em Engenharia Informática em 2004 (
\href{run:MissaoIPBoutros/desenhoCursos.pdf}{Comprovativo 3.1.4.1}).}
\item{Curso de Especialização Tecnológica em Sistemas de Informação (CETSI 2º ciclo) em 2004 (
\href{run:MissaoIPBoutros/desenhoCursos.pdf}{Comprovativo 3.1.4.2}).}
\item{Mestrado em Informática em 2008 (
\href{run:MissaoIPBoutros/desenhoCursos.pdf}{Comprovativo 3.1.4.3}).}
\item{Mestrado em Sistemas de Informação em 2009 (
\href{run:MissaoIPBoutros/desenhoCursos.pdf}{Comprovativo 3.1.4.4}).}
\item{Cursos de Especialização Tecnológica da ESTiG em 2007 e 2008 (
\href{run:MissaoIPBCargos/tarefasSub.pdf}{Comprovativo 3.1.4.5}).}
\item{Cursos Técnicos Superiores Profissionais da ESTiG em 2015 (
\href{run:MissaoIPBCargos/tarefasSub.pdf}{Comprovativo 3.1.4.6}).}
\end{enumerate}
\subsubsection{Publicação e disponibilização de materiais didáticos}O material pedagógico das disciplinas foi disponibilizado aos alunos numa primeira fase numa página pessoal (
\href{run:Virtual/usoPagPessoal.pdf}{Comprovativo 3.1.5.1}) e numa segunda fase, numa plataforma de intranet (
\href{run:Virtual/intranet.pdf}{Comprovativo 3.1.5.2}). A partir de 2009 passou a ser usada após a atual plataforma, IPB virtual (
\href{run:Virtual/usoIPBVirtual.pdf}{Comprovativo 3.1.5.3}).Todos os anos o material é actualizado e disponibilizado por estes meios.
\subsubsection{Inovação Pedagógica}Dado o carácter prático das disciplinas de programação tem sido, desde sempre, adotada uma metodologia de {
\em{ problem-based learning}} no sentido das matérias serem introduzidas através da realização de exercícios práticos promovendo em alturas oportunas o recurso a suporte teórico. A avaliação, seguindo o mesmo raciocínio, baseia-se numa metodologia de {
\em{ project-based learning}} onde alguns dos projetos propostos correspondem a solicitações externas à disciplina potenciando a interdisciplinariedade.No caso da disciplina de Processamento de Linguagens a metodologia adotada pela docente deu origem à publicação de um capítulo em livro e um artigo em conferência internacional:
\begin{itemize}
\item{Pereira M.J.V., Oliveira N., Da Cruz D., Henriques P.R., {
\bf{ An effective way to teaching Language Processing Courses}} para o livro {
\bf{ Innovative teaching strategies and New Learning Paradigms in Computer Programming}}, pp. 131-152, IGI Global, 2014(
\href{run:Publicacoes/PublicacoesSCOPUS.pdf}{SCOPUS})(
\href{run:Publicacoes/publicacoes/78.pdf}{pdf 2.2.1}).}
\item{Pereira M.J.V., Oliveira N., Cruz D., Henriques P.R., {
\bf{ Choosing Grammars to Support Language Processing Courses}}, SLATE 2013 - Symposium on Languages, Applications and Technologies, Faculdade de Ciências da Universidade do Porto, Junho de 2013, OpenAccess Series in Informatics (OASIcs), pp. 155-168, vol 29, 2013 (
\href{run:Publicacoes/PublicacoesSCOPUS.pdf}{SCOPUS},
\href{run:Publicacoes/ComprovativosDBLP.pdf}{DBLP}) (
\href{run:Publicacoes/publicacoes/71.pdf}{pdf 2.2.3.12})}
\end{itemize}No âmbito das disciplinas de programação alguns trabalhos foram desenvolvidos pela docente por forma a aumentar a motivação dos alunos e consequentemente o seu sucesso escolar. Esse esforço foi feito em disciplinas de programação (dadas por outros docentes) aplicando metodologias publicadas em:
\begin{itemize}
\item{Cruz D., Henriques P.R., Pereira M.J.V., {
\bf{ Constructing program animations using a pattern-based approach}}, ComSIS - Computer Science and Information Systems Journal, Special Issue on Advances in Programming Languages, Faculty of Technical Sciences, Novi Sad, Serbia, Volume 4, Number 2, pp. 99-116, Dec 2007. (ISSN: 1820-0214) (
\href{run:Publicacoes/ComprovativosDBLP.pdf}{DBLP}) (
\href{run:Publicacoes/publicacoes/25.pdf}{pdf 2.2.2.13})}
\item{Chuchulashvili M., Goziashvili N., Pereira M.J.V, Lopes R., {
\bf{ Micro atividades para a Aprendizagem de Programação}}, CMEA 2016 - VII Congresso Mundial de Estilos de Aprendizagem, Instituto Politécnico de Bragança, Julho 2016. (
\href{run:Publicacoes/publicacoes/91.pdf}{pdf 2.2.3.1})}
\item{Tavares P., Gomes E.F., Pereira M.J.V., Henriques P.R., {
\bf{ Técnicas para aumentar o Envolvimento dos Alunos na Aprendizagem da Programação}}, CMEA 2016 - VII Congresso Mundial de Estilos de Aprendizagem, Instituto Politécnico de Bragança, Julho 2016. (
\href{run:Publicacoes/publicacoes/89.pdf}{pdf 2.2.3.3})}
\end{itemize}No âmbito da disciplina de Métodos Numéricos dada por docentes do departamento de Matemática foram realizadas experiências baseadas no sistema publicado em:
\begin{itemize}
\item{Alves L., Balsa C., Pereira M.J.V., {
\bf{ Simulador Gráfico de Algoritmos Matemáticos}}, CMEA 2016 - VII Congresso Mundial de Estilos de Aprendizagem, Instituto Politécnico de Bragança, Julho 2016. (
\href{run:Publicacoes/publicacoes/90.pdf}{pdf 2.2.3.2})}
\item{Balsa C., Alves L., Pereira M.J.V., Rodrigues P.J. and Lopes R., {
\bf{ Graphical Simulation of Numerical Algorithms, An approach based on code instrumentation and java technologies}}, CSEDU 2012 - 4th International Conference on Computer Supported Education, Porto, pp. 164-169, Abril 2012. (
\href{run:Publicacoes/PublicacoesSCOPUS.pdf}{SCOPUS}) (
\href{run:Publicacoes/publicacoes/66.pdf}{pdf 2.2.3.17})}
\item{Balsa C., Alves L., Pereira M.J.V., Rodrigues P., {
\bf{ Graphical simulator of mathematical algorithms (GraSMA)}}, Proceedings of IASK International Conference Teaching and Learning, pp. 594-600, Seville, Spain, Dec 2010. (
\href{run:Publicacoes/publicacoes/84.pdf}{pdf 2.2.3.29})}
\end{itemize}No âmbito da disciplina de Paradigmas da Programação do Mestrado em Sistemas de Informação, lecionada pela docente nos últimos anos, foram feitas algumas experiências inovadoras baseadas em inquéritos que permitem distinguir linguagens de programação de domínio geral e linguagens de programação de domínio específico, conforme o artigo:
\begin{itemize}
\item{Kosar T., Oliveira N., Mernik M., Pereira M.J.V., Crepinsek M., Cruz D., Henriques P.R., {
\bf{ Comparing General-Purpose and Domain-Specific Languages: An Empirical Study}}, ComSIS - Computer Science and Information Systems Journal, Special Issue on Compilers, Related Technologies and Applications, Vol. 7, Number 2, pp. 247-264, April 2010. (ISSN: 1820-0214) (
\href{run:Publicacoes/ComprovativosISI.pdf}{ISI}, 
\href{run:Publicacoes/PublicacoesSCOPUS.pdf}{SCOPUS},
\href{run:Publicacoes/ComprovativosDBLP.pdf}{DBLP}) (Ci=35, AC=1) (Cs=67, AC=1) (
\href{run:Publicacoes/publicacoes/49.pdf}{pdf 2.2.2.8})}
\end{itemize}Foi também possível dar oportunidade aos alunos de experimentarem o uso de linguagens visuais usando o sistema publicado em:
\begin{itemize}
\item{Oliveira N., Pereira M.J.V., Henriques P.R., Cruz D., Cramer B., {
\bf{ VisualLISA: A Visual Environment to Develop Attribute Grammars}}, ComSIS - Computer Science and Information Systems Journal, Special Issue on Compilers, Related Technologies and Applications, Vol. 7, Number 2, pp. 265-290, April 2010. (ISSN: 1820-0214) (
\href{run:Publicacoes/ComprovativosISI.pdf}{ISI}, 
\href{run:Publicacoes/PublicacoesSCOPUS.pdf}{SCOPUS},
\href{run:Publicacoes/ComprovativosDBLP.pdf}{DBLP}) (Ci=3, AC=1) (Cs=10, AC=2) (
\href{run:Publicacoes/publicacoes/52.pdf}{pdf 2.2.2.9})}
\end{itemize}
\subsection{Participação em júris de provas}
\subsubsection{Arguente de Provas de Doutoramento}
\begin{enumerate}
\item{Arguente das Provas de Doutoramento de Leandro Oliveira Freitas da Universidade do Minho em Março de 2022 com o título "Uncertainty and Incompleteness Handling in Context-Aware Systems" e orientação do Prof. Pedro Henriques.}
\item{Arguente das Provas de Doutoramento de Daniel Rodríguez Cerezo da "Facultad de Informática de la Universidad Complutense de Madrid" em Fevereiro de 2016 (
\href{run:JuriProvas/TesisDaniel.pdf}{Comprovativo 3.2.1.1}).}
\item{Arguente das Provas de Doutoramento requeridas pelo Mestre Paulo Jorge Teixeira Matos na Universidade do Minho em Maio de 2005 (
\href{run:JuriProvas/JuriDoutoramentoPauloMatos.pdf}{Comprovativo 3.2.1.2}).}
\end{enumerate}
\subsubsection{Arguente de Provas de Mestrado}
\begin{enumerate}
\item{Arguência da tese de mestrado do aluno Lucas Guedes Barboza de dupla diplomação UTFPR-IPB (com o título Comparing Sentiment Analysis Tools on GitHub Project Discussions), Junho 2023.}
\item{Arguência da tese de mestrado do aluno Pedro Mimoso da Universidade do Minho (com o título Automatic Generation of ASTs from a Programming Language Grammar e orientador Pedro Henriques), Janeiro de 2023.(zoom 14:30 e 16 valores)}
\item{Arguência da tese de mestrado do aluno Carlos Barbosa da Universidade do Minho (com o título Meta Data Migrator e orientador José Carlos Ramalho), Abril 2022.}
\item{Arguente das Provas de Mestrado do aluno Tiago Santos (IPB) - "Ontologias para detecção de fraude em meio académico", dezembro de 2021.}
\item{Arguente das Provas de Mestrado do aluno Alexandre Dias (UM) - "ONTODL+ - An ontology description language and its compiler", Universidade do Minho, Setembro 2021.}
\item{Arguente das Provas de Mestrado do aluno Tarlon Gomes (UTFPR) - "Modelo de previsão do sucesso académico utilizando métodos de Learning Analytics", Mestrado em Informática, IPB, Fevereiro 2021.}
\item{Arguente das Provas de Mestrado do aluno Cassiano Andrade (UTFPR) - "“Web Application for the Mentoring Academy Program”", Mestrado em sistemas de Informação, Instituto Politécnico de Bragança, Fevereiro de 2020.}
\item{Arguente das Provas de Mestrado de Pedro Miguel Lopes Pereira, "Automatic fix of Source Code Vulnerabilities", aluno da Universidade do Minho, dezembro de 2019.}
\item{Arguente das Provas de Mestrado do Tiago Sanches Franco (UTFPR)- "Big Data Analytics para a classificação do risco de abandono escolar em cursos do ensino superior", Mestrado em sistemas de Informação, Instituto Politécnico de Bragança, Março de 2019.}
\item{Arguente das Provas de Mestrado de João Carlos Alves da Cruz, aluno da Universidade do Minho, dezembro de 2015 (
\href{run:JuriProvas/ArguenteJoaoCruz.pdf}{Comprovativo 3.2.2.1}).}
\item{Arguente das Provas de Mestrado de Pedro José Ribeiro Moreira, aluno da Universidade do Minho, dezembro de 2014 (
\href{run:JuriProvas/ArguentePedroMoreira.pdf}{Comprovativo 3.2.2.2}).}
\item{Arguente das Provas de Mestrado da aluna Sónia Pires Fernandes, do Mestrado em Sistemas de Informação 2009/2010, com o título "As TIC nas Juntas de Freguesia dos Conselhos de Bragança", 2010 (
\href{run:JuriProvas/ArguenteSoniaPires.pdf}{Comprovativo 3.2.2.3}).}
\end{enumerate}
\subsubsection{Membro do júri em Provas de Doutoramento}
\begin{enumerate}
\item{Membro do júri de avaliação de pré$-$tese da aluna de doutoramento da Universidade do Minho, Paula Tavares (docente do ISEP), reunido a 28 de Fevereiro de 2014 (
\href{run:JuriProvas/ArguentePaulaTavares.pdf}{Comprovativo 3.2.}3.1).}
\item{Membro do júri para reconhecimento de habilitações do Grau de Doutor de Mario Marcelo Beron, reunido a 25 de Maio de 2012 na Universidade do Minho (
\href{run:JuriProvas/RecMarioBeron.pdf}{Comprovativo 3.2.3.2}).}
\end{enumerate}
\subsubsection{Membro do júri em Provas de Mestrado}
\begin{enumerate}
\item{Presidente de 3 júris de provas de defesa de tese de Mestrado em Informática, Junho 2023. }
\item{Membro do júri das Provas de Mestrado na qualidade de orientadora do aluno Henrique Marcuzzo de dupla diplomação UTFPR-IPB, junho 2023.}
\item{Presidente do júri de 1 prova de defesa de Tese de Mestrado em Informática, IPB, dezembro 2022. (Aluno: Afonso Rocha ) }
\item{Presidente do júri de 1 prova de defesa de Tese de Mestrado em Informática, IPB, dezembro 2022. (Aluno: André Matos) }
\item{Presidente do júri de 1 prova de defesa de Tese de Mestrado em Informática, IPB, dezembro 2022. (Aluno: Seifeldien Sameh Soliman Mahmoud Soliman) }
\item{Presidente do júri de 1 prova de defesa de Tese de Mestrado em Informática, IPB, julho 2022. (Aluno: João Paulo Baptista Pereira)}
\item{Presidente do júri de 1 prova de defesa de Tese de Mestrado em Informática, IPB, dezembro 2021.}
\item{Presidente do júri de 2 provas de defesa de Teses de Mestrado em Contabilidade e Finanças, IPB, dezembro 2021.}
\item{Presidente do júri de 1 prova de defesa de Tese de Mestrado em Informática, IPB, junho 2021.}
\item{Presidente do júri de 3 provas de defesa de Teses de Mestrado em Sistemas de Informação, IPB, abril e julho de 2020.}
\item{Presidente do júri de 2 provas de defesa de Teses de Mestrado em Contabilidade e Finanças, IPB, dezembro 2019.}
\item{Presidente do júri de 3 provas de defesa de Teses de Mestrado em Sistemas de Informação, setembro 2019.}
\item{Membro do júri das Provas de Mestrado na qualidade de orientadora da aluna Joana Miguel da Universidade do Minho, dezembro 2019.}
\item{Membro do júri das Provas de Mestrado na qualidade de orientadora do aluno Martinho Aragão da Universidade do Minho, dezembro 2019.}
\item{Março de 2019 - Membro do júri de Reconhecimento de grau de mestre de Nilton Hideki Takagi do Programa de Pós-graduação em Informática da Universidade Federal de Paraíba; de Cristiane Maria Santos Ferreira do Programa de Pós-graduação em Computação da Universidade Federa Fluminense e de Nidhi Sharma do Master of Computer Applications da Rajasthan Technical University, India (este último foi indeferido).}
\item{Membro do júri das Provas de Mestrado  na qualidade de orientadora da aluna Gohar Tomeyan da Arménia, Julho de 2017.}
\item{Membro do júri das Provas de Mestrado na qualidade de orientadora do aluno Marcos Ramos da Universidade do Minho, maio 2017.}
\item{Membro do júri das Provas de Mestrado na qualidade de orientadora do aluno Daniel Novais da Universidade do Minho, dezembro 2016.}
\item{Membro do júri das Provas de Mestrado na qualidade de orientadora do aluno  da Universidade do Minho, dezembro 2014 }
\item{Presidente do júri da Tese de Mestrado da aluna Elena Gudorlava (Georgia) do Mestrado em Sistemas de Informação do IPB, julho 2016. (comprovativos em Investigação2016/TemasMSI)}
\item{Membro do júri das Provas de Mestrado na qualidade de orientadora da aluna Mariami Chuchulashvili do Mestrado em Sistemas de Informação do IPB, julho 2016.(comprovativos em Investigação2016/TemasMSI)}
\item{Membro do júri das Provas de Mestrado na qualidade de orientadora da aluna Nino Goziashvili do Mestrado em Sistemas de Informação do IPB, julho 2016.(comprovativos em Investigação2016/TemasMSI)}
\item{Presidente do júri da defesa de uma Tese do Mestrado em Engenharia da Construção, dezembro 2015 (
\href{run:JuriProvas/PresidenteMEC.pdf}{Comprovativo 3.2.4.1}).}
\item{Presidente do júri de 18 Provas de Mestrado no âmbito do Mestrado em Sistemas de Informação do IPB desde 2010 até hoje (
\href{run:JuriProvas/TeseMSI.pdf}{Comprovativo 3.2.4.2}).}
\item{Membro do júri das Provas de Mestrado na qualidade de orientadora do aluno Nuno Pereira da Universidade do Minho, dezembro 2014 (
\href{run:JuriProvas/juriNunoPereira.pdf}{Comprovativo 3.2.4.3}).}
\item{Membro do júri das Provas de Mestrado na qualidade de orientadora do aluno João Fonseca da Universidade do Minho, outubro 2014 (
\href{run:JuriProvas/juriJoaoFonseca.pdf}{Comprovativo 3.2.4.4}).}
\item{Membro do júri de Provas de Mestrado requeridas pela Licenciada Eva Oliveira na Universidade do Minho em Setembro de 2006 (
\href{run:JuriProvas/JuriMestradoEva.pdf}{Comprovativo 3.2.4.5}).}
\end{enumerate}
\subsection{Congressos e Conferências sobre docência}
\begin{itemize}
\item{Algumas palestras organizadas pelo mentoring academy do IPB.}
\item{Formação pedagógica “Testes online e integridade académica”, Junho 2020.}
\item{Formação pedagógica “Apresentação e análise dos resultados do inquérito de avaliação do modelo de ensino remoto do IPB”, Junho 2020.}
\item{Formação Pedagógica para docentes "Criação de apresentações visuais de alto impacto I", 2 horas, ESTiG-IPB, Junho 2019.}
\item{Membro da Comissão Organizadora do VII Congresso Mundial de Estilos de Aprendizagem:Educação, Tecnologias e Inovação, julho de 2016, Instituto Politécnico de Bragança (
\href{run:CongressoDocencia/cmea2016.pdf}{Comprovativo 3.3.1}).}
\item{Participação no XXI Encontro da Associação das Universidades de Língua Portuguesa (AULP), junho de 2011, Instituto Politécnico de Bragança (
\href{run:CongressoDocencia/AULP.pdf}{Comprovativo 3.3.2}).}
\end{itemize}
\subsection{Dedicação e qualidade das atividades docentes}
\subsubsection{Inquéritos}
\begin{itemize}
\item{Resultado dos inquéritos pedagógicos até 2009 (
\href{run:Inqueritos/inqueritosATE2009.pdf}{Comprovativo 3.4.1.1}).}
\item{Resultado dos inquéritos pedagógicos 2010 (
\href{run:Inqueritos/relatorioPedagogico2010.pdf}{Comprovativo 3.4.1.2}).}
\item{Resultado dos inquéritos pedagógicos 2011 (
\href{run:Inqueritos/relatorioPedagogico2011.pdf}{Comprovativo3.4.1.3}).}
\item{Resultado dos inquéritos pedagógicos 2012 (
\href{run:Inqueritos/relatorioPedagogico2012.pdf}{Comprovativo3.4.1.4}).}
\item{Resultado dos inquéritos pedagógicos 2013 (
\href{run:Inqueritos/relatorioPedagogico2013.pdf}{Comprovativo3.4.1.5}).}
\item{Resultado dos inquéritos pedagógicos 2014 (
\href{run:Inqueritos/relatorioPedagogico2014.pdf}{Comprovativo3.4.1.6}).}
\end{itemize}
\subsubsection{Utilização de ferramentas de elearning}A plataforma IPB virtual (
\url{https://virtual.ipb.pt}) é baseada em áreas às quais está associada uma lista de participantes o que permite ser usada como ferramenta de elearning. No caso das áreas correspondentes a disciplinas, o professor e os alunos constituem o grupo de participantes dentro do qual é feita a partilha de informação. Em cada uma dessas áreas é possível fazer o upload de recursos, o envio e receção de mensagens entre participantes, o depósito de documentos no cacifo digital por parte dos alunos (entrega de trabalhos) entre outras facilidades (
\href{run:Virtual/mensagensvirtual.pdf}{Comprovativo 3.4.2.1}). \\
Os recursos que normalmente são disponibilizados consistem em conteúdos teóricos, fichas de trabalho, enunciados de trabalhos práticos, diapositivos e outras ligações importantes. A comunicação entre a docente e os alunos em contexto fora da sala de aula foi sempre assegurada por email pessoal mas também por mensagens e avisos nesta plataforma de elearning.A plataforma dá também acesso à plataforma de sumários o que permite aos alunos terem acesso ao plano da aula em antecipação. As fichas de disciplina são atualizadas e depositadas no Guia ECTS (
\url{https://apps.ipb.pt/guia$-$ects}) no início de cada ano letivo. Os horários de atendimentos são disponibilizados na página da ESTiG no início de cada semestre. \\
A disponibilização de toda esta informação e a facilidade de comunicação entre participantes permite que alunos que não têm acesso a aulas presenciais possam acompanhar a disciplina à distância.Mais se acrescenta que estas plataformas são usadas pela docente desde a sua implementação (comprovativos na secção 3.1.5), o mesmo acontece com a plataforma de pautas eletrónicas desde o ano letivo 2008/2009 e a plataforma de sumários desde 2009/2010  (
\href{run:Virtual/pautassumarios.pdf}{Declaração 3.4.2.2}).
\subsubsection{Internacionalização da atividade pedagógica}
\begin{itemize}
\item{Lecionação de módulos de SQL (3,5 horas) ao Curso de Verão em Bases de Dados para a Web na Fundação Rei Afonso Henriques, em Zamora, em Julho durante os anos de 2007, 2008, 2009 (
\href{run:MissaoIPBoutros/AfonsoHenriques.pdf}{Comprovativo 3.4.3.1}).}
\item{Lecionação de um módulo de SQL (3 horas) ao Curso de Verão em Bases de Dados para a Web na Fundação Rei Afonso Henriques, em Bragança, Julho de 2010 (
\href{run:MissaoIPBoutros/AfonsoHenriques2.pdf}{Comprovativo 3.4.3.2}).}
\item{Orientação de estágios de verão no âmbito de um protocolo entre o Instituto Politécnico de Bragança e o B.K. Birla Institute of Engineering and Technology (Pilani, Índia) no ano letivo 2010/2011 (
\href{run:CoOrientTrabalhos/AlunoIndiano.pdf}{Comprovativo 3.4.3.3}).}
\item{Orientação de estágios de verão no âmbito de um protocolo entre o Instituto Politécnico de Bragança e a École Nationale Supérieur d'Electrotechnique, d'Eletronique, d'Informatique,d'Hydraulique et de Télècommunications (ENSEEIHT) de l'Institut Nationale Polytechnique de Toulouse (ENSEEIHT-INPT) no ano letivo 2014/2015 (
\href{run:CoOrientTrabalhos/AlunosFrancesesBalsa.pdf}{Comprovativo 3.4.3.4}).}
\item{Participação no Projeto Erasmus+ International Credit Mobility (ICM) concretizada pela participação em reuniões com as Instituições Parceiras incluindo visitas a duas Universidades Georgianas para divulgação do projeto e cooperação na área da Engenharia Informática e do Mestrado em Sistemas de Informação, nomeadamente na coorientação de duas teses de mestrado em Sistemas de Informação, abril 2016 (
\href{run:MissaoIPBoutros/ICM.pdf}{Comprovativo 3.4.3.5}).}
\item{Participação no Projeto Erasmus+ International Credit Mobility (ICM) concretizada pela participação em reuniões com as Instituições Parceiras incluindo visitas às várias faculdades da Universidade Nacional da Arménia para divulgação do projeto e cooperação na área da Engenharia Informática e do Mestrado em Sistemas de Informação, nomeadamente na coorientação de uma tese de mestrado em Sistemas de Informação, maio 2017.}
\item{Visita aos campos da UTFPR de Curitiba e de Ponta Grossa para promoção de investigação conjunta no âmbito do CEDRI e estabelecimento de novos protocolos de dupla diplomação em especial para os cursos de Engenharia Informática e Mestrado em Sistemas de Informação, Junho 2018.}
\item{Participação no Projeto Erasmus+ International Credit Mobility (ICM) concretizada pela participação em reuniões com as Instituições Parceiras incluindo visitas à Universidade de Biskek, Quirguistão para divulgação do projeto e cooperação na área da Engenharia Informática e do Mestrado em Sistemas de Informação, Março 2019.}
\end{itemize}
\subsection{Orientação de dissertações e trabalhos conducentes a grau académico}
\subsubsection{Teses de Mestrado}
\begin{enumerate}
\item{Coorientação da Tese de Mestrado de Eva Oliveira, com o título 
\emph{Características de um Sistema de Visualização para Compreensão de Programas Web}, no Departamento de Informática da Universidade do Minho, defendido em Setembro de 2006 (
\href{run:CoOrientTrabalhos/OrientacaoEva.pdf}{Comprovativo 3.5.1.1}).}
\item{Coorientação da Tese de Mestrado de Nuno Oliveira, com o título 
\emph{Improving Program Comprehension Tools for DSLs} no Departamento de Informática da Universidade do Minho, defendido em Dezembro de 2009 (
\href{run:CoOrientTrabalhos/NunoOliveira.pdf}{Comprovativo 3.5.1.2}).}
\item{Coorientação da Tese de Mestrado da aluna Albertina Neto do Mestrado em Sistemas de Informação do IPB, com o título 
\emph{O uso das TIC nas escolas do 1
\º ciclo do Ensino Básico do Distrito de Bragança}, defendido em Dezembro de 2010 (
\href{run:CoOrientTrabalhos/TesesMSI.pdf}{Comprovativo 3.5.1.3}).}
\item{Coorientação da Tese de Mestrado da aluna Ermelinda Afonso Gonçalves do Mestrado em Sistemas de Informação do IPB, com o título 
\emph{Utilização de Ferramentas Web pelos professores do Ensino Secundário para Acompanhamento Escolar dos Alunos em Contexto Fora da Sala de Aula}, defendido em Dezembro de 2013 (
\href{run:CoOrientTrabalhos/TesesMSI.pdf}{Comprovativo 3.5.1.4}).}
\item{Coorientação da Tese de Mestrado do aluno Nuno Pereira do Departamento de Informática da Universidade do Minho, com o título {
\em{ Concept location over the system dependency graph}}, defendida em Dezembro de 2015 (
\href{run:CoOrientTrabalhos/OrientNunoPereira.pdf}{Comprovativo 3.5.1.5}).}
\item{Coorientação da Tese de Mestrado do aluno João Fonseca do Departamento de Informática da Universidade do Minho, com o título {
\em{ Criação de DSL com base em ontologias}}, defendida em Outubro de 2015(
\href{run:CoOrientTrabalhos/OrientJoaoFonseca.pdf}{Comprovativo 3.5.1.6}).}
\item{Coorientação da Tese de Mestrado da aluna Mariami Tchutchulashvili da Georgia, no âmbito do Projeto ICM (Erasmus+) do Instituto Politécnico de Bragança, com o tema {
\em{ Design and implementation of an online pedagogical methodology for Java programmer student beginners}}, defesa prevista para Julho de 2016 (
\href{run:CoOrientTrabalhos/emdesenvolvimento/Mariami.pdf}{Comprovativo 3.5.1.7}).}
\item{Coorientação da Tese de Mestrado da aluna Nino Godziashvili da Georgia, no âmbito do Projeto ICM (Erasmus+) do Instituto Politécnico de Bragança, com o tema {
\em{ Design and implementation of an online pedagogical methodology for C programmer student beginners}}, Julho de 2016 (
\href{run:CoOrientTrabalhos/emdesenvolvimento/Nino.pdf}{Comprovativo 3.5.1.8}).}
\item{Coorientação da Tese de Mestrado do aluno Marcos Ramos do Departamento de Informática da Universidade do Minho, com tema, {
\em{ Querying and Answering Systems for Programming Languages}}, maio de 2017 (
\href{run:CoOrientTrabalhos/emdesenvolvimento/MarcosRamos.pdf}{Comprovativo 3.5.1.9}).}
\item{Coorientação da Tese de Mestrado do aluno Daniel Novais do Departamento de Informática da Universidade do Minho, com tema, {
\em{ Programmer Profiling through Code Analysis}},dezembro de 2016 (
\href{run:CoOrientTrabalhos/emdesenvolvimento/DanielNovais.pdf}{Comprovativo 3.5.1.10}).}
\item{Orientação da Tese de Mestrado da aluna Gohar Tomeyan da Arménia, no âmbito do Projeto ICM (Erasmus+) do Instituto Politécnico de Bragança, com o tema {
\em{ A text uniqueness checking system for Armenian Language}}, Julho de 2017. }
\item{Coorientação da Tese de Mestrado do aluno Johnny Lima (UTFPR), no âmbito do projeto de dupla diplomação do Mestrado em Sistemas de Informação, com o título {
\em{ Modelo para Classificação do Risco de Abandono Escolar em Cursos de Engenharia com Base em Métodos de Academic Analytics}}, Junho 2018.}
\item{Coorientação da Tese de Mestrado do aluno Martinho Aragão do Departamento de Informática da Universidade do Minho, com tema, {
\em{ Intelligent feedback system for programmer’s profile improvement}},dezembro de 2019.}
\item{Coorientação da Tese de Mestrado da aluna Joana Margarida Miguel do Departamento de Informática da Universidade do Minho, com tema, {
\em{ Privas: assuring the privacy in database exploring systems}},dezembro de 2019.}
\item{Coorientação da Tese de Mestrado do aluno João Gris (UTFPR), no âmbito do projeto de dupla diplomação do Mestrado em Sistemas de Informação, com o título {
\em{ Course Direction Support Information System}}, dezembro 2019.}
\item{Coorientação da Tese de Mestrado do aluno João Reis do Departamento de Informática da Universidade do Minho, com tema, {
\em{ GSD: A Web Application for Teacher Timetable Management}}, Março de 2020.}
\item{Coorientação da Tese de Mestrado da aluna Marcela Almeida (CEFET-Minas Gerais), no âmbito do projeto de dupla diplomação do Mestrado em Sistemas de Informação, com o título {
\em{ EasyCoding - Metodologia para apoiar o aprendizado de programação}}, Junho 2020.}
\item{Orientação do Caio Nakai (IPB), com o título {
\em{ Digitalização de Processos de Gestão de Espaços}}, defendida em julho 2020.}
\item{Coorientação do Diogo Soares (UM), com o título {
\em{ Python-Tutor on Program Comprehension}}, defendida em dezembro de 2020.}
\item{Coorientação do Manuel Sousa (UM), com o título {
\em{ Applying Attribute Grammars to teach Linguistic Rules}}, defendida em julho de 2021.}
\item{Orientação do Benarous Farouq (IPB), com o título {
\em{ Almond Variety Detection using Deep Learning}}, defendida em dezembro 2020.}
\item{Orientação da Rafaela Pinho (UM), com o título {
\em{ Biometric Analysis of Behaviours in Serious Games}}, defendida em Abril de 2022.}
\item{Orientação da tese de mestrado do Mestrado em Informática do aluno Beka Kokhodze da Georgia com o titulo “Marketplace for Circular Bioeconomy”, defendida em Julho de 2022.}
\item{Coorientação da tese de mestrado da aluna Mariana de Oliveira Pereira  da Universidade do Minho, defendida a 3 de Março de 2023, com o título Avaliação Automática de Testes de Atenção e de Acuidade Visual (protótipo da mariana 
\url{https://daisy.epl.di.uminho.pt/}).}
\item{Orientação da tese de mestrado do Henrique Marcuzzo (UTFPR - IPB), com o título {
\em{ Implementation of a Thermal-based Food Recommendation System}}, Junho 2023 (18 valores). }
\item{Orientação do Marco Silva (IPB) (em curso - desenvolvimento de uma palicação web de auto-avaliação de sustentabilidade ambiental para empresas vinícolas).}
\item{Orientação do Paulo Pereira (IPB) (em curso).}
\item{Orientação do Daniel Farina (UTFPR-IPB) (em curso).}
\item{Orientação do Guilherme Tonello (UTFPR-IPB) (em curso).}
\item{Orientação do Eduardo Silva (UM).}
\item{Orientação do Gustavo Lourença (UM).}
\item{Orientação do Tiago Freitas (UM).}
\item{Orientação do Martim Bento (UM).}
\end{enumerate}
\subsubsection{Projetos de Fim de Curso}
\begin{enumerate}
\item{Orientação de um estágio da Licenciatura em Engenharia Informática, na empresa IT Sector em 2023 (Ricardo Vieira).}
\item{Projeto de fim de curso de Engenharia Informática da Universidade do Minho, João Guilherme Rodrigues Reis, João Carlos Viana Pereira Marques, João Domingos Pereira Barbosa, Nuno Alexandre Pereira Machado, Desenvolvimento de uma aplicação para gestão de viaturas oficiais de uma instituição do ensino superior.(uniauto.di.uminho.pt/ endereço:gestor@mail.com e password: gestor)}
\item{Coorientadora do projeto de fim de curso com o título "Compreensão de programas em Visual-Basic" do aluno Moisés Ramires da Licenciatura em Ciências da Computação da Universidade do Minho, no ano letivo de 2018/2019.}
\item{Coorientadora do projeto de fim de curso com o título "Animador de Algoritmos Específicos"dos alunos José Simões e Pedro Fernandes da Licenciatura em Ciências da Computação da Universidade do Minho, no ano letivo de 2015/2016.}
\item{Coorientadora do projeto de fim de curso com o título "Animador de Programas em C (semelhante ao Jeliot)"dos alunos António Oliveira e Pedro Lemos da Licenciatura em Ciências da Computação da Universidade do Minho, no ano letivo de 2015/2016.}
\item{Orientadora de projeto de fim de curso de Engenharia Informática tendo por objetivo o desenvolvimento de um "Sistema cliente/servidor em Java para Avaliação Automática de Alunos", Lúcia Torres Teixeira e Gina Rodrigues, durante o ano letivo 2004/2005 (
\href{run:CoOrientTrabalhos/projFimCursoEI.pdf}{Comprovativo 3.5.2.1}).}
\item{Orientadora de projeto de fim de curso de Tecnologias e Sistemas de Informação cujo objetivo foi o desenvolvimento de um "Sistema de Interfaces para um Simulador de Processos de Polimerização", Márcia Sousa e Elisabete Freitas durante o ano letivo 2005/2006 (
\href{run:CoOrientTrabalhos/projFimCursoTSI.pdf}{Comprovativo 3.5.2.2}).}
\item{Orientadora de projeto de fim de curso de Engenharia Informática cujo objetivo foi a "Construção e teste de uso de um animador de programas no ensino da programação em C", Pedro Franco, durante o ano letivo 2006/2007 (
\href{run:CoOrientTrabalhos/projFimCursoEI.pdf}{Comprovativo 3.5.2.3}).}
\item{Orientadora de projeto de fim de curso de Engenharia Informática cujo objetivo foi a construção de um "Jogo de Matemática Interactivo (Geometria)", Nuno Afonso, durante o ano letivo 2006/2007 (
\href{run:CoOrientTrabalhos/projFimCursoEI.pdf}{Comprovativo 3.5.2.4}).}
\item{Orientadora de projeto de fim de curso de Engenharia Informática cujo objetivo foi a construção de um "Jogo de Matemática Interactivo (Cálculo de Operações para os anos 1 e 2)", Tiago Martins, durante o ano letivo 2006/2007 (
\href{run:CoOrientTrabalhos/projFimCursoEI.pdf}{Comprovativo 3.5.2.5}).}
\item{Orientadora de projeto de fim de curso de Engenharia Informática cujo objetivo foi a construção de um "Jogo de Matemática Interactivo (Cálculo de Operações para os anos 3 e 4)", Ana Silva e Francisco Gonçalves, durante o ano letivo 2007/2008 (
\href{run:CoOrientTrabalhos/projFimCursoEI.pdf}{Comprovativo 3.5.2.6}).}
\item{Orientadora de projeto de fim de curso de Engenharia Informática cujo objetivo foi a construção de um "Portal Web - Juventude S.Pedro,  Liliana Fernandes e Ricardo Cruz, durante o ano letivo 2007/2008 (
\href{run:CoOrientTrabalhos/projFimCursoEI.pdf}{Comprovativo 3.5.2.7}).}
\item{Orientadora de projeto de fim de curso de Engenharia Informática cujo objetivo foi a implementação de uma aplicação de "Balanced Scored Card", Teresa Vaz, durante o ano letivo 2008/2009 (
\href{run:CoOrientTrabalhos/projFimCursoEI.pdf}{Comprovativo 3.5.2.8}).}
\end{enumerate}
\subsubsection{Orientação de Estagiários e Bolseiros}
\begin{enumerate}
\item{Orientação do bolseiro Gustavo Quieregato do projeto Aquae Vitae.}
\item{Orientadora responsável pela estância do Iván Arias Rodriguez da Universidade Complutense de Madrid no âmbito da sua tese de doutoramento sobre redes neuronais.}
\item{Coorientadora de um estágio de uma aluna da Licenciatura em Matemática e Ciências da Computação da Universidade do Minho, com o título {
\em{ LISS, A linguagem e o ambiente de programação}}, Daniela Cruz, durante o ano letivo 2006/2007  (
\href{run:Projectos/declaracoesUM.pdf}{Comprovativo 3.5.3.1}).}
\item{Coorientadora de um estágio de um aluno da Licenciatura em Engenharia de Sistemas e Informática da Universidade do Minho, com o título {
\em{ Implementação do WebAppViewer: uma ferramenta para compreender aplicações Web}}, Ruben Fonseca, durante o ano letivo 2006/2007 (
\href{run:Projectos/declaracoesUM.pdf}{Comprovativo 3.5.3.2}).}
\item{Coorientadora de um projecto do Mestrado de Informática da UM para desenvolvimento do sistema {
\em{ VisualLISA}} do aluno Nuno Oliveira, durante o ano letivo 2008/2009 (
\href{run:Projectos/declaracoesUM.pdf}{Comprovativo 3.5.3.3}).}
\item{Coorientadora de um bolseiro de iniciação à atividade científica (Pedro Faria) da Universidade do Minho cujo trabalho se intitula {
\em{ Visual Languages: Comparative study of generators}}, durante o ano letivo 2009/2010 (
\href{run:Projectos/declaracoesUM.pdf}{Comprovativo 3.5.3.4}).}
\item{Orientação de um estágio a decorrer na secretaria de alunos da ESTiG de uma aluna do Curso de Especialização Tecnológica de Secretariado e Assessoria Administrativa em 2013 (
\href{run:CoOrientTrabalhos/estagiosESE.pdf}{Comprovativo 3.5.3.5}).}
\item{Orientação de dois estágios (um a decorrer na secretaria de alunos da ESTiG e outro na Gabinete de Relações com o Exterior da ESTiG) de duas alunas do Curso de Especialização Tecnológica de Secretariado e Assessoria Administrativa em 2014 (
\href{run:CoOrientTrabalhos/estagiosESE.pdf}{Comprovativo 3.5.3.6}).}
\item{Orientação de um estágio a decorrer no Gabinete de Relações com o Exterior da ESTiG de uma aluna do Curso de Especialização Tecnológica de Secretariado e Assessoria Administrativa em 2015 (
\href{run:CoOrientTrabalhos/estagiosESE.pdf}{Comprovativo 3.5.}3.7).}
\end{enumerate}
\end{document}