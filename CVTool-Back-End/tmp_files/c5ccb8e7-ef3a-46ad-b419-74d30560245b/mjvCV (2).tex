
\documentclass[11pt]{article}
\usepackage{portuges}
\usepackage[pdftex]{hyperref}
\newcommand{
\daterange}[1]{#1}
\hypersetup{pdfnewwindow=true}
\usepackage[latin1]{inputenc}
\title{CURRICULUM VITAE  \\

\vspace{3cm}
\vspace{3cm}}
\author{Maria João Tinoco Varanda Pereira}
\date{
\today} 
\begin{document} 
\maketitle 
\newpage
\section{Outras atividades consideradas relevantes para a missão do IPB}
\subsection{Exercícios de cargos e funções académicas}
\subsubsection{Participação em órgãos colegiais}
\begin{itemize}
\item{Membro do primeiro CTC (Conselho Técnico-científico) da ESTiG desde Julho de 2009 até hoje (
\href{run:MissaoIPBCargos/ctc.pdf}{Comprovativo 4.1.1.1}). }
\item{Vice-presidente do Conselho Diretivo da Escola Superior de Tecnologia e Gestão de Bragança, desde Março de 2007 até Junho de 2009 (
\href{run:MissaoIPBCargos/subdiretora.pdf}{Comprovativo 4.1.1.2}).}
\item{Coordenadora do Departamento de Informática e Comunicações da ESTiG, desde Setembro de 2003 até Setembro de 2005 (
\href{run:MissaoIPBCargos/coordDIC.pdf}{Comprovativo 4.1.1.3}).}
\item{Membro da Comissão Científica do Mestrado em Sistemas de Informação desde a primeira edição (novembro de 2008) até hoje (
\href{run:MissaoIPBCargos/msi.pdf}{Comprovativo 4.1.1.4}).}
\item{Membro do Colégio Eleitoral para a eleição do Presidente do Instituto Politécnico de Bragança, desde Novembro de 2005 a Fevereiro de 2006 (
\href{run:MissaoIPBCargos/colegioeleitoral.pdf}{Comprovativo 4.1.1.5}).}
\item{Membro do Conselho Pedagógico da ESTiG de dezembro 1998 até hoje (
\href{run:MissaoIPBCargos/CP.pdf}{Comprovativo 4.1.1.6}).  }
\item{Membro da Comissão de Curso do CTESP em Desenvolvimento de Software.}
\item{Vogal da Comissão Executiva do Conselho Pedagógico da ESTiG desde dezembro de 2002 até novembro de 2004 (
\href{run:MissaoIPBCargos/CP.pdf}{Comprovativo 4.1.1.7}).}
\item{Membro e Vice-presidente da Assembleia de Representantes da ESTiG, desde Julho de 2005 até Maio de 2009 (
\href{run:MissaoIPBCargos/AssembleiaRepresentantes.pdf}{Comprovativo 4.1.1.8}).}
\item{Comissão Diretiva do CETSI (Curso de Especialização em Tecnologias e Sistemas de Informação) durante o ano letivo 2004-2005 (
\href{run:MissaoIPBCargos/avaliacaoReestruturacao.pdf}{Comprovativo 4.1.1.9}).}
\item{Membro da Comissão de Curso de Engenharia Informática da ESTiG desde dezembro de 2002 até dezembro de 2006 (
\href{run:MissaoIPBCargos/CP.pdf}{Comprovativo 4.1.1.10}).}
\item{Representante da área de informática no Comissão de Curso de Engenharia Eletrotécnica da ESTiG entre dezembro de 1998 e dezembro de 2002 (
\href{run:MissaoIPBCargos/CP.pdf}{Comprovativo 4.1.1.11}).}
\item{Representante da área de Informática no Conselho de Curso de Auditoria e Controlo de Gestão da ESTiG (96/97 e 97/98) (
\href{run:MissaoIPBCargos/InfACG.pdf}{Comprovativo 4.1.1.12}).}
\item{Membro da Comissão de Autoavaliação da Licenciatura Bietápica em Engenharia Informática desde abril de 2004 a dezembro de 2004 (
\href{run:MissaoIPBCargos/CALEI.pdf}{Comprovativo 4.1.1.13}).}
\item{Membro por inerência do Conselho Científico da ESTiG de junho de 1999 até julho de 2009 (
\href{run:MissaoIPBCargos/ConselhoCientifico.pdf}{Comprovativo 4.1.1.14}). }
\end{itemize}
\subsubsection{Outros cargos e funções por designação}
\begin{itemize}
\item{Subdiretora (por nomeação) da ESTiG desde 2009 até novembro de 2022 (
\href{run:MissaoIPBCargos/subdiretora.pdf}{Comprovativo 4.1.2.1}).}
\item{Membro da Comissão de Horários durante o ano letivo 1996/1997 (
\href{run:MissaoIPBoutros/comissaoHorarios.pdf}{Comprovativo 4.1.2.2}).}
\item{Membro do júri de provas de maiores de 23 em 2006 (
\href{run:JuriProvas/Maiores23.pdf}{Comprovativo 4.1.2.3}).}
\item{Membro do júri de creditação de experiência profissional em 2008 (
\href{run:JuriProvas/JuriExpProf.pdf}{Comprovativo 4.1.2.4}).}
\item{Membro do júri de atribuição de uma bolsa para Técnico de Investigação na área de Informática em 2010 (
\href{run:JuriProvas/BolsaCIMO.pdf}{Comprovativo 4.1.2.5}).
\end}{
\itemdocument{}Presidente do júri de atribuição de uma bolsa de Gestão de Ciência e Tecnologia na área de Informática de Gestão em 2012 (
\href{run:JuriProvas/BolsaIG.pdf}{Comprovativo 4.1.2.6}).}
\item{Presidente do júri de atribuição de uma bolsa de Gestão de Ciência e Tecnologia na área de Engenharia Mecânica em 2015 (
\href{run:JuriProvas/BolsaFabLab.pdf}{Comprovativo 4.1.2.7}).}
\item{Presidente do Júri do processo concursal para dirigente intermédio do 3º grau para os Serviços Académicos do IPB, julho de 2017.}
\item{Presidente do Júri de quatro concursos de recrutamento de professores coordenadores para as áreas de Informática, Engenharia Química, Engenharia Civil e Engenharia Eletrotécnica e Computadores.}
\item{Presidente da Comissão Eleitoral para eleição do primeiro coordenador do Centro de Investigação em Digitalização e Robótica Inteligente (CeDRI).}
\item{Membro do júri para contratação de um bolseiro de doutoramento (FCT) para o CEDRI, 2020.}
\item{Membro do júri para recrutamento de Prof. Adjunto do IPCA, 2021.}
\item{Presidente do Júri do processo concursal para dirigente intermédio do 3º grau para os Serviços Académicos do IPB, Janeiro 2022.}
\item{Membro do júri para contratação de um bolseiro de doutoramento (FCT) para o CEDRI, 2022.}
\item{Membro do júri de um concurso internacional para recrutamento de Prof. Adjunto para a Escola Superior de Tecnologia e Gestão de Águeda (ESTGA) da Universidade de Aveiro, 2023.}
\item{Presidente do júri do concurso para recrutamento de quatro professores coordenadores para a área de Engenharia Civil da ESTiG, IPB, 2023 (a decorrer).}
\item{Membro do júri do concurso para Prof. Adjunto do DIC, 2023 (a decorrer).}
\item{Membro do júri do concurso para recrutamento de professor adjunto para o ISEP, 2023 (a decorrer).}
\item{Membro do júri do concurso para recrutamento de professor adjunto para o IPVC, 2023 (a decorrer).}
\item{Membro de júris de contratação de bolseiros para os projetos Aquae Vitae e Bacchustech.}
\end{itemize}
\subsection{Funções como membro da Direção da ESTiG}Como membro da Direção da ESTiG foram produzidos relatórios de atividades (
\href{run:MissaoIPBCargos/RelActividadesMJ.pdf}{Comprovativo 4.2.1}) mas as principais tarefas realizadas todos os anos desde 2007 estão descritas abaixo (
\href{run:MissaoIPBCargos/tarefasSub.pdf}{Declaração 4.2.2}) :
\begin{itemize}
\item{Responsável pela Gestão Pedagógica de todos os Cursos da ESTiG desde 2007.}
\item{Responsável pela Gestão Pedagógica do CET em Contabilidade e Gestão, deslocalizado em Mogadouro, em 2010/2011.}
\item{Responsável pela Gestão Pedagógica do CET em Condução de Obra, deslocalizado em Mogadouro, em 2011/2012 e 2012/201}
\item{Responsável pela Gestão Pedagógica do CET em Contabilidade e Gestão, deslocalizado em Penafiel, em 2012/2013, 2013/2014 e 2014/2015.}
\item{Responsável pela Gestão Pedagógica do CET em Energias Renováveis, deslocalizado em Amarante, em 2012/2013.}
\item{Responsável pela Gestão Pedagógica do CET em Energias Renováveis, deslocalizado em Chaves, em 2013/2014 e 2014/2015.}
\item{Responsável pela Gestão Pedagógica do CET em Contabilidade e Gestão, deslocalizado em Chaves, em 2014/2015.}
\item{Responsável pela Gestão Pedagógica do CTESP em Energias Renováveis e Instalações Elétricas, deslocalizado em Chaves, em 2015/2016.}
\item{Responsável pela Gestão Pedagógica do CTESP em Prospeção Mineral e Geotécnica, deslocalizado em Torre de Moncorvo, em 2015/2016.}
\item{Responsável pela seriação dos alunos candidatos aos Cursos de Especialização Tecnológica (CET), desde 2007 até 2014.}
\item{Responsável pela seriação dos alunos candidatos aos Cursos Técnicos Superiores Profissionais (CTESP), desde 2015. }
\item{Responsável pela Elaboração dos Horários Escolares, desde 2007. }
\item{Responsável pelo funcionamento da Secretaria de Alunos da ESTiG, desde 2007. }
\item{Membro da Comissão de Avaliação do Desempenho do Pessoal não docente da ESTiG, desde 2007. }
\item{Responsável pela validação das férias dos docentes da ESTiG, desde 2014. }
\item{Membro das Comissões de Autoavaliação dos cursos da ESTiG em avaliação, desde 2007. }
\item{Responsável pela implementação de estratégias de combate ao insucesso escolar nos cursos da ESTiG, em especial dos CTESPs, desde 2015. }
\item{Responsável pelas propostas dos Novos Cursos Técnicos Superiores Profissionais, cujos dossiers que acompanham os pedidos de registo foram enviados para a DGES durante o ano de 2015. }
\item{Participação na elaboração de propostas de Cursos de Especialização Tecnológica da ESTiG em 2007 e 2008.}
\item{Responsável pela organização das reuniões das Comissões Externas de Avaliação (CAE), aquando das avaliações dos cursos da ESTiG, desde 2011. }
\item{Membro de 21 júris de seleção de Pessoal Docente Especialmente Contratado (PDEC), desde 2010. }
\item{Participação no Desenho do curso Engenharia Informática para a Universidade Politécnica Fernando Marcelino no Huambo-Angola em 2007.}
\item{Responsável pela candidatura dos cursos de Engenharias da ESTiG para inclusão no índice FEANI, no ano 2012/2013. }
\item{Participação na organização da STG (Semana da Tecnologia e Gestão), desde 2013. }
\item{Responsável pelo acolhimento de alunos internacionais e Erasmus da ESTiG no que diz respeito a facilitar contacto direto com os professores, horário escolar e na identificação de disciplinas a lecionar em língua inglesa, desde 2011.}
\item{Vogal do Júri do Concurso para Técnico de 2ª Classe estagiário, área Administrativa, conforme Edital nº 618/2007, publicado no Diário da República, 2ª Série, nº 145, em Julho de 2007 (
\href{run:MissaoIPBoutros/Tecnico2Classe.pdf}{Comprovativo 4.2.3}).}
\item{Presidente do Júri do Concurso Interno de Acesso Limitado para o preenchimento de dois lugares de Operário Altamente Qualificado Principal, conforme aviso nº 8/2007 (
\href{run:MissaoIPBoutros/Operario.pdf}{Comprovativo 4.2.4}).}
\item{Presidente do Júri do Concurso Interno de Acesso Limitado para um lugar de Técnico Principal da Carreira Técnica do IPB, conforme o aviso nº 11/2008 (
\href{run:MissaoIPBoutros/TecnicoPrincipal.pdf}{Comprovativo 4.2.5}).}
\item{Presidente do Júri do Concurso Interno de Acesso Limitado para dois lugares de Assistente Administrativo Especialista da Carreira Administrativa do IPB, conforme o aviso nº 14/2008 (
\href{run:MissaoIPBoutros/AssAdmin.pdf}{Comprovativo 4.2.6}).}
\item{Responsável pela comunicação de novos planos de estudos à DGES na sequênica da avaliação da A3ES}
\item{Responsável pela gestão das candidaturas de CTESP, Mestrados e Internacionais.}
\item{Responsável pela interação com os serviços académicos}
\item{Membro da equipa de elaboração de inquéritos online aos docentes e alunos na pandemia COVID}
\item{Responsável pela tabela de creditações CTESP -Licenciaturas}
\item{Responsável pela envio da reformulação de CTESPs, Licenciaturas e Mestrados à DGES como consequência de novos financiamentos de CTESPs e da avaliação da A3ES de Licenciaturas e Mestrados.}
\end{itemize}
\subsection{Atividades de Extensão}
\begin{itemize}
\item{Emissão de parecer sobre o período experimental de uma Prof. Adjunta do IPVC.}
\item{Participação no Projecto Cognita para apoio científico ao desenvolvimento de software da empresa JCANAO de Viana do Castelo durante o ano de 2019.}
\item{Responsável pela interação da ESTiG com a Altice Labs com o objectivo de dinamizar o protocolo estabelecido entre as duas entidades em novembro de 2018.}
\item{Responsável pela interação da ESTiG com a empresa Faurecia durante os anos letivos 2014-2015 e2015-2016 (
\href{run:MissaoIPBoutros/FAURECIA.pdf}{Declaração 4.3.1}), no que diz respeito às seguintes atividades:}
\begin{itemize}
\item{realização de reuniões preparatórias para recolha de possíveis projetos a serem desenvolvidos em parceria;}
\item{realização de sessões de divulgação desses projetos aos alunos da ESTiG;}
\item{atribuição desses projetos a disciplinas/docentes ou a alunos finalistas para serem desenvolvidos como projetos de fim de curso;}
\item{estabelecimento de acordos de formação, planos de trabalhos e horários de entrada e saída da fábrica;}
\item{realização de reuniões periódicas para verificação do estado de desenvolvimento dos projetos;}
\item{organização de sessões na ESTiG e na Faurecia de apresentação de trabalhos;}
\item{receção de pessoal técnico da Faurecia para realização de palestras no âmbito de eventos realizados pela ESTiG;}
\item{receção de pessoal técnico da Faurecia para lecionação de aulas no âmbito de disciplinas específicas;}
\item{divulgação de estágios profissionais a realizar na Faurecia.}
\end{itemize}
\item{Responsável pela organização do terceiro dia da Semana da Tecnologia e Gestão (STG 2015) da ESTiG com o tema {
\em{ Cooperação com empresas e Empregabilidade}} aberto ao público em geral em 23 de Abril de 2015 (
\href{run:MissaoIPBoutros/STG2015.pdf}{Declaração 4.3.2}).}
\end{itemize}
\subsection{Atividades de Serviço à Comunidade}
\begin{itemize}
\item{Lecionação de aulas de Introdução à Informática à Universidade Sénior de Bragança durante o segundo semestre do ano letivo 2006/2007 (30h), na Escola Superior de Tecnologia e Gestão de Bragança (
\href{run:MissaoIPBoutros/univSenior.pdf}{Comprovativo 4.4}).}
\end{itemize} 
\subsection{Atividades de formação contínua de profissionais}
\begin{itemize}
\item{Lecionação de um curso de Ms-Project - Gestão de Projetos (15 horas) na Escola Superior de Tecnologia e Gestão de Bragança, Dezembro de 2012 (
\href{run:MissaoIPBoutros/msproject2012.pdf}{Comprovativo 4.5.1}).}
\item{Lecionação de um curso de Ms-Project - Gestão de Projetos (10 horas) na Escola Superior de Tecnologia e Gestão de Bragança, Novembro 2008 (
\href{run:MissaoIPBoutros/msproject2008.pdf}{Comprovativo 4.5.2}).}
\item{Lecionação de um Módulo de Ms-Project - Gestão de Projetos (15 horas) a um grupo de engenheiros da empresa FAURECIA, em 2005, na Escola Superior de Tecnologia e Gestão de Bragança (
\href{run:MissaoIPBoutros/CursoFaurecia.pdf}{Comprovativo 4.5.3}).}
\item{Lecionação de um Mini-curso de Java (4h) aos professores de informática do 3o ciclo (grupo 39) em profissionalização em serviço na Universidade do Minho, Instituto de Estudos da Criança, Maio de 1998.}
\end{itemize}
\subsection{Atividades de participação em projetos e ações de interesse social}
\begin{itemize}
\item{Participação no projeto Ciência@Bragança, nº 16911, financiado pelo Ciência Viva - Agência Nacional para a Cultura Científica e Tecnológica (
\href{run:MissaoIPBoutros/CienciaBraganca.pdf}{Comprovativo 4.6.1}).}
\item{Participação em Atividades de Divulgação do IPB em Escolas Secundárias em 2007 e 2009 (
\href{run:MissaoIPBoutros/divulgacao.pdf}{Comprovativo 4.6.2}).}
\end{itemize}
\end{document}